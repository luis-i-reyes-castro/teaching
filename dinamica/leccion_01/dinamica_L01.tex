% =============================================
% =============================================
% Document class: Article
\documentclass[ a4paper, twoside, 11pt]{article}
% Packages: LaTeX (Depth-1)
\usepackage[ vlined, linesnumbered, ruled]{algorithm2e}
\usepackage{ amsfonts, amsmath, amssymb, amsthm}
\usepackage[ titletoc, title]{appendix}
\usepackage{ bbm}
\usepackage{ color}
\usepackage{ dsfont}
\usepackage{ enumitem}
\usepackage{ graphicx}
\usepackage{ fancyhdr, float, fullpage}
\usepackage{ hyperref}
\usepackage{ lastpage, latexsym, lipsum}
\usepackage{ mathrsfs, mathtools, multicol}
\usepackage{ parskip}
\usepackage{ setspace, stmaryrd, subcaption}
\usepackage{ tabularx}
\usepackage{ wasysym}
\usepackage[ dvipsnames, table]{ xcolor}
\usepackage{ xfrac}
% Packages: LaTeX (Depth-2)
\usepackage{ epstopdf}

% =============================================
\topmargin 			= -1.6cm
\headheight 		= .90cm
\headsep 			= .80cm
\textheight 		= 24.0cm
\textwidth 			= 15.5cm
\oddsidemargin		= 0.cm
\evensidemargin 	= 0.cm

% =============================================
% =============================================
% Macros: Language
\newcommand{\define}{\triangleq}
\newcommand{\done}{\hfill $\square$}
%\newcommand{\eqCIRC}{\stackrel{\circ}{=}}
%\newcommand{\eqSTAR}{\stackrel{*}{=}}
\renewcommand{\epsilon}{\varepsilon}
\newcommand{\eg}{\textit{e.g.,\;}}
\newcommand{\egc}{\textit{e.g.:\;}}
\newcommand{\Eg}{\textit{E.g.,\;}}
\newcommand{\Egc}{\textit{E.g.:\;}}
\newcommand{\ie}{\textit{i.e.,\;}}
\newcommand{\iec}{\textit{i.e.:\;}}
\newcommand{\Ie}{\textit{I.e.,\;}}
\newcommand{\Iec}{\textit{I.e.:\;}}
\newcommand{\QED}{\hfill $\blacksquare$}
\renewcommand{\tilde}[1]{\widetilde{#1}}
\newcommand{\tsup}[1]{\ensuremath{^{\text{#1}}}}
\newcommand{\tsub}[1]{\ensuremath{_{\text{#1}}}}
\renewcommand{\vec}[1]{{\boldsymbol{#1}}}

% Macros: Optimization & Probability
\DeclareMathOperator*{\argmax}{arg\,max}
\DeclareMathOperator*{\argmin}{arg\,min}
\newcommand{\Exp}{\mathbb{E}}
\newcommand{\Indicate}[1]{ \IndFun \, \{ \, #1 \, \} }
\renewcommand{\Pr}{\mathbb{P}}
\newcommand{\Normal}{\mathcal{N}}
\newcommand{\std}{\text{std}}
\newcommand{\var}{\text{var}}

% Macros: Sets
\newcommand{\Complex}{\mathbb{C}}
\renewcommand{\emptyset}{\varnothing}
\newcommand{\Nat}{\mathbb{N}}
\renewcommand{\Re}{\mathbb{R}}
\newcommand{\ReNN}{{\Re}_{\geq 0}}
\newcommand{\ReSP}{{\Re}_{> 0}}
\renewcommand{\subset}{\subseteq}
\renewcommand{\supset}{\supseteq}
\newcommand{\Z}{\mathbb{Z}}
\newcommand{\ZNN}{{\Z}_{\geq 0}}

% Macros: Spacing & Other Commands
\newcommand{\fullcut}{\vspace{-\baselineskip}}
\newcommand{\fullskip}{\vspace{\baselineskip}}
\newcommand{\halfcut}{\vspace{-0.5\baselineskip}}
\newcommand{\halfskip}{\vspace{0.5\baselineskip}}
\renewcommand{\figurename}{Figura}
\renewcommand{\tablename}{Tabla}

% =============================================
% Sesion de Clase
\newcommand{\sesion}{01}
% Macros para definiciones, teoremas, etc
\newcounter{sesion}
\setcounter{sesion}{\sesion}
\theoremstyle{definition}
\newtheorem{definition}{Definici\'on}[sesion]
\newtheorem{example}[definition]{Ejemplo}
\newtheorem{exercise}[definition]{Ejercicio}
\newtheorem{note}[definition]{Nota}
\newtheorem{problem}[definition]{Problema}
\newtheorem{theorem}[definition]{Teorema}

% =============================================
% =============================================
\newcommand{\HeaderLine}{}
\newcommand{\FooterLine}{P\'agina \thepage ~de \pageref*{LastPage}}

\pagestyle{fancyplain}
\fancyhf{}

\rhead[]{\fancyplain{}{\HeaderLine}}
\lhead[\fancyplain{}{\HeaderLine}]{}
\lfoot[\fancyplain{}{\FooterLine}]{}
\rfoot[]{\fancyplain{}{\FooterLine}}

\renewcommand{\headrulewidth}{0.4pt}
\renewcommand{\footrulewidth}{0.4pt}
\renewcommand{\thefootnote}{\fnsymbol{footnote}}

% =============================================
% =============================================
\begin{document}
\allowdisplaybreaks

\begin{center}
\Large Din\'amica (FIMCP-01271): Lecci\'on \sesion \\[0.5ex]
\small \textbf{A\~no:} 2016-2017 \qquad \textbf{T\'ermino:} II \qquad
\textbf{Instructor:} Luis I. Reyes Castro \qquad \textbf{Paralelo:} 02
\end{center}
\halfskip

\fbox{

\begin{minipage}[b][\height][t]{\textwidth}
\vspace{0.2 cm}

\begin{center}
\textbf{COMPROMISO DE HONOR}
\end{center}
\vspace{0.4 cm}

\scriptsize
{
Yo, \rule{60mm}{.1pt} al firmar este compromiso, reconozco que la presente lecci\'on est\'a dise\~nada para ser resuelta de manera individual, que puedo usar un l\'apiz o pluma y una calculadora cient\'ifica, \linebreak que solo puedo comunicarme con la persona responsable de la recepci\'on de la lecci\'on, y que cualquier instrumento de comunicaci\'on que hubiere tra\'ido debo apagarlo. Tambi\'en estoy conciente que no debo consultar libros, notas, \linebreak ni materiales did\'acticos adicionales a los que el instructor entregue durante la lecci\'on o autorice a utilizar. Finalmente, me comprometo a desarrollar y presentar mis respuestas de manera clara y ordenada. \\

Firmo al pie del presente compromiso como constancia de haberlo le\'ido y aceptado. 
\vspace{0.4 cm}

Firma: \rule{60mm}{.1pt} \qquad N\'umero de matr\'icula: \rule{40mm}{.1pt} \hspace{0.5cm} \\[-0.8ex]

}

\end{minipage}

}
\fullskip

% =============================================
\begin{problem}
La aceleraci\'on de una part\'icula est\'a definida por la relaci\'on $a = -K \sqrt{v}$, donde $K$ es una constante. Si se sabe que en $t = 0$ tenemos $x = 0$ m y $v = 81$ m/s, y que cuando $x = 18$ m tenemos $v = 36$ m/s, determine: 
\begin{enumerate}[label=\alph*.]
\item \textbf{[3 Puntos]} Una expresi\'on matem\'atica que relacione la posici\'on $x(t)$, la velocidad $v(t)$ y la constante $K$. \\ \emph{Soluci\'on:}
\begin{itemize}
\item Como $a = v \, \frac{dv}{dx}$, tenemos: 
\begin{align*}
v \, \frac{dv}{dx} \; = \; -K v^{1/2} \quad
& \Longrightarrow \quad 
v^{1/2} \, dv \; = \; -K \, dx \\
& \Longrightarrow \quad 
\frac{2}{3} \, ( v(t)^{3/2} - v(0)^{3/2} ) 
\; = \; -K \, ( x(t) - x(0) )
\end{align*}
\item Dado que $x(0) = 0$ m y $v(0) = 81$ m/s, tenemos: 
\[
\frac{2}{3} \, ( v(t)^{3/2} - 81^{3/2} ) 
\; = \; -K \, x(t) \quad \Longleftrightarrow \quad
v(t)^{3/2} \; = \; -\frac{3K}{2} \, x(t) + 81^{3/2}
\]
\end{itemize}
\item \textbf{[2 Puntos]} El valor de $K$. \\ \emph{Soluci\'on:}
\begin{itemize}
\item Como $x(t) = 18$ m cuando $v(t) = 36$ m/s, tenemos: 
\[
\frac{2}{3} \, ( 36^{3/2} - 81^{3/2} ) 
\; = \; -18K \quad \Longrightarrow \quad K \; = \; 19
\]
\end{itemize}
\item \textbf{[1 Punto]} La velocidad de la part\'icula cuando $x = 20$ m. \\ \emph{Soluci\'on:}
\begin{itemize}
\item Como $x(t) = 20$ m, tenemos: 
\[
v(t)^{3/2} \; = \; -\frac{(3)(19)}{2} \, (20) + 81^{3/2} \quad \Longrightarrow \quad 
v(t) \; = \; 159^{2/3} \text{ m/s} \; = \; 29.35 \text{ m/s}
\]
\end{itemize}
\item \textbf{[3 Puntos]} Una expresi\'on matem\'atica que relacione la velocidad $v(t)$ con el tiempo $t$. \\ \emph{Soluci\'on:}
\begin{itemize}
\item Como $a = \frac{dv}{dt}$, tenemos: 
\begin{align*}
\frac{dv}{dt} \; = \; -K v^{1/2} \quad
& \Longrightarrow \quad 
v^{-1/2} \, dv \; = \; -K \, dt \\
& \Longrightarrow \quad 
2 \, ( v(t)^{1/2} - v(0)^{1/2} ) \; = \; -K t \\
& \Longrightarrow \quad 
v(t) \; = \; \left( \sqrt{v(0)} - \frac{1}{2} \, K t \, \right)^2
\end{align*}
\item Dado que $v(0) = 81$ m/s y que $K = 19$, tenemos: 
\[
v(t) \; = \; 
\left( 9 - \frac{19}{2} \, t \, \right)^2
\]
\end{itemize}
\item \textbf{[1 Punto]} El tiempo requerido para que la part\'icula quede en reposo. \\ \emph{Soluci\'on:}
\begin{itemize}
\item De la \'ultima expresi\'on, vemos que $v(t) = 0$ cuando: 
\[
9 - \frac{19}{2} \, t \; = \; 0 \quad \Longleftrightarrow \quad
t \; = \; 0.947 \text{ s}
\]
\end{itemize}
\end{enumerate}

\end{problem}
\vspace{\baselineskip}

\end{document}
