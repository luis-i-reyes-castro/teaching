% -----------------------------------------------------------------
% Document class: Article
\documentclass[ a4paper, twoside, 11pt]{article}
\usepackage{../../macros-general}
\usepackage{../../macros-article}
% Number of the handout, quiz, exam, etc.
\newcommand{\numero}{01}
\setcounter{numero}{\numero}

% -----------------------------------------------------------------
\begin{document}
\allowdisplaybreaks

% Indices
\newcommand{\iava}{$i$\tsup{ava} }
\newcommand{\iavo}{$i$\tsup{avo} }
\newcommand{\java}{$j$\tsup{ava} }
\newcommand{\javo}{$j$\tsup{avo} }
\newcommand{\kava}{$k$\tsup{ava} }
\newcommand{\kavo}{$k$\tsup{avo} }
\newcommand{\tava}{$t$\tsup{ava} }
\newcommand{\tavo}{$t$\tsup{avo} }
\newcommand{\tmava}{$(t-1)$\tsup{ava} }
\newcommand{\tmavo}{$(t-1)$\tsup{avo} }
\newcommand{\tMava}{$(t+1)$\tsup{ava} }
\newcommand{\tMavo}{$(t+1)$\tsup{avo} }

\begin{center}
\Large Programaci\'on Entera (INDG-1019): Gu\'ia de Estudio \numero \\[1ex]
\small \textbf{Semestre:} 2018-2019 T\'ermino I \qquad
\textbf{Instructor:} Luis I. Reyes Castro
\end{center}
\fullskip

% -----------------------------------------------------------------
\begin{problem}
Considere un programa entero que contiene $m \geq 10$ variables binarias
\[
x_1, \, x_2, \, x_3, \, \dots, \, x_m \in \{ 0, 1 \}
\]
junto con otras variables adicionales. En cada uno de los siguientes literales, se le presentar\'a un conjunto de suposiciones seguido de una regla o preferencia. Exprese cada una de las reglas o preferencias en el lenguaje de la programaci\'on lineal entera mediante la introducci\'on de variables enteras y/o de restricciones lineales. Por favor considere que cada literal es independiente de todos los otros. 
\begin{enumerate}[label=\alph*)]
% -----------------------------------------------------------------
\item \emph{Suposiciones:} $2 \leq n < m$ ; \, $y \in \{ 0, 1\}$. \\
\emph{Regla o preferencia:} Si el n\'umero de \'indices $i \in \upto{m}$ tales que $x_i = 1$ es igual o mayor a $n$ entonces $y = 1$. \\[1ex]
\emph{Soluci\'on}: Introducimos la siguiente restricci\'on lineal: 
\[
\sum_{i=1}^m x_i \; \leq \; (n-1) + (m-n+1) \, y
\]

\item \emph{Suposiciones:} $2 \leq n < m$ ; \, $y \in \{ 0, 1\}$. \\
\emph{Regla o preferencia:} Si el n\'umero de \'indices $i \in \upto{m}$ tales que $x_i = 1$ es menor o igual a $n$ entonces $y = 1$. 

\item \emph{Suposiciones:} $2 \leq n < m$ ; \, $y \in \{ 0, 1\}$. \\
\emph{Regla o preferencia:} La variable $y = 1$ si y solo si el n\'umero de \'indices $i \in \upto{m}$ tales que $x_i = 1$ es igual o mayor a $n$. 

\item \emph{Suposiciones:} $2 \leq n < m$ ; \, $y \in \{ 0, 1\}$. \\
\emph{Regla o preferencia:} La variable $y = 1$ si y solo si el n\'umero de \'indices $i \in \upto{m}$ tales que $x_i = 1$ es exactamente igual a $n$. 

% -----------------------------------------------------------------
\item \emph{Suposiciones:} $z \in \{ 0, 1\}$ ; \, $2 \leq n < m$. \\
\emph{Regla o preferencia:} Si $z = 1$ entonces el n\'umero de \'indices $i \in \upto{m}$ tales que $x_i = 1$ debe ser igual o mayor a $n$. 

\item \emph{Suposiciones:} $z \in \{ 0, 1\}$ ; \, $2 \leq n < m$. \\
\emph{Regla o preferencia:} Si $z = 1$ entonces el n\'umero de \'indices $i \in \upto{m}$ tales que $x_i = 1$ debe ser exactamente igual a $n$. 

% -----------------------------------------------------------------
\item \emph{Suposiciones:} $S \subset \upto{m}$ es un subconjunto de \'indices; \, $w \in \{ 0, 1\}$. \\
\emph{Regla o preferencia:} Si para cualquier \'indice $i \in S$ tenemos $x_i = 1$ entonces $w = 1$. \\[1ex]
\emph{Soluci\'on}: Introducimos el siguiente juego de restricciones lineales: 
\[
\forall \, i \in S \; \colon \; x_i \leq w
\]

\item \emph{Suposiciones:} $S \subset \upto{m}$ es un subconjunto de \'indices; \, $w \in \{ 0, 1\}$. \\
\emph{Regla o preferencia:} Si para todo \'indice $i \in S$ tenemos $x_i = 1$ entonces $w = 1$. 

\item \emph{Suposiciones:} $S \subset \upto{m}$ es un subconjunto de \'indices; \, $w \in \{ 0, 1\}$. \\
\emph{Regla o preferencia:} La variable $w = 1$ si y solo si para todo \'indice $i \in S$ tenemos $x_i = 1$. 

% -----------------------------------------------------------------
\item \emph{Suposiciones:} $S, T \subset \upto{m}$ son subconjuntos de \'indices. \\
\emph{Regla o preferencia:} El n\'umero de \'indices $i \in S$ tales que $x_i = 1$ es mayor o igual al n\'umero de \'indices $j \in T$ tales que $x_j = 1$. \\[1ex]
\emph{Soluci\'on:} Introducimos la siguiente restricci\'on lineal: 
\[
\sum_{i \in S} x_i \; \geq \; \sum_{j \in S} x_j
\]

\item \emph{Suposiciones:} $S, T \subset \upto{m}$ son subconjuntos de \'indices; \, $w \in \{ 0, 1\}$. \\
\emph{Regla o preferencia:} Si el n\'umero de \'indices $i \in S$ para los cuales $x_i = 1$ es igual o mayor al n\'umero de \'indices $j \in S$ para los cuales $x_j = 1$ entonces $w = 1$. 

\item \emph{Suposiciones:} $S, T \subset \upto{m}$ son subconjuntos de \'indices. \\
\emph{Regla o preferencia:} Si $x_i = 1$ para al menos un \'indice $i \in S$ entonces $x_j = 1$ para al menos un \'indice $j \in T$. 

\item \emph{Suposiciones:} $S, T \subset \upto{m}$ son subconjuntos de \'indices. \\
\emph{Regla o preferencia:} Si $x_i = 1$ para al menos un \'indice $i \in S$ entonces $x_j = 0$ para todo \'indice $j \in T$. 

\item \emph{Suposiciones:} $S, T \subset \upto{m}$ son subconjuntos de \'indices. \\
\emph{Regla o preferencia:} Si $x_i = 1$ para al menos un \'indice $i \in S$ entonces $x_j = 0$ para todo \'indice $j \in T$, y vice-versa (\ie si $x_j = 1$ para al menos un \'indice $j \in T$ entonces $x_i = 0$ para todo \'indice $i \in S$). 

\end{enumerate}
\QED

\end{problem}
\vspace{\baselineskip}

% -----------------------------------------------------------------
\begin{problem}
Considere un programa entero que contiene las series temporales de variables binarias $\{x_t\}_{t=1}^T \in \{ 0, 1 \}$, $\{y_t\}_{t=1}^T \in \{ 0, 1 \}$ y $\{z_t\}_{t=1}^T \in \{ 0, 1 \}$, donde $T \geq 10$, junto con otras variables adicionales. En este modelo, las series temporales $\{x_t\}$, $\{y_t\}$ y $\{z_t\}$ representan la ocurrencia o no-ocurrencia de tres tipos diferentes de eventos de inter\'es en un problema de planificaci\'on con un horizonte de $T$ per\'iodos. 

En cada uno de los siguientes literales, se le presentar\'a una regla o preferencia, posiblemente precedida por un conjunto de suposiciones. Exprese cada una de las reglas o preferencias en el lenguaje de la programaci\'on lineal entera mediante la introducci\'on de variables enteras y/o de restricciones lineales. Recuerde que cada literal es independiente de los otros. 
\begin{enumerate}[label=\alph*)]
% -----------------------------------------------------------------
\item \emph{Regla o preferencia:} Si para alg\'un periodo $k \in \upto{T-q}$ tenemos $x_k = 1$ entonces $x_{k+\ell} = 1$ para todo $\ell \geq 1$. \\[1ex]
\emph{Soluci\'on}: Introducimos el siguiente juego de restricciones lineales: 
\[
\forall \, t \in \upto{T-1} \; \colon \; x_t \leq x_{t+1}
\]

\item \emph{Regla o preferencia:} Existe al menos un periodo $k \in \upto{T-1}$ tal que $x_k = 1$ y $x_{k+1} = 1$. 

\item \emph{Suposiciones:} $2 \leq M < T$. \\
\emph{Regla o preferencia:} Existe al menos un periodo $k \in \upto{T-M}$ tal que $x_{k+\ell} = 1$ para todo periodo $\ell \in \{ 0, 1, \dots, M-1 \}$. 

\item \emph{Regla o preferencia:} Si para alg\'un periodo $k \in \upto{T-1}$ tenemos $x_k = 1$ entonces $y_{k+\ell} = 1$ para al menos un $\ell \geq 1$. 

\item \emph{Regla o preferencia:} Si para alg\'un periodo $k \in \upto{T-1}$ tenemos $x_k = 1$ entonces $y_{k+1} = 1$. 

\item \emph{Suposiciones:} $2 \leq M < T$. \\
\emph{Regla o preferencia:} Si para alg\'un periodo $k \in \upto{T-M}$ tenemos $x_k = 1$ entonces $y_{k+\ell} = 1$ para todo $\ell \in \{ 1, 2, \dots, M \}$. 

\item \emph{Regla o preferencia:} Si para alg\'un periodo $k \in \upto{T-2}$ tenemos $x_k = 1$ y $y_{k+1} = 1$ entonces $z_{k+2} = 1$. 

\item \emph{Regla o preferencia:} Si para alg\'un periodo $k \in \upto{T-1}$ tenemos $x_k = 1$ y $y_{k} = 1$ entonces $z_{k+\ell} = 1$ para todo $\ell \geq 1$. 

\item \emph{Regla o preferencia:} Si para alg\'un periodo $k \in \upto{T-1}$ tenemos $x_k = 1$ o $y_{k} = 1$ entonces $z_{k+ \ell} = 1$ para al menos un $\ell \geq 1$. 

\end{enumerate}
\QED

\end{problem}
\vspace{\baselineskip}

\end{document}
