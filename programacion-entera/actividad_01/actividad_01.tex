% -----------------------------------------------------------------
% Document class: Article
\documentclass[ a4paper, twoside, 11pt]{article}
\usepackage{../../macros-general}
\usepackage{../../macros-article}
% Number of the handout, quiz, exam, etc.
\newcommand{\numero}{01}
\setcounter{numero}{\numero}

% -----------------------------------------------------------------
\begin{document}
\allowdisplaybreaks

% Indices
\newcommand{\iava}{$i$\tsup{ava} }
\newcommand{\iavo}{$i$\tsup{avo} }
\newcommand{\java}{$j$\tsup{ava} }
\newcommand{\javo}{$j$\tsup{avo} }
\newcommand{\kava}{$k$\tsup{ava} }
\newcommand{\kavo}{$k$\tsup{avo} }
\newcommand{\tava}{$t$\tsup{ava} }
\newcommand{\tavo}{$t$\tsup{avo} }
\newcommand{\tmava}{$(t-1)$\tsup{ava} }
\newcommand{\tmavo}{$(t-1)$\tsup{avo} }
\newcommand{\tMava}{$(t+1)$\tsup{ava} }
\newcommand{\tMavo}{$(t+1)$\tsup{avo} }

\begin{center}
\Large Programaci\'on Entera (INDG-1019): Actividad \numero \\[1ex]
\small \textbf{Semestre:} 2018-2019 T\'ermino I \qquad
\textbf{Instructor:} Luis I. Reyes Castro
\end{center}
\halfskip

\textbf{Instrucciones:}

Suponga que la Universidad XYZ (U-XYZ) contrata profesores no-titulares a tiempo medio, tiempo parcial o tiempo completo. Los profesores a tiempo medio dictan dos materias, mientras que los profesores a tiempo parcial y tiempo completo dictan tres y cuatro materias, respectivamente. Suponga adem\'as que usted, reci\'en graduado de su maestr\'ia en Ingenier\'ia Industrial de una presitigiosa universidad de Estados Unidos, desea aplicar a un puesto como profesor no-titular. Ahora, para su suerte o desgracia, la U-XYZ es administrada por el malvado Prof. Luis I. Reyes Castro, quien, aun despues de haber liquidado al departamento de Pedagog\'ia y de haber despedido a la mitad del departamento de recursos humanos, sigue obsesionado con recortar todo tipo de costos, incluyendo los de planta docente. 

Por estos motivos, el Prof. Reyes ha decidio implementar una subasta combinatorial para seleccionar a los docentes que necesita para dictar las 25 materias que se ofrecen en el departamento de Ingenier\'ia Industrial. Para participar de la subasta, por favor descargue el la hoja de c\'alculo \texttt{Actividad\_Subastas-combinatoriales.xlsx}, \'abralo, y desarrolle tres propuestas. En particular, usted debe presentar: 

\begin{enumerate}
\item Una propuesta de dos materias como profesor a tiempo medio, junto con una propuesta de salario por materia. Para esto por favor refi\'erase a la columna Tiempo Medio en la hoja de c\'alculo y coloque un n\'umero uno en las filas correspondientes a su selecci\'on de materias. Luego, en la \'ultima fila, especifique su salario por materia deseado. 
\item Una propuesta de tres materias como profesor a tiempo parcial, junto con una \linebreak propuesta de salario por materia. Para presentarla por favor siga un proceso similar al de la propuesta como profesor a tiempo medio. 
\item Una propuesta de cuatro materias como profesor a tiempo completo, junto con una \linebreak propuesta de salario por materia. Para presentarla por favor siga un proceso similar al de la propuesta como profesor a tiempo medio. 
\end{enumerate}

Al escoger sus materias, tenga en cuenta que si escoge materias demasiado populares estar\'a compitiendo contra muchos otros de sus compa\~neros. Al escoger su salario por materia, tenga en cuenta que el malvado Prof. Reyes no est\'a dispuesto a pagar m\'as de \$500.00 por materia, y se sobre-entiende que mientras m\'as bajo sea su salario por materia deseado, m\'as atractiva ser\'a su propuesta. 

Finalmente, una vez desarrolladas sus propuestas, s\'ubalas a SIDWeb como si fueren entregas del trabajo ficticio ``Subasta Combinatoriales'' de acuerdo al siguiente formato de nombre de archivo: 
\begin{center}
\texttt{Propuestas\_Apellido1\_Apellido2\_Nombre1\_Nombre2.xlsx}
\end{center}

Por ejemplo, el instructor presentar\'ia sus propuestas en el archivo: 
\begin{center}
\texttt{Propuestas\_Reyes\_Castro\_Luis\_Ignacio.xlsx}
\end{center}

Una vez recolectadas todas las propuestas, el Prof. Reyes resolver\'a un problema de cobertura de conjuntos para escoger a los profesores que cubren su demanda al menor costo. 

\emph{Nota:} Se le otorgar\'a un punto directo a la calificaci\'on de su Primera Evaluaci\'on a los cinco estudiantes que logren ser `contratados' al mayor salario por materia. 
 

\end{document}
