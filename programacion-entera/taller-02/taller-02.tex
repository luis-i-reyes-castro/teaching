% -----------------------------------------------------------------
% Document class: Article
\documentclass[ a4paper, twoside, 11pt]{article}
\usepackage{../../macros-general}
\usepackage{../../macros-article}
% Number of the handout, quiz, exam, etc.
\newcommand{\numero}{02}
\setcounter{numero}{\numero}

% -----------------------------------------------------------------
\begin{document}
\allowdisplaybreaks

% Indices
\newcommand{\iava}{$i$\tsup{ava} }
\newcommand{\iavo}{$i$\tsup{avo} }
\newcommand{\java}{$j$\tsup{ava} }
\newcommand{\javo}{$j$\tsup{avo} }
\newcommand{\kava}{$k$\tsup{ava} }
\newcommand{\kavo}{$k$\tsup{avo} }
\newcommand{\tava}{$t$\tsup{ava} }
\newcommand{\tavo}{$t$\tsup{avo} }
\newcommand{\tmava}{$(t-1)$\tsup{ava} }
\newcommand{\tmavo}{$(t-1)$\tsup{avo} }
\newcommand{\tMava}{$(t+1)$\tsup{ava} }
\newcommand{\tMavo}{$(t+1)$\tsup{avo} }

\begin{center}
\Large Programaci\'on Entera (INDG-1019): Taller \numero \\[1ex]
\small \textbf{Semestre:} 2018-2019 T\'ermino I \qquad
\textbf{Instructor:} Luis I. Reyes Castro
\end{center}
\fullskip

% -----------------------------------------------------------------
\begin{problem}
El due\~no de un nuevo centro comercial ha recibido ofertas por parte de varias empresas interesadas en alquilar locales comerciales. En particular: 
\begin{itemize}
\item El centro comercial tiene $p$ pisos, y cada piso puede albergar hasta $\ell$ locales comerciales. 
\item Existen $m$ empresas diferentes interesados en alquilar locales comerciales en el centro. Para cada empresa $i \in \upto{p}$ y cada piso $j \in \upto{p}$ denotamos al precio ofertado por esa empresa para alquilar un local en ese piso como $u_{ij}$. Las empresas se clasifican de acuerdo a su tipo de negocio: 

\begin{table}[htb]
\centering
\begin{tabular}{|l|c|}
\hline
\multicolumn{1}{|c|}{\textbf{Tipo de Negocio}} & \textbf{S\'imbolo} \\ \hline
Ropa                                           & RP                 \\ \hline
Muebles o Electrodom\'esticos                  & ME                 \\ \hline
Bienes Inmuebles                               & BI                 \\ \hline
Deportes y Salud                               & DS                 \\ \hline
Lectura y Arte                                 & LA                 \\ \hline
Comida                                         & C                  \\ \hline
Banco (Servicios Bancarios)                    & BK                 \\ \hline
Servicios al Cliente o T\'ecnicos              & SCT                \\ \hline
\end{tabular}
\end{table}

\end{itemize}

El due\~no del negocio desea maximizar sus ganancias por alquiler de locales comerciales sujeto a las siguiente restricciones: 
\begin{enumerate}[label=\textbf{\alph*)}]
\item En todo piso donde haya tres o m\'as locales de servicios al cliente o t\'ecnicos (SCT) debe haber al menos un banco (BK). 
\item En todo piso donde haya cinco o m\'as locales de ropa (R) debe haber al menos un local de deportes y salud (DS) y un local de lectura y artes (LA). 
\item En ning\'un piso pueden haber locales de comida (C) y bancos (BK). 
\item Todos los locales de comida (C) deben estar concentrados en el mismo piso, el cual pasar\'a a contener la Plaza de Comidas del centro comercial. 
\item En todo piso donde haya al menos un local de bienes inmuebles (BI) debe haber al menos \textit{(i)} tres locales de muebles o electrodom\'esticos (ME) y un banco (BK), o \textit{(ii)} dos locales de ropa (R) y dos locales de lectura y arte (LA). 
\end{enumerate}

Con todo esto en mente, escriba el problema de decisi\'on del due\~no del centro comercial como un Programa Lineal Entero (PLE). \QED

% =========================================================
\emph{Soluci\'on A:} Primero definimos una variable binaria para cada piso $i \in \upto{p}$ y empresa $j \in \upto{m}$, denotada $x_{ij}$, que toma el valor uno si y solo si en ese piso se le alquila un local a esa empresa. Segundo, modelamos las restricciones impuestas: 
\begin{enumerate}[label=\textbf{\alph*)}]
% ---------------------------------------------------------
\item \textbf{[Literal anulado porque fue utilizado de ejemplo]} \\[1ex]
\emph{Restricci\'on:} En todo piso donde haya tres o m\'as locales de servicios al cliente o t\'ecnicos (SCT) debe haber al menos un banco (BK). \\[1ex]
\emph{Implementaci\'on:}
\begin{itemize}
\item Introducimos una variable binaria indicadora para cada piso: 
\[
\forall \, i \in \upto{p} \; \colon \; a_i \in \{ 0, 1 \}
\]
\item Introducimos una restricci\'on para cada piso $i \in \upto{p}$ tal que si $x_{ij} = 1$ para tres o m\'as empresas $j \in \text{SCT}$ entonces $a_i = 1$. M\'as precisamente: 
\[
\forall \, i \in \upto{p}, \, \forall \, j \in \text{SCT} \; \colon \;
x_{ij} \; \leq \; 2 + (\ell - 2) \, a_i
\]
\item Introducimos una restricci\'on para cada piso $i \in \upto{p}$ tal que si $a_i = 1$ entonces $x_{ij} = 1$ para al menos una empresa $j \in \text{BK}$. M\'as precisamente: 
\[
\forall \, i \in \upto{p} \; \colon \;
a_i \; \leq \; \sum_{j \in \text{BK}} x_{ij}
\]
\end{itemize}
% ---------------------------------------------------------
\item \textbf{[4 Puntos]} \\[1ex]
\emph{Restricci\'on:} En todo piso donde haya cinco o m\'as locales de ropa (R) debe haber al menos un local de deportes y salud (DS) y un local de lectura y artes (LA). \\[1ex]
\emph{Implementaci\'on:}
\begin{itemize}
\item Introducimos una variable binaria indicadora para cada piso: 
\[
\forall \, i \in \upto{p} \; \colon \; a_i \in \{ 0, 1 \}
\]
\item Introducimos una restricci\'on para cada piso $i \in \upto{p}$ tal que si $x_{ij} = 1$ para cinco o m\'as empresas $j \in \text{R}$ entonces $a_i = 1$. M\'as precisamente: 
\[
\forall \, i \in \upto{p}, \, \forall \, j \in \text{R} \; \colon \;
x_{ij} \; \leq \; 4 + (\ell - 4) \, a_i
\]
\item Introducimos una restricci\'on para cada piso $i \in \upto{p}$ tal que si $a_i = 1$ entonces $x_{ij} = 1$ para al menos una empresa $j \in \text{DS}$. M\'as precisamente: 
\[
\forall \, i \in \upto{p} \; \colon \;
a_i \; \leq \; \sum_{j \in \text{DS}} x_{ij}
\]
\item Introducimos una restricci\'on para cada piso $i \in \upto{p}$ tal que si $a_i = 1$ entonces $x_{ij} = 1$ para al menos una empresa $j \in \text{LA}$. M\'as precisamente: 
\[
\forall \, i \in \upto{p} \; \colon \;
a_i \; \leq \; \sum_{j \in \text{LA}} x_{ij}
\]
\end{itemize}
% ---------------------------------------------------------
\item \textbf{[4 Puntos]} \\[1ex]
\emph{Restricci\'on:} En ning\'un piso pueden haber locales de comida (C) y bancos (BK). \\[1ex]
\emph{Implementaci\'on:}
\begin{itemize}
\item Introducimos un par de variables binarias indicadoras para cada piso: 
\[
\forall \, i \in \upto{p} \; \colon \; a_i, b_i \in \{ 0, 1 \}
\]
\item Introducimos una restricci\'on o juego de restricciones para cada piso $i \in \upto{p}$ tal que si $x_{ij} = 1$ para al menos una empresa $j \in \text{C}$ entonces $a_i = 1$. Esto puede ser logrado de al menos dos maneras: 
\begin{itemize}
\item Introduciendo la siguiente restricci\'on: 
\[
\forall \, i \in \upto{p} \; \colon \;
\sum_{j \in \text{C}} x_{ij} \; \leq \; |C| \, a_i
\]
\item Introduciendo el siguiente juego de restricciones: 
\[
\forall \, i \in \upto{p}, \, \forall \, j \in \text{C} \; \colon \;
x_{ij} \; \leq \; a_i
\]
\end{itemize}
\item Introducimos una restricci\'on o juego de restricciones para cada piso $i \in \upto{p}$ tal que si $x_{ij} = 1$ para al menos una empresa $j \in \text{BK}$ entonces $b_i = 1$. Esto puede ser logrado de al menos dos maneras: 
\begin{itemize}
\item Introduciendo la siguiente restricci\'on: 
\[
\forall \, i \in \upto{p} \; \colon \;
\sum_{j \in \text{BK}} x_{ij} \; \leq \; |BK| \, b_i
\]
\item Introduciendo el siguiente juego de restricciones: 
\[
\forall \, i \in \upto{p}, \, \forall \, j \in \text{BK} \; \colon \;
x_{ij} \; \leq \; b_i
\]
\end{itemize}
\item Introducimos una restricci\'on para cada piso $i \in \upto{p}$ que impide que $a_i$ y $b_i$ tomen el valor uno simult\'aneamente. M\'as precisamente: 
\[
\forall \, i \in \upto{p} \; \colon \; a_i + b_i \; \leq \; 1
\]
\end{itemize}
% ---------------------------------------------------------
\item \textbf{[3 Puntos]} \\[1ex]
\emph{Restricci\'on:} Todos los locales de comida (C) deben estar concentrados en el mismo piso, el cual pasar\'a a contener la Plaza de Comidas del centro comercial. \\[1ex]
\emph{Implementaci\'on:}
\begin{itemize}
\item Introducimos una variable binaria indicadora para cada piso: 
\[
\forall \, i \in \upto{p} \; \colon \; a_i \in \{ 0, 1 \}
\]
\item Introducimos una restricci\'on o juego de restricciones para cada piso $i \in \upto{p}$ tal que si $x_{ij} = 1$ para al menos una empresa $j \in \text{C}$ entonces $a_i = 1$. Esto puede ser logrado de al menos dos maneras: 
\begin{itemize}
\item Introduciendo la siguiente restricci\'on: 
\[
\forall \, i \in \upto{p} \; \colon \;
\sum_{j \in \text{C}} x_{ij} \; \leq \; |C| \, a_i
\]
\item Introduciendo el siguiente juego de restricciones: 
\[
\forall \, i \in \upto{p}, \, \forall \, j \in \text{C} \; \colon \;
x_{ij} \; \leq \; a_i
\]
\end{itemize}
\item Introducimos una restricci\'on tal que $a_i = 1$ para exactamente un solo piso $i \in \upto{p}$. M\'as precisamente: 
\[
\sum_{i \in \upto{p}} a_i \; = \; 1
\]
\end{itemize}

% ---------------------------------------------------------
\item \textbf{[5 Puntos]} \\[1ex]
\emph{Restricci\'on:} En todo piso donde haya al menos un local de bienes inmuebles (BI) debe haber al menos \textit{(i)} tres locales de muebles o electrodom\'esticos (ME) y un banco (BK), o \textit{(ii)} dos locales de ropa (R) y dos locales de lectura y arte (LA). \\[1ex]
\emph{Implementaci\'on:}
\begin{itemize}
\item Introducimos un tr\'io de variables binarias indicadoras para cada piso: 
\[
\forall \, i \in \upto{p} \; \colon \; a_i, b_i, c_i \in \{ 0, 1 \}
\]
\item Introducimos una restricci\'on o juego de restricciones para cada piso $i \in \upto{p}$ tal que si $x_{ij} = 1$ para al menos una empresa $j \in \text{BI}$ entonces $a_i = 1$. Esto puede ser logrado de al menos dos maneras: 
\begin{itemize}
\item Introduciendo la siguiente restricci\'on: 
\[
\forall \, i \in \upto{p} \; \colon \;
\sum_{j \in \text{BI}} x_{ij} \; \leq \; |BI| \, a_i
\]
\item Introduciendo el siguiente juego de restricciones: 
\[
\forall \, i \in \upto{p}, \, \forall \, j \in \text{BI} \; \colon \;
x_{ij} \; \leq \; a_i
\]
\end{itemize}
\item Introducimos un par de restricciones para cada piso $i \in \upto{p}$ tal que si $b_i = 1$ entonces $x_{ij} = 1$ para al menos tres empresas $j \in \text{ME}$ y al menos una empresa $j \in \text{BK}$. \linebreak M\'as precisamente: 
\begin{align*}
& \forall \, i \in \upto{p} \; \colon \;
3 \, a_i \; \leq \; \sum_{j \in \text{ME}} x_{ij} \\
& \forall \, i \in \upto{p} \; \colon \;
1 \, a_i \; \leq \; \sum_{j \in \text{BK}} x_{ij}
\end{align*}
\item Introducimos un par de restricciones para cada piso $i \in \upto{p}$ tal que si $c_i = 1$ entonces $x_{ij} = 1$ para al menos dos empresas $j \in \text{R}$ y al menos dos empresas $j \in \text{LA}$. \linebreak M\'as precisamente: 
\begin{align*}
& \forall \, i \in \upto{p} \; \colon \;
2 \, a_i \; \leq \; \sum_{j \in \text{R}} x_{ij} \\
& \forall \, i \in \upto{p} \; \colon \;
2 \, a_i \; \leq \; \sum_{j \in \text{LA}} x_{ij}
\end{align*}
\item Introducimos una restricci\'on para cada piso $i \in \upto{p}$ tal que si $a_i = 1$ entonces $b_i = 1$ o $c_i = 1$. M\'as precisamente: 
\[
\forall \, i \in \upto{p} \; \colon \; a_i \; \leq \; b_i + c_i
\]
\end{itemize}

\end{enumerate}

% =========================================================
\emph{Soluci\'on B:} Primero definimos una variable binaria para cada empresa $i \in \upto{m}$ y piso $j \in \upto{p}$, denotada $x_{ij}$, que toma el valor uno si y solo si a esa empresa se le alquila un local en ese piso. Segundo, modelamos las restricciones impuestas: 
\begin{enumerate}[label=\textbf{\alph*)}]
% ---------------------------------------------------------
\item \textbf{[Literal anulado porque fue utilizado de ejemplo]} \\[1ex]
\emph{Restricci\'on:} En todo piso donde haya tres o m\'as locales de servicios al cliente o t\'ecnicos (SCT) debe haber al menos un banco (BK). \\[1ex]
\emph{Implementaci\'on:}
\begin{itemize}
\item Introducimos una variable binaria indicadora para cada piso: 
\[
\forall \, j \in \upto{p} \; \colon \; a_j \in \{ 0, 1 \}
\]
\item Introducimos una restricci\'on para cada piso $j \in \upto{p}$ tal que si $x_{ij} = 1$ para tres o m\'as empresas $i \in \text{SCT}$ entonces $a_j = 1$. M\'as precisamente: 
\[
\forall \, j \in \upto{p}, \, \forall \, i \in \text{SCT} \; \colon \;
x_{ij} \; \leq \; 2 + (\ell - 2) \, a_j
\]
\item Introducimos una restricci\'on para cada piso $j \in \upto{p}$ tal que si $a_j = 1$ entonces $x_{ij} = 1$ para al menos una empresa $i \in \text{BK}$. M\'as precisamente: 
\[
\forall \, j \in \upto{p} \; \colon \;
a_j \; \leq \; \sum_{i \in \text{BK}} x_{ij}
\]
\end{itemize}
% ---------------------------------------------------------
\item \textbf{[4 Puntos]} \\[1ex]
\emph{Restricci\'on:} En todo piso donde haya cinco o m\'as locales de ropa (R) debe haber al menos un local de deportes y salud (DS) y un local de lectura y artes (LA). \\[1ex]
\emph{Implementaci\'on:}
\begin{itemize}
\item Introducimos una variable binaria indicadora para cada piso: 
\[
\forall \, j \in \upto{p} \; \colon \; a_j \in \{ 0, 1 \}
\]
\item Introducimos una restricci\'on para cada piso $j \in \upto{p}$ tal que si $x_{ij} = 1$ para cinco o m\'as empresas $i \in \text{R}$ entonces $a_j = 1$. M\'as precisamente: 
\[
\forall \, j \in \upto{p}, \, \forall \, i \in \text{R} \; \colon \;
x_{ij} \; \leq \; 4 + (\ell - 4) \, a_j
\]
\item Introducimos una restricci\'on para cada piso $j \in \upto{p}$ tal que si $a_j = 1$ entonces $x_{ij} = 1$ para al menos una empresa $i \in \text{DS}$. M\'as precisamente: 
\[
\forall \, j \in \upto{p} \; \colon \;
a_j \; \leq \; \sum_{i \in \text{DS}} x_{ij}
\]
\item Introducimos una restricci\'on para cada piso $j \in \upto{p}$ tal que si $a_j = 1$ entonces $x_{ij} = 1$ para al menos una empresa $i \in \text{LA}$. M\'as precisamente: 
\[
\forall \, j \in \upto{p} \; \colon \;
a_j \; \leq \; \sum_{i \in \text{LA}} x_{ij}
\]
\end{itemize}
% ---------------------------------------------------------
\item \textbf{[4 Puntos]} \\[1ex]
\emph{Restricci\'on:} En ning\'un piso pueden haber locales de comida (C) y bancos (BK). \\[1ex]
\emph{Implementaci\'on:}
\begin{itemize}
\item Introducimos un par de variables binarias indicadoras para cada piso: 
\[
\forall \, j \in \upto{p} \; \colon \; a_j, b_j \in \{ 0, 1 \}
\]
\item Introducimos una restricci\'on o juego de restricciones para cada piso $j \in \upto{p}$ tal que si $x_{ij} = 1$ para al menos una empresa $i \in \text{C}$ entonces $a_j = 1$. Esto puede ser logrado de al menos dos maneras: 
\begin{itemize}
\item Introduciendo la siguiente restricci\'on: 
\[
\forall \, j \in \upto{p} \; \colon \;
\sum_{i \in \text{C}} x_{ij} \; \leq \; |C| \, a_j
\]
\item Introduciendo el siguiente juego de restricciones: 
\[
\forall \, j \in \upto{p}, \, \forall \, i \in \text{C} \; \colon \;
x_{ij} \; \leq \; a_j
\]
\end{itemize}
\item Introducimos una restricci\'on o juego de restricciones para cada piso $j \in \upto{p}$ tal que si $x_{ij} = 1$ para al menos una empresa $i \in \text{BK}$ entonces $b_j = 1$. Esto puede ser logrado de al menos dos maneras: 
\begin{itemize}
\item Introduciendo la siguiente restricci\'on: 
\[
\forall \, j \in \upto{p} \; \colon \;
\sum_{i \in \text{BK}} x_{ij} \; \leq \; |BK| \, b_j
\]
\item Introduciendo el siguiente juego de restricciones: 
\[
\forall \, j \in \upto{p}, \, \forall \, i \in \text{BK} \; \colon \;
x_{ij} \; \leq \; b_j
\]
\end{itemize}
\item Introducimos una restricci\'on para cada piso $j \in \upto{p}$ que impide que $a_j$ y $b_j$ tomen el valor uno simult\'aneamente. M\'as precisamente: 
\[
\forall \, j \in \upto{p} \; \colon \; a_j + b_j \; \leq \; 1
\]
\end{itemize}
% ---------------------------------------------------------
\item \textbf{[3 Puntos]} \\[1ex]
\emph{Restricci\'on:} Todos los locales de comida (C) deben estar concentrados en el mismo piso, el cual pasar\'a a contener la Plaza de Comidas del centro comercial. \\[1ex]
\emph{Implementaci\'on:}
\begin{itemize}
\item Introducimos una variable binaria indicadora para cada piso: 
\[
\forall \, j \in \upto{p} \; \colon \; a_j \in \{ 0, 1 \}
\]
\item Introducimos una restricci\'on o juego de restricciones para cada piso $j \in \upto{p}$ tal que si $x_{ij} = 1$ para al menos una empresa $i \in \text{C}$ entonces $a_j = 1$. Esto puede ser logrado de al menos dos maneras: 
\begin{itemize}
\item Introduciendo la siguiente restricci\'on: 
\[
\forall \, j \in \upto{p} \; \colon \;
\sum_{i \in \text{C}} x_{ij} \; \leq \; |C| \, a_j
\]
\item Introduciendo el siguiente juego de restricciones: 
\[
\forall \, j \in \upto{p}, \, \forall \, i \in \text{C} \; \colon \;
x_{ij} \; \leq \; a_j
\]
\end{itemize}
\item Introducimos una restricci\'on tal que $a_j = 1$ para exactamente un solo piso $j \in \upto{p}$. M\'as precisamente: 
\[
\sum_{j \in \upto{p}} a_j \; = \; 1
\]
\end{itemize}

% ---------------------------------------------------------
\item \textbf{[5 Puntos]} \\[1ex]
\emph{Restricci\'on:} En todo piso donde haya al menos un local de bienes inmuebles (BI) debe haber al menos \textit{(i)} tres locales de muebles o electrodom\'esticos (ME) y un banco (BK), o \textit{(ii)} dos locales de ropa (R) y dos locales de lectura y arte (LA). \\[1ex]
\emph{Implementaci\'on:}
\begin{itemize}
\item Introducimos un tr\'io de variables binarias indicadoras para cada piso: 
\[
\forall \, j \in \upto{p} \; \colon \; a_j, b_j, c_j \in \{ 0, 1 \}
\]
\item Introducimos una restricci\'on o juego de restricciones para cada piso $j \in \upto{p}$ tal que si $x_{ij} = 1$ para al menos una empresa $i \in \text{BI}$ entonces $a_j = 1$. Esto puede ser logrado de al menos dos maneras: 
\begin{itemize}
\item Introduciendo la siguiente restricci\'on: 
\[
\forall \, j \in \upto{p} \; \colon \;
\sum_{i \in \text{BI}} x_{ij} \; \leq \; |BI| \, a_j
\]
\item Introduciendo el siguiente juego de restricciones: 
\[
\forall \, j \in \upto{p}, \, \forall \, i \in \text{BI} \; \colon \;
x_{ij} \; \leq \; a_j
\]
\end{itemize}
\item Introducimos un par de restricciones para cada piso $j \in \upto{p}$ tal que si $b_j = 1$ entonces $x_{ij} = 1$ para al menos tres empresas $i \in \text{ME}$ y al menos una empresa $i \in \text{BK}$. \linebreak M\'as precisamente: 
\begin{align*}
& \forall \, j \in \upto{p} \; \colon \;
3 \, a_j \; \leq \; \sum_{i \in \text{ME}} x_{ij} \\
& \forall \, j \in \upto{p} \; \colon \;
1 \, a_j \; \leq \; \sum_{i \in \text{BK}} x_{ij}
\end{align*}
\item Introducimos un par de restricciones para cada piso $j \in \upto{p}$ tal que si $c_j = 1$ entonces $x_{ij} = 1$ para al menos dos empresas $i \in \text{R}$ y al menos dos empresas $i \in \text{LA}$. \linebreak M\'as precisamente: 
\begin{align*}
& \forall \, j \in \upto{p} \; \colon \;
2 \, a_j \; \leq \; \sum_{i \in \text{R}} x_{ij} \\
& \forall \, j \in \upto{p} \; \colon \;
2 \, a_j \; \leq \; \sum_{i \in \text{LA}} x_{ij}
\end{align*}
\item Introducimos una restricci\'on para cada piso $j \in \upto{p}$ tal que si $a_j = 1$ entonces $b_j = 1$ o $c_j = 1$. M\'as precisamente: 
\[
\forall \, i \in \upto{p} \; \colon \; a_j \; \leq \; b_j + c_j
\]
\end{itemize}

\end{enumerate}

\end{problem}
\vspace{\baselineskip}

% -----------------------------------------------------------------
\begin{problem}
Considere el problema de planificar la operaci\'on de una m\'aquina a lo largo de un horizonte de $T$ per\'iodos. En cada periodo la m\'aquina puede estar ocupada fabricando un lote de alguno de los $m$ productos diferentes que puede producir, puede estar recibiendo mantenimiento, o puede estar sin trabajar. Para representar estas actividades, introducimos tres series temporales de variables binarias: 
\begin{itemize}
\item Para cada periodo $t \in \upto{T}$ y cada producto $k \in \upto{m}$ la variable $x_{tk} = 1$ si y solo si en ese periodo la m\'aquina fabric\'o ese producto. 
\item Para cada periodo $t \in \upto{T}$ la variable $y_t = 1$ si y solo si la m\'aquina recibi\'o mantenimiento durante ese periodo. 
\item Para cada periodo $t \in \upto{T}$ la variable $z_t = 1$ si y solo si la m\'aquina no trabaj\'o durante ese periodo. 
\end{itemize}

Con todo esto en mente, escriba las siguientes restricciones temporales en el lenguage de la Progamaci\'on Lineal Entera (PLE). 
\begin{enumerate}[label=\textbf{\alph*)}]
\item No se permite fabricar el producto 1 por m\'as de dos per\'iodos consecutivos. 
\item Si se fabrica el producto 1 por dos periodos consecutivos entonces la m\'aquina debe recibir mantenimiento en el siguiente periodo. 
\item Si se fabrica el producto 2 por tres o m\'as per\'iodos consecutivos entonces la m\'aquina debe descansar (\ie no trabajar) en el siguiente periodo. 
\item Si se fabrica el producto 3 entonces eventualmente la m\'aquina debe descansar por un periodo y recibir mantenimiento en el posterior. 
\item Si se fabrica el producto 4 y se desea posteriormente fabricar el producto 5 entonces se debe dar mantenimiento a la m\'aquina antes de fabricar el producto 5. 
\end{enumerate}
\QED

% =========================================================
\emph{Soluci\'on:}
\begin{enumerate}[label=\textbf{\alph*)}]
% ---------------------------------------------------------
\item \textbf{[2 Puntos]} \\[1ex]
\emph{Restricci\'on:} No se permite fabricar el producto 1 por m\'as de dos per\'iodos consecutivos. \\[1ex]
\emph{Implementaci\'on:} Introducimos el siguiente juego de restricciones, donde $k = 1$ : 
\[
\forall \, t \in \upto{T-2} \; \colon \;
x_{t,k} + x_{t+1,k} + x_{t+2,k} \; \leq \; 2
\]
% ---------------------------------------------------------
\item \textbf{[2 Puntos]} \\[1ex]
\emph{Restricci\'on:} Si se fabrica el producto 1 por dos periodos consecutivos entonces la m\'aquina debe recibir mantenimiento en el siguiente periodo. \\[1ex]
\emph{Implementaci\'on:} Introducimos el siguiente juego de restricciones, donde $k = 1$ : 
\[
\forall \, t \in \upto{T-1} \; \colon \;
x_{t,k} + x_{t+1,k} \; \leq \; 1 + y_{t+2}
\]
% ---------------------------------------------------------
\item \textbf{[3 Puntos]} \\[1ex]
\emph{Restricci\'on:} Si se fabrica el producto 2 por tres o m\'as per\'iodos consecutivos entonces la m\'aquina debe descansar (\ie no trabajar) en el siguiente periodo. \\[1ex]
\emph{Implementaci\'on:} Introducimos el siguiente patr\'on de restricciones, donde $k = 2$ : 
\begin{align*}
& \forall \, t \in \upto{T-2} \; \colon \;
\sum_{i = t}^{t+2} x_{i,k} \; \leq \; 2 + x_{t+3,k} + z_{t+3} \\
& \forall \, t \in \upto{T-3} \; \colon \;
\sum_{i = t}^{t+3} x_{i,k} \; \leq \; 3 + x_{t+4,k} + z_{t+4} \\
& \forall \, t \in \upto{T-4} \; \colon \;
\sum_{i = t}^{t+4} x_{i,k} \; \leq \; 4 + x_{t+5,k} + z_{t+5} \\
& \dots
\end{align*}
% ---------------------------------------------------------
\item \textbf{[3 Puntos]} \\[1ex]
\emph{Restricci\'on:} Si se fabrica el producto 3 entonces eventualmente la m\'aquina debe descansar por un periodo y recibir mantenimiento en el posterior. \\[1ex]
\emph{Implementaci\'on:} Fijamos $k = 3$. Luego: 
\begin{itemize}
\item Introducimos la serie temporal de variables binarias $\{ w_t \}_{t=1}^{T-2}$. 
\item Introducimos una restricci\'on para cada periodo $t \in \upto{T-2}$ tal que si $x_{t,k} = 1$ entonces eventualmente $w_t = 1$. M\'as precisamente: 
\[
\forall \, t \in \upto{T-2} \; \colon \;
x_{t,k} \; \leq \; \sum_{i=t+1}^{T-2} w_t
\]
\item Introducimos un par de restricciones para cada periodo $t \in \upto{T-2}$ tal que si $w_t = 1$ entonces $z_{t+1} = 1$ y $y_{t+2} = 1$. M\'as precisamente: 
\begin{align*}
& \forall \, t \in \upto{T-2} \; \colon \;
w_t \; \leq \; z_{t+1} \\
& \forall \, t \in \upto{T-3} \; \colon \;
w_t \; \leq \; y_{t+2}
\end{align*}
\end{itemize}
% ---------------------------------------------------------
\item \textbf{[Literal anulado por exhibir muy alta complejidad]} \\[1ex]
\emph{Restricci\'on:} Si se fabrica el producto 4 y se desea posteriormente fabricar el producto 5 entonces se debe dar mantenimiento a la m\'aquina antes de fabricar el producto 5. \\[1ex]
\emph{Implementaci\'on:} Ser\'a mostrada en clase. 

\end{enumerate}

\end{problem}
\vspace{\baselineskip}

\end{document}
