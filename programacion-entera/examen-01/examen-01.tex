% -----------------------------------------------------------------
% Document class: Article
\documentclass[ a4paper, twoside, 11pt]{article}
\usepackage{../../macros-general}
\usepackage{../../macros-article}
% Number of the handout, quiz, exam, etc.
\newcommand{\numero}{01}
\setcounter{numero}{\numero}

% -----------------------------------------------------------------
\begin{document}
\allowdisplaybreaks

% Indices
\newcommand{\iava}{$i$\tsup{ava} }
\newcommand{\iavo}{$i$\tsup{avo} }
\newcommand{\java}{$j$\tsup{ava} }
\newcommand{\javo}{$j$\tsup{avo} }
\newcommand{\kava}{$k$\tsup{ava} }
\newcommand{\kavo}{$k$\tsup{avo} }
\newcommand{\tava}{$t$\tsup{ava} }
\newcommand{\tavo}{$t$\tsup{avo} }
\newcommand{\tmava}{$(t-1)$\tsup{ava} }
\newcommand{\tmavo}{$(t-1)$\tsup{avo} }
\newcommand{\tMava}{$(t+1)$\tsup{ava} }
\newcommand{\tMavo}{$(t+1)$\tsup{avo} }

\begin{center}
\Large Programaci\'on Entera (INDG-1019): Examen \numero \\[1ex]
\small \textbf{Semestre:} 2018-2019 T\'ermino I \qquad
\textbf{Instructor:} Luis I. Reyes Castro
\end{center}
\fullskip

% -----------------------------------------------------------------
\begin{problem}
Considere un programa entero que contiene $m \geq 10$ variables binarias
\[
x_1, \, x_2, \, x_3, \, \dots, \, x_m \in \{ 0, 1 \}
\]
junto con otras variables adicionales. En cada uno de los siguientes literales, se le presentar\'a un conjunto de suposiciones seguido de una regla o preferencia. Exprese cada una de las reglas o preferencias en el lenguaje de la programaci\'on lineal entera mediante la introducci\'on de variables enteras y/o de restricciones lineales. Por favor considere que cada literal es independiente de todos los otros. 
\begin{enumerate}[label=\alph*)]
\item \emph{Suposiciones:} $2 \leq n < m$ ; \, $y \in \{ 0, 1\}$. \\
\emph{Regla o preferencia:} La variable $y = 1$ si y solo si el n\'umero de \'indices $i \in \upto{m}$ tales que $x_i = 1$ es exactamente igual a $n$. 

\item \emph{Suposiciones:} $S \subset \upto{m}$ es un subconjunto de \'indices; \, $w \in \{ 0, 1\}$. \\
\emph{Regla o preferencia:} La variable $w = 1$ si y solo si para todo \'indice $i \in S$ tenemos $x_i = 1$. 

\item \emph{Suposiciones:} $S, T \subset \upto{m}$ son subconjuntos de \'indices. \\
\emph{Regla o preferencia:} Si $x_i = 1$ para al menos un \'indice $i \in S$ entonces $x_j = 0$ para todo \'indice $j \in T$, y vice-versa (\ie si $x_j = 1$ para al menos un \'indice $j \in T$ entonces $x_i = 0$ para todo \'indice $i \in S$). 

\item \emph{Suposiciones:} $S, T \subset \upto{m}$ son subconjuntos de \'indices; \, $w \in \{ 0, 1\}$. \\
\emph{Regla o preferencia:} Si el n\'umero de \'indices $i \in S$ para los cuales $x_i = 1$ es igual o mayor al n\'umero de \'indices $j \in S$ para los cuales $x_j = 1$ entonces $w = 1$. 

\item \emph{Suposiciones:} $S, T \subset \upto{m}$ son subconjuntos de \'indices. \\
\emph{Regla o preferencia:} Si $x_i = 1$ para al menos dos \'indices $i \in S$ entonces $x_j = 1$ para al menos cuatro \'indices $j \in T$. 

\end{enumerate}
\QED

\end{problem}
\fullskip

% -----------------------------------------------------------------
\begin{problem}
Considere un programa entero que contiene las series temporales de variables binarias $\{x_t\}_{t=1}^T \in \{ 0, 1 \}$, $\{y_t\}_{t=1}^T \in \{ 0, 1 \}$ y $\{z_t\}_{t=1}^T \in \{ 0, 1 \}$, donde $T \geq 10$, junto con otras variables adicionales. En este modelo, las series temporales $\{x_t\}$, $\{y_t\}$ y $\{z_t\}$ representan la ocurrencia o no-ocurrencia de tres tipos diferentes de eventos de inter\'es en un problema de planificaci\'on con un horizonte de $T$ per\'iodos. 

En cada uno de los siguientes literales, se le presentar\'a una regla o preferencia, posiblemente precedida por un conjunto de suposiciones. Exprese cada una de las reglas o preferencias en el lenguaje de la programaci\'on lineal entera mediante la introducci\'on de variables enteras y/o de restricciones lineales. Recuerde que cada literal es independiente de los otros. 
\begin{enumerate}[label=\alph*)]
\item \emph{Regla o preferencia:} Si para alg\'un periodo $k \in \upto{T-1}$ tenemos $x_k = 1$ entonces $y_{k+1} = 1$. 

\item \emph{Regla o preferencia:} Existe al menos un periodo $k \in \upto{T-1}$ tal que $x_k = 1$ y $x_{k+1} = 1$. 

\item \emph{Regla o preferencia:} Si para alg\'un periodo $k \in \upto{T-1}$ tenemos $x_k = 1$ entonces $y_{k+\ell} = 1$ para al menos un $\ell \geq 1$. 

\item \emph{Regla o preferencia:} Si para alg\'un periodo $k \in \upto{T-2}$ tenemos $x_k = 1$ y $y_{k+1} = 1$ entonces $z_{k+2} = 1$. 

\item \emph{Regla o preferencia:} Si para alg\'un periodo $k \in \upto{T-1}$ tenemos $x_k = 1$ y $y_{k} = 1$ entonces $z_{k+ \ell} = 1$ para al menos un $\ell \geq 1$. 

\end{enumerate}
\QED

\end{problem}
\fullskip

% -----------------------------------------------------------------
\begin{problem}
Considere un problema de producci\'on y almacenamiento con m\'ultiples per\'iodos y con una estructura de costos decrementales. En particular: 
\begin{itemize}
\item El horizonte del problema es de $T$ per\'iodos. 
\item Los costos de producci\'on por periodo obecen el patr\'on mostrado en la siguiente tabla, donde $c_1 > c_2 > c_3 > \cdots > c_M > 0$. 
\begin{table}[htb]
\centering
\begin{tabular}{|c|c|c|}
\hline
\textbf{Cantidad M\'inima} & \textbf{Cantidad M\'axima} & \textbf{Costo por Unidad} \\ \hline
$q_0 = 0$ & $q_1$ & $c_1$ \\ \hline
$q_1+1$ & $q_2$ & $c_2$ \\ \hline
$q_2+1$ & $q_3$ & $c_3$ \\ \hline
$\vdots$ & $\vdots$ & $\vdots$ \\ \hline
$q_{M-1}+1$ & $q_M$ & $c_M$ \\ \hline
\end{tabular}
\end{table}

En general, para cualquier $k \in \upto{m}$, si se producen entre $q_{k-1} + 1$ y $q_k$ unidades se incurre un costo de producci\'on de $c_k$ d\'olares por unidad. 

\item Las demanadas del producto est\'an representadas por la series de tiempo $\{ d_t \}_{t=1}^T$. 
\item Se tiene una bodega con capacidad para hasta $B$ unidades, y se incurre un costo de almacenamiento por unidad por periodo de $c_{inv}$ d\'olares. 

\end{itemize}

Con esto en mente, modelaremos este problema de producci\'on y almacenamiento como un Programa Lineal Entero. Para esto introducimos las siguientes variables: 
\begin{itemize}
\item Para todo $t \in \upto{T}$ y todo $k \in \upto{M}$ la variable binaria $x_{tk}$ indica si en el periodo $t$ se produjeron entre $q_{k-1} + 1$ unidades y $q_k$ unidades. 
\item Para todo $t \in \upto{T}$ y todo $k \in \upto{M}$ la variable entera $z_{tk}$ indica el n\'umero de unidades producidas en los casos cuando se producen entre $q_{k-1}+1$ y $q_k$ unidades. 
\item Para todo $t \in \upto{T-1}$ la variable entera $w_t$ indica el n\'umero de unidades puestas en almacenamiento entre el periodo $t-1$ y el periodo $t$. 
\end{itemize}

Utilizando las variables anteriores, complete las siguientes actividades: 
\begin{enumerate}[label=\alph*)]
\item Escriba las restricciones que dictan que en cada periodo $t$ exactamente una de las variables $x_{tk}$ debe tomar el valor uno. 
\item Escriba las restricciones que dictan que en cada periodo $t$ es el caso que si $x_{tk} = 0$ entonces $z_{tk} = 0$. 
\item Escriba las restricciones que dictan que en cada periodo $t$ es el caso que si $x_{tk} = 1$ entonces $z_{tk} \geq q_{k-1} + 1$. 
\item Escriba las restricciones que dictan que en cada periodo $t$ es el caso que si $x_{tk} = 1$ entonces $z_{tk} \leq q_{k}$. 
\item Escriba una expresi\'on, v\'alida para cada periodo $t$, para el n\'umero de unidades producidas el periodo $t$. 
\item Escriba las restricciones asociadas con la producci\'on, el almacenamiento y la demanda a lo largo del horizonte de planificaci\'on. 
\item Escriba una expresi\'on, v\'alida para cada periodo $t$, para el costo de producci\'on incurrido en el periodo $t$. 
\item Escriba la funci\'on de costo que buscamos minimizar. 
\end{enumerate}
\QED

\end{problem}
\fullskip

\end{document}
