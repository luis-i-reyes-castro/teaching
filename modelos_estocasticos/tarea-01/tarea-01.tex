% -----------------------------------------------------------------
% Document class: Article
\documentclass[ a4paper, twoside, 11pt]{article}
\usepackage{../../macros-general}
\usepackage{../../macros-article}
% Number of the handout, quiz, exam, etc.
\newcommand{\numero}{01}
\setcounter{numero}{\numero}

% -----------------------------------------------------------------
\begin{document}
\allowdisplaybreaks

% Indices
\newcommand{\iava}{$i$\tsup{ava} }
\newcommand{\iavo}{$i$\tsup{avo} }
\newcommand{\java}{$j$\tsup{ava} }
\newcommand{\javo}{$j$\tsup{avo} }
\newcommand{\kava}{$k$\tsup{ava} }
\newcommand{\kavo}{$k$\tsup{avo} }
\newcommand{\tava}{$t$\tsup{ava} }
\newcommand{\tavo}{$t$\tsup{avo} }
\newcommand{\tmava}{$(t-1)$\tsup{ava} }
\newcommand{\tmavo}{$(t-1)$\tsup{avo} }
\newcommand{\tMava}{$(t+1)$\tsup{ava} }
\newcommand{\tMavo}{$(t+1)$\tsup{avo} }

\begin{center}
\Large Modelos Estoc\'asticos (INDG-1008): Tarea \numero \\[1ex]
\small \textbf{Semestre:} 2017-2018 T\'ermino I \qquad
\textbf{Instructor:} Luis I. Reyes Castro
\end{center}
\fullskip

%\fbox{

\begin{minipage}[b][\height][t]{\textwidth}
\vspace{0.2 cm}

\begin{center}
\textbf{COMPROMISO DE HONOR}
\end{center}
\vspace{0.4 cm}

\scriptsize
{
Yo, \rule{60mm}{.1pt} al firmar este compromiso, reconozco que la presente lecci\'on est\'a dise\~nada para ser resuelta de manera individual, que puedo usar un l\'apiz o pluma y una calculadora cient\'ifica, \linebreak que solo puedo comunicarme con la persona responsable de la recepci\'on de la lecci\'on, y que cualquier instrumento de comunicaci\'on que hubiere tra\'ido debo apagarlo. Tambi\'en estoy conciente que no debo consultar libros, notas, \linebreak ni materiales did\'acticos adicionales a los que el instructor entregue durante la lecci\'on o autorice a utilizar. Finalmente, me comprometo a desarrollar y presentar mis respuestas de manera clara y ordenada. \\

Firmo al pie del presente compromiso como constancia de haberlo le\'ido y aceptado. 
\vspace{0.4 cm}

Firma: \rule{60mm}{.1pt} \qquad N\'umero de matr\'icula: \rule{40mm}{.1pt} \hspace{0.5cm} \\[-0.8ex]
}

\end{minipage}

}

\vspace{\baselineskip}



% =============================================
\begin{problem}
Suponga que usted ha sido encargado con del manejo del inventario de algunos productos no-perecederos en un supermercado que abre todos los d\'ias a sus clientes por la misma cantidad de tiempo. Para mantener una consistencia en el manejo de inventario a trav\'es de los varios productos que se venden en el supermercado, el gerente ha dispuesto que todos los inventarios se controlen mediante pol\'iticas de punto de reposici\'on. 

M\'as precisamente, al final de cada d\'ia se cuenta el inventario del producto y si este es menor o igual a $I_{rep}$ unidades se hace un pedido de reposici\'on por $Q_{rep}$ unidades al proveedor, el cual es entregado por el mismo al comienzo del siguiente d\'ia antes de la hora de apertura de la tienda; caso contrario, no se hace un pedido de reposici\'on. F\'ijese que bajo estas suposiciones el m\'aximo inventario posible es:
\[
I_{max} \, = \, I_{rep} + Q_{rep}
\]
Adicionalmente,como es de costumbre en este campo de estudio, usted hace la suposici\'on simplificatoria que las demandas del producto $D_1, \, D_2, \, \dots$ constituyen una secuencia de variables aleatorias i.i.d. con distribuci\'on Poisson con par\'ametro $\lambda$. M\'as a\'un, si para cada d\'ia $t$ denotamos a $I_t$ como el inventario a la hora de apertura de la tienda del \tavo d\'ia, a $D_t$ como la demanda del d\'ia, \ie la cantidad que se vender\'ia ese d\'ia si el inventario fuere inagotable, a $R_t$ como la cantidad repuesta entre la noche del \tavo d\'ia y la hora de apertura del \tMavo d\'ia, tenemos que:
\[
I_{t+1} \, = \, \max \{ \, 0, \, I_t - D_t \, \} + R_t \, ; \qquad
R_t \, = \, 
\begin{cases}
Q_{rep}, & \text{si } 0 \leq I_t \leq I_{rep} \, ; \\
0, & \text{caso contrario;}
\end{cases}
\]

Con todo esto en mente, para cada problema de manejo de inventario con demanda $\lambda$ y pol\'itica de punto de reposici\'on $(I_{rep},Q_{rep})$ podemos construir una Cadena de Markov en $n = I_{max} + 1$ estados. M\'as precisamente, los estados ser\'ian: Cero unidades en inventario, una unidad en inventario, y asi sucesivamente, hasta $I_{max}$ unidades en inventario. 

\textbf{Primera Actividad: (X Puntos)}

Escriba una funci\'on en Python que tome como entrada (1) el par\'ametro $\lambda$ de la distribuci\'on Poisson de la demanda, (2) el inventario de reposici\'on $I_{rep}$ y (3) la cantidad por reposici\'on $Q_{rep}$, y que produzca como \'unica salida un arreglo {\tt numpy} de taman\~no $(n,n)$. Por ejemplo, yo usar\'ia su funci\'on de la siguiente manera:
\halfskip

\begin{center}
\begin{minipage}{0.8\textwidth}
\begin{minted}[mathescape,
               linenos,
               numbersep=5pt,
               gobble=0,
               frame=lines,
               framesep=2mm]{Python}
# Entradas:
lambda_demanda = 6.9 # Parametro de la demanda
I_reposicion = 5 # Inventario de reposicion
Q_reposicion = 5 # Cantidad por reposicion
# Salida: Matriz de probabilidad de la Cadena de Markov
matriz_P = su_funcion( lambda_demanda, I_reposicion, Q_reposicion)
\end{minted}
\end{minipage}
\end{center}
\halfskip

\emph{Sugerencia:} Para hacerse la vida m\'as f\'acil a la hora de calcular probabilidades le recomiendo que utilice la funcion {\tt poisson.pmf()} de la librer\'ia {\tt scipy.stats}. \href{https://docs.scipy.org/doc/scipy/reference/generated/scipy.stats.poisson.html}{Click aqu\'i} para la documentaci\'on. Por ejemplo, para calcular l

\end{problem}
\vspace{\baselineskip}

\end{document}
