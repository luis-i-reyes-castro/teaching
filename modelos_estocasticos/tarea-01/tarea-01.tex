% -----------------------------------------------------------------
% Document class: Article
\documentclass[ a4paper, twoside, 11pt]{article}
\usepackage{../../macros-general}
\usepackage{../../macros-article}
% Number of the handout, quiz, exam, etc.
\newcommand{\numero}{01}
\setcounter{numero}{\numero}

% -----------------------------------------------------------------
\begin{document}
\allowdisplaybreaks

\begin{center}
\Large Modelos Estoc\'asticos (INDG-1008): Tarea \numero \\[1ex]
\small \textbf{Semestre:} 2017-2018 T\'ermino I \qquad
\textbf{Instructor:} Luis I. Reyes Castro
\end{center}
\fullskip

%\fbox{

\begin{minipage}[b][\height][t]{\textwidth}
\vspace{0.2 cm}

\begin{center}
\textbf{COMPROMISO DE HONOR}
\end{center}
\vspace{0.4 cm}

\scriptsize
{
Yo, \rule{60mm}{.1pt} al firmar este compromiso, reconozco que la presente lecci\'on est\'a dise\~nada para ser resuelta de manera individual, que puedo usar un l\'apiz o pluma y una calculadora cient\'ifica, \linebreak que solo puedo comunicarme con la persona responsable de la recepci\'on de la lecci\'on, y que cualquier instrumento de comunicaci\'on que hubiere tra\'ido debo apagarlo. Tambi\'en estoy conciente que no debo consultar libros, notas, \linebreak ni materiales did\'acticos adicionales a los que el instructor entregue durante la lecci\'on o autorice a utilizar. Finalmente, me comprometo a desarrollar y presentar mis respuestas de manera clara y ordenada. \\

Firmo al pie del presente compromiso como constancia de haberlo le\'ido y aceptado. 
\vspace{0.4 cm}

Firma: \rule{60mm}{.1pt} \qquad N\'umero de matr\'icula: \rule{40mm}{.1pt} \hspace{0.5cm} \\[-0.8ex]
}

\end{minipage}

}

\vspace{\baselineskip}



% =============================================
\begin{problem}
Suponga que usted ha sido encargado con del manejo del inventario de algunos productos no-perecederos en un supermercado que abre todos los d\'ias a sus clientes por la misma cantidad de tiempo. Para mantener una consistencia en el manejo de inventario a trav\'es de los varios productos que se venden en el supermercado, el gerente ha dispuesto que todos los inventarios se controlen mediante pol\'iticas de punto de reposici\'on. 

M\'as precisamente, al final de cada d\'ia se cuenta el inventario del producto y se hace un pedido por $Q_{rep}$ unidades al proveedor si el inventario es menor o igual a $I_{rep}$ unidades, \linebreak el cual es entregado por el proveedor al comienzo del siguiente d\'ia antes de la hora de apertura de la tienda; caso contrario, no se hace un pedido. F\'ijese que bajo estas suposiciones el m\'aximo inventario posible es $I_{rep} + Q_{rep}$. 

Adicionalmente,como es de costumbre en este campo de estudio, usted hace la suposici\'on simplificatoria que las demandas del producto $D_1, \, D_2, \, \dots$ constituyen una secuencia de variables aleatorias i.i.d. con distribuci\'on Poisson con par\'ametro $\lambda$. 

Ahora, asuma que usted desea modelar las pol\'iticas de manejo inventario 


 usando Cadenas de Markov. En particular, usted quiere ser capaz de construir la matriz de transici\'on para cualquier producto con demanda $\lambda$, n\'umero unidades por reposici\'on $Q_{rep}$ y punto de reposici\'on $I_{rep}$. 


\fullskip
\begin{center}
\begin{minipage}{0.8\textwidth}
\begin{minted}[mathescape,
               linenos,
               numbersep=5pt,
               gobble=0,
               frame=lines,
               framesep=2mm]{Python}
# Entradas:
lambda_demanda = 6.9 # Parametro de la demanda
I_reposicion = 5 # Inventario de reposicion
Q_reposicion = 5 # Cantidad por reposicion

# Salida: Matriz de probabilidad de la Cadena de Markov
# como un arreglo numpy de tamano (n,n),
# donde n = I_reposicion + Q_reposicion + 1
P = su_funcion( lambda_demanda, I_reposicion, Q_reposicion)
\end{minted}
\end{minipage}
\end{center}




\end{problem}
\vspace{\baselineskip}

\end{document}
