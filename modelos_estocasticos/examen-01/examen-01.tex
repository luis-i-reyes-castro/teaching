% -----------------------------------------------------------------
% Document class: Article
\documentclass[ a4paper, twoside, 11pt]{article}
\usepackage{../../macros-general}
\usepackage{../../macros-article}
% Number of the handout, quiz, exam, etc.
\newcommand{\numero}{01}
\setcounter{numero}{\numero}

% -----------------------------------------------------------------
\begin{document}
\allowdisplaybreaks

\begin{center}
\Large Modelos Estoc\'asticos (INDG-1008): Examen \numero \\[1ex]
\small \textbf{Semestre:} 2017-2018 T\'ermino I \qquad
\textbf{Instructor:} Luis I. Reyes Castro
\end{center}
\fullskip

% Indices
\newcommand{\iava}{$i$\tsup{ava} }
\newcommand{\iavo}{$i$\tsup{avo} }
\newcommand{\java}{$j$\tsup{ava} }
\newcommand{\javo}{$j$\tsup{avo} }
\newcommand{\kava}{$k$\tsup{ava} }
\newcommand{\kavo}{$k$\tsup{avo} }
\newcommand{\tava}{$t$\tsup{ava} }
\newcommand{\tavo}{$t$\tsup{avo} }
\newcommand{\tmava}{$(t-1)$\tsup{ava} }
\newcommand{\tmavo}{$(t-1)$\tsup{avo} }
\newcommand{\tMava}{$(t+1)$\tsup{ava} }
\newcommand{\tMavo}{$(t+1)$\tsup{avo} }
\fbox{

\begin{minipage}[b][\height][t]{\textwidth}
\vspace{0.2 cm}

\begin{center}
\textbf{COMPROMISO DE HONOR}
\end{center}
\vspace{0.4 cm}

\scriptsize
{
Yo, \rule{60mm}{.1pt} al firmar este compromiso, reconozco que la presente evaluaci\'on est\'a dise\~nada para ser resuelta de manera individual, que puedo usar un l\'apiz o pluma y una calculadora cient\'ifica, \linebreak que solo puedo comunicarme con la persona responsable de la recepci\'on de la evaluaci\'on, y que cualquier instrumento de comunicaci\'on que hubiere tra\'ido debo apagarlo. Tambi\'en estoy conciente que no debo consultar libros, notas, \linebreak ni materiales did\'acticos adicionales a los que el instructor entregue durante la evaluaci\'on o autorice a utilizar. Finalmente, me comprometo a desarrollar y presentar mis respuestas de manera clara y ordenada. \\

Firmo al pie del presente compromiso como constancia de haberlo le\'ido y aceptado. 
\vspace{0.4 cm}

Firma: \rule{60mm}{.1pt} \qquad N\'umero de matr\'icula: \rule{40mm}{.1pt} \hspace{0.5cm} \\[-0.8ex]
}

\end{minipage}

}

\vspace{\baselineskip}



% -----------------------------------------------------------------
\begin{problem}
\textbf{[7 Puntos]} Un operador de servicios de telefon\'ia celular est\'a en proceso de instalar una antena en un barrio promedio, donde se puede esperar que la antena reciba en promedio $\lambda$ paquetes de datos por ciclo (\eg por milisegundo). Para poder satisfacer esta demanda se instal\'o un buffer, \ie una cola, con capacidad de $M$ paquetes. Adem\'as se instalaron $n$ transmisores que operan en canales independientes pero ruidosos. En particular, cada transmisor que es encargado con el env\'io de un paquete de datos logra transmitirlo exitosamente con probabilidad $\gamma$. Cuando una transmisi\'on fracasa se mantiene al paquete en el buffer y se reintenta la transmisi\'on en el siguiente per\'iodo. 

Cada ciclo de operaci\'on, digamos el \tavo ciclo, avanza de la siguiente manera: 
\begin{enumerate}
\item Se empieza el ciclo con $X_{t-1}$ paquetes en el b\'uffer. 
\item Se reciben nuevos paquetes si el buffer lo permite, de tal manera que el nuevo n\'umero de paquetes en el buffer es: 
\[
\min \; \{ \; X_{t-1} + D_t \, , \; M \, \} \, ,
\qquad \text{donde } D_t \sim \Poisson(\lambda)
\]
\item Los transmisores intentan enviar cuantos paquetes puedan. Si hay $n$ paquetes o m\'as en el buffer, entonces cada uno de los transmisores es asignado aleatoriamente a un \'unico paquete, y cada transmisor logra enviar su paquete con \'exito con probabilidad $\gamma$, independiente de los otros. Si hay menos de $n$ paquetes en el buffer se opera de la misma manera, pero en este caso habr\'a uno o m\'as transmisores a los que no se les asignen paquetes durante este ciclo. 
\end{enumerate}

Con todo esto en mente, construya un modelo de Cadena de Markov de este proceso para el caso particular cuando $M = 5$ y $n = 3$. En particular, explique cuales son los estados y liste todas las probabilidades de transici\'on positivas. Al desarrollar su respuesta por favor haga uso de la siguiente notaci\'on: 
\begin{align*}
p(\lambda;k) \, 
& \define \, \Pr \, ( \, \Poisson(\lambda) = k \, ) \\
\hat{p}(\lambda;k) \, 
& \define \, \Pr \, ( \, \Poisson(\lambda) \geq k \, ) \\
b(n,\gamma;k) \, 
& \define \, \Pr \, ( \, \Binomial(n,\gamma) = k \, )
\end{align*}

\emph{Nota:} En vez de listar las probabilidades usted puede construir la matriz de transici\'on siempre y cuando la desarrolle con suficiente espacio, \eg en una hoja vac\'ia en formato retrato. 

\end{problem}
\vspace{\baselineskip}

% -----------------------------------------------------------------
\begin{problem}




\end{problem}
\vspace{\baselineskip}

\end{document}
