% -----------------------------------------------------------------
% Document class: Article
\documentclass[ a4paper, twoside, 11pt]{article}
\usepackage{../../macros-general}
\usepackage{../../macros-article}
%\graphicspath{{./figures/}}
% Uncomment the following line if you need to write Python code
% \usepackage{minted}

% Number of the handout, quiz, exam, etc.
\newcommand{\numero}{04}
\setcounter{numero}{\numero}

% -----------------------------------------------------------------
\begin{document}
\allowdisplaybreaks

% Indices
\newcommand{\iava}{$i$\tsup{ava} }
\newcommand{\iavo}{$i$\tsup{avo} }
\newcommand{\java}{$j$\tsup{ava} }
\newcommand{\javo}{$j$\tsup{avo} }
\newcommand{\kava}{$k$\tsup{ava} }
\newcommand{\kavo}{$k$\tsup{avo} }
\newcommand{\tava}{$t$\tsup{ava} }
\newcommand{\tavo}{$t$\tsup{avo} }
\newcommand{\tmava}{$(t-1)$\tsup{ava} }
\newcommand{\tmavo}{$(t-1)$\tsup{avo} }
\newcommand{\tMava}{$(t+1)$\tsup{ava} }
\newcommand{\tMavo}{$(t+1)$\tsup{avo} }

\begin{center}
\Large Modelos Estoc\'asticos (INDG-1008): Trabajo Aut\'onomo \numero \\[1ex]
\small \textbf{Semestre:} 2017-2018 T\'ermino II \qquad
\textbf{Instructor:} Luis I. Reyes Castro
\end{center}
\fullskip

%\fbox{

\begin{minipage}[b][\height][t]{\textwidth}
\vspace{0.2 cm}

\begin{center}
\textbf{COMPROMISO DE HONOR}
\end{center}
\vspace{0.4 cm}

\scriptsize
{
Yo, \rule{60mm}{.1pt} al firmar este compromiso, reconozco que la presente evaluaci\'on est\'a dise\~nada para ser resuelta de manera individual, que puedo usar un l\'apiz o pluma y una calculadora cient\'ifica, \linebreak que solo puedo comunicarme con la persona responsable de la recepci\'on de la evaluaci\'on, y que cualquier instrumento de comunicaci\'on que hubiere tra\'ido debo apagarlo. Tambi\'en estoy conciente que no debo consultar libros, notas, \linebreak ni materiales did\'acticos adicionales a los que el instructor entregue durante la evaluaci\'on o autorice a utilizar. Finalmente, me comprometo a desarrollar y presentar mis respuestas de manera clara y ordenada. \\

Firmo al pie del presente compromiso como constancia de haberlo le\'ido y aceptado. 
\vspace{0.4 cm}

Firma: \rule{60mm}{.1pt} \qquad N\'umero de matr\'icula: \rule{40mm}{.1pt} \hspace{0.5cm} \\[-0.8ex]
}

\end{minipage}

}

\vspace{\baselineskip}


\halfskip

% -----------------------------------------------------------------
\begin{problem}
\label{prob:H&L_P15_3_3} Lea el enunciado del Problema 15.3-3 del texto de Hillier \& Lieberman y complete las siguientes actividades: 
\begin{enumerate}[label=\textbf{\alph*)}]
\item \textbf{1 Punto:} Determine la alternativa \'optima de acuerdo a la distribuci\'on \emph{a priori}. 
\item \textbf{2 Puntos:} Cu\'al es el m\'aximo precio que usted deber\'ia pagar por informaci\'on exacta sobre el estado futuro que ocurrir\'a?
\end{enumerate}

\end{problem}
\fullskip

% -----------------------------------------------------------------
\begin{problem}
\label{prob:H&L_P15_3_5} Lea el enunciado del Problema 15.3-5 del texto de Hillier \& Lieberman y complete las siguientes actividades: 
\begin{enumerate}[label=\textbf{\alph*)}]
\item \textbf{2 Puntos:} Se tiene la oportunidad de obtener informaci\'on que dir\'a con certidumbre si ocurrir\'a el estado de la naturaleza $S_2$. Cu\'al es el m\'aximo precio que usted deber\'ia pagar por esa informaci\'on? Y c\'omo deber\'ia usarla? 
\item \textbf{2 Puntos:} Se tiene la oportunidad de obtener informaci\'on que dir\'a con certidumbre si ocurrir\'a el estado de la naturaleza $S_3$. Cu\'al es el m\'aximo precio que usted deber\'ia pagar por esa informaci\'on? Y c\'omo deber\'ia usarla? 
\end{enumerate}


\end{problem}
\fullskip

% -----------------------------------------------------------------
\begin{problem}
\label{prob:H&L_P15_3_6}
\textbf{[3 Puntos]} Lea el enunciado del Problema 15.3-6 del texto de Hillier \& Lieberman y determine la pol\'itica \'optima resultante. 

\end{problem}
\fullskip

% -----------------------------------------------------------------
\begin{problem}
\label{prob:H&L_P15_3_7} Lea el enunciado del Problema 15.3-7 del texto de Hillier \& Lieberman y complete las siguientes actividades: 
\begin{enumerate}[label=\textbf{\alph*)}]
\item \textbf{1 Punto:} Determine la alternativa \'optima de acuerdo a la distribuci\'on \emph{a priori}. 
\item \textbf{2 Puntos:} Calcule la distribuci\'on posterior y la distribuci\'on posterior predictiva. 
\item \textbf{2 Puntos:} Cu\'al es el m\'aximo precio que deber\'ia pagar por la investigaci\'on? Adem\'as, asumiendo que se decide pagar por la investigaci\'on, cu\'al es la pol\'itica \'optima?  
\end{enumerate}

\end{problem}
\fullskip

% -----------------------------------------------------------------
\begin{problem}
\label{prob:H&L_P15_3_10}
\textbf{[2 Puntos]} Lea el enunciado del Problema 15.3-10 del texto de Hillier \& Lieberman y calcule la distribuci\'on posterior y la distribuci\'on posterior predictiva. 

\end{problem}
\fullskip

% -----------------------------------------------------------------
\begin{problem}
\label{prob:H&L_P15_3_11} Lea el enunciado del Problema 15.3-11 del texto de Hillier \& Lieberman y complete las siguientes actividades: 
\begin{enumerate}[label=\textbf{\alph*)}]
\item \textbf{1 Punto:} Determine la alternativa \'optima de acuerdo a la distribuci\'on \emph{a priori}. 
\item \textbf{2 Puntos:} Calcule la distribuci\'on posterior y la distribuci\'on posterior predictiva. 
\item \textbf{2 Puntos:} Cu\'al es el m\'aximo precio que deber\'ia pagar por la investigaci\'on? Adem\'as, asumiendo que se decide pagar por la investigaci\'on, cu\'al es la pol\'itica \'optima?  
\end{enumerate}

\end{problem}
\fullskip

% -----------------------------------------------------------------
\begin{problem}
\label{prob:H&L_P15_3_12}
\textbf{[6 Puntos]} Lea el enunciado del Problema 15.3-12 del texto de Hillier \& Lieberman. En caso de que lo encuentre dif\'icil de entender, lo parafraseo a continuaci\'on: 

`` La compa\~n\'ia Hit-and-Miss produce art\'iculos en lotes de 150 unidades. Cada art\'iculo producido tiene una probabilidad $p$ de salir defectuoso, independiente de todos los otros. La experiencia indica que el 80\% de los lotes se producen con $p = 0.05$, mientras que el 20\% restante con $p = 0.25$. La f\'abrica incurre un costo de \$100 por cada art\'iculo defectuoso que produce y es eventualmente devuelto por alg\'un distribuidor. Para lidiar con la variable calidad de estos art\'iculos, la f\'abrica est\'a considerando los dos siguientes tipos de inspecci\'on: 
\begin{itemize}
\item Un inspecci\'on inicial r\'apida y barata, a un costo fijo de $Q$ d\'olares por lote. Esta inspecci\'on consiste en elegir uno de los art\'iculos del lote producido, al azar, y determinar si es defectuoso o no. Si el art\'iculo es defectuoso, se lo reemplaza ah\'i mismo. 
\item Una inspecci\'on exhaustiva pero costosa, que puede ser llevada a cabo despu\'es de la inspecci\'on inicial (si se decide hacer la misma). Esta inspecci\'on consiste en literalmente analizar cada art\'iculo del lote y reemplazar los defectuosos. Esta inspecci\'on tiene un costo de \$10 por unidad. '' 
\end{itemize}

Con esto en mente, encuentre el m\'aximo costo fijo $Q$ para el cual es justificable hacer la inspecci\'on inicial r\'apida y barata. Adem\'as, conteste las siguiente preguntas, suponiendo que se decide hacer la inspecci\'on inicial: 
\begin{enumerate}[label=\textbf{\alph*)}]
\item Si en la inspecci\'on inicial se encontr\'o que el art\'iculo es defectuoso, deber\'ia realizarse la inspecci\'on exhaustiva? 
\item Si en la inspecci\'on inicial se encontr\'o que el art\'iculo no es defectuoso, deber\'ia no realizarse la inspecci\'on exhaustiva? 
\end{enumerate}

\end{problem}
\fullskip

% -----------------------------------------------------------------
\begin{problem}
\label{prob:H&L_P15_4_3}
\textbf{[2 Puntos]} Resuelva el Problema 15.4-3 del texto de Hillier \& Lieberman. 

\end{problem}
\fullskip

% -----------------------------------------------------------------
\begin{problem}
\label{prob:H&L_P15_4_4} Lea el enunciado del Problema 15.4-4 del texto de Hillier \& Lieberman y complete las siguientes actividades: 
\begin{enumerate}[label=\textbf{\alph*)}]
\item \textbf{1 Punto:} Determine la alternativa \'optima de acuerdo a la distribuci\'on \emph{a priori}. 
\item \textbf{2 Puntos:} Lea el literal (d) y calcule la distribuci\'on posterior y la distribuci\'on posterior predictiva. 
\item \textbf{2 Puntos:} Cu\'al es el m\'aximo precio que deber\'ia pagar por el pron\'ostico del gur\'u? Adem\'as, asumiendo que se paga por el pron\'ostico, cu\'al es la pol\'itica \'optima?
\end{enumerate}

\end{problem}
\fullskip

% -----------------------------------------------------------------
\begin{problem}
\label{prob:H&L_P15_4_5}
\textbf{[6 Puntos]} Resuelva el Problema 15.4-5 del texto de Hillier \& Lieberman. 

\end{problem}
\fullskip

% -----------------------------------------------------------------
\begin{problem}
\label{prob:H&L_P15_4_6}
\textbf{[6 Puntos]} Resuelva el Problema 15.4-6 del texto de Hillier \& Lieberman. 

\end{problem}
\fullskip

% -----------------------------------------------------------------
\begin{problem}
\label{prob:H&L_P15_4_11} Lea el enunciado del Problema 15.4-11 del texto de Hillier \& Lieberman y complete las siguientes actividades: 
\begin{enumerate}[label=\textbf{\alph*)}]
\item \textbf{1 Punto:} Determine la alternativa \'optima de acuerdo a la distribuci\'on \emph{a priori}. 
\item \textbf{2 Puntos:} Calcule la distribuci\'on posterior y la distribuci\'on posterior predictiva. 
\item \textbf{2 Puntos:} Cu\'al es el m\'aximo precio que deber\'ia pagar por la investigaci\'on detallada? Adem\'as, asumiendo que se paga por la investigaci\'on, cu\'al es la pol\'itica \'optima?
\end{enumerate}

\end{problem}
\fullskip

% -----------------------------------------------------------------
\begin{problem}
\label{prob:H&L_P15_5_7} Lea el enunciado del Problema 15.5-7 del texto de Hillier \& Lieberman y complete las siguientes actividades: 
\begin{enumerate}[label=\textbf{\alph*)}]
\item \textbf{1 Punto:} Determine la alternativa \'optima de acuerdo a la distribuci\'on \emph{a priori}. 
\item \textbf{2 Puntos:} Calcule la distribuci\'on posterior y la distribuci\'on posterior predictiva. 
\item \textbf{2 Puntos:} Cu\'al es el m\'aximo costo que deber\'ia incurrir para que sea justificable lanzar el producto en el mercado de prueba? Adem\'as, asumiendo que se decide lanzar el producto en el mercado de prueba, cu\'al es la pol\'itica \'optima?
\end{enumerate}

\end{problem}
\fullskip

% -----------------------------------------------------------------
\begin{problem}
\label{prob:H&L_P15_5_8}
\textbf{[6 Puntos]} Resuelva el primer literal del Problema 15.5-8 del texto de Hillier \& Lieberman. 

\end{problem}
\fullskip

\end{document}
