% -----------------------------------------------------------------
% Document class: Article
\documentclass[ a4paper, twoside, 11pt]{article}
\usepackage{../../macros-general}
\usepackage{../../macros-article}
%\graphicspath{{./figures/}}
% Number of the handout, quiz, exam, etc.
\newcommand{\numero}{04}
\setcounter{numero}{\numero}

% -----------------------------------------------------------------
\begin{document}
\allowdisplaybreaks

% Indices
\newcommand{\iava}{$i$\tsup{ava} }
\newcommand{\iavo}{$i$\tsup{avo} }
\newcommand{\java}{$j$\tsup{ava} }
\newcommand{\javo}{$j$\tsup{avo} }
\newcommand{\kava}{$k$\tsup{ava} }
\newcommand{\kavo}{$k$\tsup{avo} }
\newcommand{\tava}{$t$\tsup{ava} }
\newcommand{\tavo}{$t$\tsup{avo} }
\newcommand{\tmava}{$(t-1)$\tsup{ava} }
\newcommand{\tmavo}{$(t-1)$\tsup{avo} }
\newcommand{\tMava}{$(t+1)$\tsup{ava} }
\newcommand{\tMavo}{$(t+1)$\tsup{avo} }

\begin{center}
\Large Modelos Estoc\'asticos (INDG-1008): Lecci\'on \numero \\[2ex]
\small \textbf{Semestre:} 2017-2018 T\'ermino II \qquad
\textbf{Instructor:} Luis I. Reyes Castro
\end{center}
\fullskip

% -----------------------------------------------------------------
\begin{problem}
\textbf{[12 Puntos]} Suponga que en un futuro cercano una empresa se dedica a transportar pasageros entre sus dos estaciones en Guayaquil y Salinas utilizando carros aut\'onomos. \linebreak La empresa tiene una flota de $N = 5$ carros que oscila entre sus dos estaciones, y cada carro puede ser utilizado no m\'as de una vez al d\'ia. Todas las ma\~nanas, cada cada carro que amanece en la estaci\'on de Guayaquil es utilizado por alg\'un cliente para ir a Salinas con probabilidad igual a $p$. Similarmente, cada cada carro que amanece en la estaci\'on de Salinas es utilizado por alg\'un cliente para ir a Guayaquil con probabilidad $q$. La empresa percibe ingresos de $u$ d\'olares por viaje, sin importar su direcci\'on. 

Puesto que usualmente $p > q$, \ie en promedio m\'as clientes quieren viajar de Guayaquil a Salinas que de Salindas a Guayaquil, la empresa usualmente debe re-balancear su flota, lo cual siempre hace de noche despu\'es de cerrar sus operaciones por el d\'ia. Para esto se ordena a los carros manejarse vac\'ios de una estaci\'on a otra. El costo de cada viaje vac\'io de re-balanceo es de $c$ d\'olares. Por seguridad, un ser humano debe monitorear el viaje de los veh\'iculos en una computadora, por lo que el n\'umero de carros que se re-balancea por noche no puede exceder de $M = 2$. Por esta misma raz\'on, se incurre un costo fijo de $4c$ d\'olares por re-balanceo, sin importar el n\'umero de carros movidos. 

Con todo esto en mente, modele el problema de encontrar una pol\'itica \'optima de re-balanceo como un Proceso de Decisi\'on Markoviano (PDM). Al construir su modelo, por favor defina sus estados de tal manera que en cada per\'iodo primero se toma la decisi\'on de re-balanceo, luego se ordena a los carros viajar vac\'ios, de ser necesario, y finalmente los clientes utilizan los carros para movilizarse entre las estaciones. 

\emph{Nota:} Suponga que todos los carros aut\'onomos son id\'enticos. De esta manera, usted puede darse cuenta f\'acilmente que los viajes de re-balanceo siempre ser\'an en una sola direcci\'on. 

\emph{Sugerencia:} Para ahorrarse tiempo al describir las probabilidades de transici\'on para cada par estado-acci\'on, por favor utilice la siguiente notaci\'on: 
\begin{align*}
\alpha(k,m) \; & \define \; \Pr( \, \Binomial(p,m) = k \, ) \\
\beta(\ell,n) \; & \define \; \Pr( \, \Binomial(q,n) = \ell \, )
\end{align*}

\emph{Soluci\'on:} Primero reconocemos que el problema tiene seis estados, puesto que la flota es de cinco carros id\'enticos. Estos estados se caracterizan por el n\'umero de carros que terminan el d\'ia de trabajo en cada estaci\'on, de manera que si utilizamos la notaci\'on $(G,S)$ para representar el n\'umero de carros  en las estaciones de Guayaquil y Salinas, respectivamente, vemos que el conjunto de estados es: 
\[
\{ \,  (5,0), \, (4,1), \, (3,2), \, (2,3), \, (1,4), \, (0,5) \, \}
\]
Las acciones son re-balanceos, y pueden ser caracterizadas por el n\'umero de carros enviados en la direcci\'on Guayaquil-Salinas, de manera que un cero indica que no se hace re-balanceo, \linebreak un n\'umero positivo indica que se env\'ian carros de Guayaquil a Salinas, y un n\'umero negativo indica que se env\'ia carros de Salinas a Guayaquil. Las acciones disponibles en cada estado se muestran en la siguiente tabla. 

\begin{table}[H]
\centering
\begin{tabular}{c|c|c|c|c|c|}
\cline{2-6}
 & \multicolumn{5}{c|}{\textbf{Acci\'on}} \\ \hline
\multicolumn{1}{|c|}{\textbf{Estado} $\boldsymbol{(G,S)}$} & $+2 $& $+1$ & $\pm 0$ & $-1$ & $-2$ \\ \hline
\multicolumn{1}{|c|}{$(5,0)$}
& \checkmark  & \checkmark  & \checkmark  &  &  \\ \hline
\multicolumn{1}{|c|}{$(4,1)$}
& \checkmark  & \checkmark  & \checkmark  & \checkmark  &  \\ \hline
\multicolumn{1}{|c|}{$(3,2)$}
& \checkmark  & \checkmark  & \checkmark  & \checkmark  & \checkmark \\ \hline
\multicolumn{1}{|c|}{$(2,3)$}
& \checkmark  & \checkmark  & \checkmark  & \checkmark  & \checkmark \\ \hline
\multicolumn{1}{|c|}{$(1,4)$}
&  & \checkmark  & \checkmark  & \checkmark  & \checkmark \\ \hline
\multicolumn{1}{|c|}{$(0,5)$}
&  &  & \checkmark  & \checkmark  & \checkmark \\ \hline
\end{tabular}
\end{table}

Ahora consideramos cada estado y cada acci\'on disponible en ese estado, y listamos las probabilidades de transici\'on a todos los posibles estados sucesores: 
\begin{itemize}
% -------------------------------------------------------------------
\item Para el estado $X_t = (5,0)$ :
\begin{itemize}
\item Si tomamos la acci\'on $A_t = \pm 0$ el sistema amanece en el estado $(5,0)$. Entonces: 
\begin{table}[H]
\centering
\begin{tabular}{|c|c|}
\hline
\textbf{Estado Sucesor} $\boldsymbol{X_{t+1}}$
& $\boldsymbol{\Pr( \, X_{t+1} \mid X_t, \, A_t \, )}$ \\ \hline
$(5,0)$ & $\alpha(0,5)$ \\ \hline
$(4,1)$ & $\alpha(1,5)$ \\ \hline
$(3,2)$ & $\alpha(2,5)$ \\ \hline
$(2,3)$ & $\alpha(3,5)$ \\ \hline
$(1,4)$ & $\alpha(4,5)$ \\ \hline
$(0,5)$ & $\alpha(5,5)$ \\ \hline
\end{tabular}
\end{table}
\item Si tomamos la acci\'on $A_t = +1$ el sistema amanece en el estado $(4,1)$. Entonces: 
\begin{table}[H]
\centering
\begin{tabular}{|c|c|}
\hline
\textbf{Estado Sucesor} $\boldsymbol{X_{t+1}}$
& $\boldsymbol{\Pr( \, X_{t+1} \mid X_t, \, A_t \, )}$ \\ \hline
$(5,0)$ & $\alpha(0,4) \, \beta(1,1)$ \\ \hline
$(4,1)$ & $\alpha(0,4) \, \beta(0,1) + \alpha(1,4) \, \beta(1,1)$ \\ \hline
$(3,2)$ & $\alpha(1,4) \, \beta(0,1) + \alpha(2,4) \, \beta(1,1)$ \\ \hline
$(2,3)$ & $\alpha(2,4) \, \beta(0,1) + \alpha(3,4) \, \beta(1,1)$ \\ \hline
$(1,4)$ & $\alpha(3,4) \, \beta(0,1) + \alpha(4,4) \, \beta(1,1)$ \\ \hline
$(0,5)$ & $\alpha(4,4) \, \beta(0,1)$ \\ \hline
\end{tabular}
\end{table}
\item Si tomamos la acci\'on $A_t = +2$ el sistema amanece en el estado $(3,2)$. Entonces: 
\begin{table}[H]
\centering
\begin{tabular}{|c|c|}
\hline
\textbf{Estado Sucesor} $\boldsymbol{X_{t+1}}$
& $\boldsymbol{\Pr( \, X_{t+1} \mid X_t, \, A_t \, )}$ \\ \hline
$(5,0)$ & $\alpha(0,3) \, \beta(2,2)$ \\ \hline
$(4,1)$ & $\alpha(0,3) \, \beta(1,2) + \alpha(1,3) \, \beta(2,2)$ \\ \hline
$(3,2)$ & $\alpha(0,3) \, \beta(0,2) + \alpha(1,3) \, \beta(1,2) + \alpha(2,3) \, \beta(2,2)$ \\ \hline
$(2,3)$ & $\alpha(1,3) \, \beta(0,2) + \alpha(2,3) \, \beta(1,2) + \alpha(3,3) \, \beta(2,2)$ \\ \hline
$(1,4)$ & $\alpha(2,3) \, \beta(0,2) + \alpha(3,3) \, \beta(1,2)$ \\ \hline
$(0,5)$ & $\alpha(3,3) \, \beta(0,2)$ \\ \hline
\end{tabular}
\end{table}
\end{itemize}
% -------------------------------------------------------------------
\item Para el estado $X_t = (4,1)$ :
\begin{itemize}
\item Si tomamos la acci\'on $A_t = \pm 0$ el sistema amanece en el estado $(4,1)$, por lo que la distribuci\'on sobre los estados sucesores es id\'entica a la del caso cuando el estado es $X_t = (5,0)$ y la acci\'on es $A_t = +1$. 
\item Si tomamos la acci\'on $A_t = +1$ el sistema amanece en el estado $(3,2)$, por lo que la distribuci\'on sobre los estados sucesores es id\'entica a la del caso cuando el estado es $X_t = (5,0)$ y la acci\'on es $A_t = +2$. 
\item Si tomamos la acci\'on $A_t = +2$ el sistema amanece en el estado $(2,3)$. Entonces: 
\begin{table}[H]
\centering
\begin{tabular}{|c|c|}
\hline
\textbf{Estado Sucesor} $\boldsymbol{X_{t+1}}$
& $\boldsymbol{\Pr( \, X_{t+1} \mid X_t, \, A_t \, )}$ \\ \hline
$(5,0)$ & $\alpha(0,2) \, \beta(3,3)$ \\ \hline
$(4,1)$ & $\alpha(0,2) \, \beta(2,3) + \alpha(1,2) \, \beta(3,3)$ \\ \hline
$(3,2)$ & $\alpha(0,2) \, \beta(1,3) + \alpha(1,2) \, \beta(2,3) + \alpha(2,2) \, \beta(3,3)$ \\ \hline
$(2,3)$ & $\alpha(0,2) \, \beta(0,3) + \alpha(1,2) \, \beta(1,3) + \alpha(2,2) \, \beta(2,3)$ \\ \hline
$(1,4)$ & $\alpha(1,2) \, \beta(0,3) + \alpha(2,2) \, \beta(1,3)$ \\ \hline
$(0,5)$ & $\alpha(2,2) \, \beta(0,3)$ \\ \hline
\end{tabular}
\end{table}
\item Si tomamos la acci\'on $A_t = -1$ el sistema amanece en el estado $(5,0)$, por lo que la distribuci\'on sobre los estados sucesores es id\'entica a la del caso cuando el estado es $X_t = (5,0)$ y la acci\'on es $A_t = \pm 0$. 
\end{itemize}
% -------------------------------------------------------------------
\item Para el estado $X_t = (3,2)$ :
\begin{itemize}
\item Si tomamos la acci\'on $A_t = \pm 0$ el sistema amanece en el estado $(3,2)$, por lo que la distribuci\'on sobre los estados sucesores es id\'entica a la del caso cuando el estado es $X_t = (5,0)$ y la acci\'on es $A_t = +2$. 
\item Si tomamos la acci\'on $A_t = +1$ el sistema amanece en el estado $(2,3)$, por lo que la distribuci\'on sobre los estados sucesores es id\'entica a la del caso cuando el estado es $X_t = (4,1)$ y la acci\'on es $A_t = +2$. 
\item Si tomamos la acci\'on $A_t = +2$ el sistema amanece en el estado $(1,4)$. Entonces: 
\begin{table}[H]
\centering
\begin{tabular}{|c|c|}
\hline
\textbf{Estado Sucesor} $\boldsymbol{X_{t+1}}$
& $\boldsymbol{\Pr( \, X_{t+1} \mid X_t, \, A_t \, )}$ \\ \hline
$(5,0)$ & $\alpha(0,1) \, \beta(4,4)$ \\ \hline
$(4,1)$ & $\alpha(0,1) \, \beta(3,4) + \alpha(1,1) \, \beta(4,4)$ \\ \hline
$(3,2)$ & $\alpha(0,1) \, \beta(2,4) + \alpha(1,1) \, \beta(3,4)$ \\ \hline
$(2,3)$ & $\alpha(0,1) \, \beta(1,4) + \alpha(1,1) \, \beta(2,4)$ \\ \hline
$(1,4)$ & $\alpha(0,1) \, \beta(0,4) + \alpha(1,1) \, \beta(1,4)$ \\ \hline
$(0,5)$ & $\alpha(1,1) \, \beta(0,4)$ \\ \hline
\end{tabular}
\end{table}
\item Si tomamos la acci\'on $A_t = -1$ el sistema amanece en el estado $(4,1)$, por lo que la distribuci\'on sobre los estados sucesores es id\'entica a la del caso cuando el estado es $X_t = (5,0)$ y la acci\'on es $A_t = +1$. 
\item Si tomamos la acci\'on $A_t = -2$ el sistema amanece en el estado $(5,0)$, por lo que la distribuci\'on sobre los estados sucesores es id\'entica a la del caso cuando el estado es $X_t = (5,0)$ y la acci\'on es $A_t = \pm 0$. 
\end{itemize}
% -------------------------------------------------------------------
\item Para el estado $X_t = (2,3)$ :
\begin{itemize}
\item Si tomamos la acci\'on $A_t = \pm 0$ el sistema amanece en el estado $(2,3)$, por lo que la distribuci\'on sobre los estados sucesores es id\'entica a la del caso cuando el estado es $X_t = (4,1)$ y la acci\'on es $A_t = +2$. 
\item Si tomamos la acci\'on $A_t = +1$ el sistema amanece en el estado $(1,4)$, por lo que la distribuci\'on sobre los estados sucesores es id\'entica a la del caso cuando el estado es $X_t = (3,2)$ y la acci\'on es $A_t = +2$. 
\item Si tomamos la acci\'on $A_t = +2$ el sistema amanece en el estado $(0,5)$. Entonces: 
\begin{table}[H]
\centering
\begin{tabular}{|c|c|}
\hline
\textbf{Estado Sucesor} $\boldsymbol{X_{t+1}}$
& $\boldsymbol{\Pr( \, X_{t+1} \mid X_t, \, A_t \, )}$ \\ \hline
$(5,0)$ & $\beta(5,5)$ \\ \hline
$(4,1)$ & $\beta(4,5)$ \\ \hline
$(3,2)$ & $\beta(3,5)$ \\ \hline
$(2,3)$ & $\beta(2,5)$ \\ \hline
$(1,4)$ & $\beta(1,5)$ \\ \hline
$(0,5)$ & $\beta(0,5)$ \\ \hline
\end{tabular}
\end{table}
\item Si tomamos la acci\'on $A_t = -1$ el sistema amanece en el estado $(3,2)$, por lo que la distribuci\'on sobre los estados sucesores es id\'entica a la del caso cuando el estado es $X_t = (5,0)$ y la acci\'on es $A_t = +2$. 
\item Si tomamos la acci\'on $A_t = -2$ el sistema amanece en el estado $(4,1)$, por lo que la distribuci\'on sobre los estados sucesores es id\'entica a la del caso cuando el estado es $X_t = (5,0)$ y la acci\'on es $A_t = +1$. 
\end{itemize}
% -------------------------------------------------------------------
\item Para el estado $X_t = (1,4)$ :
\begin{itemize}
\item Si tomamos la acci\'on $A_t = \pm 0$ el sistema amanece en el estado $(1,4)$, por lo que la distribuci\'on sobre los estados sucesores es id\'entica a la del caso cuando el estado es $X_t = (3,2)$ y la acci\'on es $A_t = +2$. 
\item Si tomamos la acci\'on $A_t = +1$ el sistema amanece en el estado $(0,5)$, por lo que la distribuci\'on sobre los estados sucesores es id\'entica a la del caso cuando el estado es $X_t = (2,3)$ y la acci\'on es $A_t = +2$. 
\item Si tomamos la acci\'on $A_t = -1$ el sistema amanece en el estado $(2,3)$, por lo que la distribuci\'on sobre los estados sucesores es id\'entica a la del caso cuando el estado es $X_t = (4,1)$ y la acci\'on es $A_t = +2$. 
\item Si tomamos la acci\'on $A_t = -2$ el sistema amanece en el estado $(3,2)$, por lo que la distribuci\'on sobre los estados sucesores es id\'entica a la del caso cuando el estado es $X_t = (5,0)$ y la acci\'on es $A_t = +2$. 
\end{itemize}
% -------------------------------------------------------------------
\item Para el estado $X_t = (0,5)$ :
\begin{itemize}
\item Si tomamos la acci\'on $A_t = \pm 0$ el sistema amanece en el estado $(0,5)$, por lo que la distribuci\'on sobre los estados sucesores es id\'entica a la del caso cuando el estado es $X_t = (2,3)$ y la acci\'on es $A_t = +2$. 
\item Si tomamos la acci\'on $A_t = -1$ el sistema amanece en el estado $(1,4)$, por lo que la distribuci\'on sobre los estados sucesores es id\'entica a la del caso cuando el estado es $X_t = (3,2)$ y la acci\'on es $A_t = +2$. 
\item Si tomamos la acci\'on $A_t = -2$ el sistema amanece en el estado $(2,3)$, por lo que la distribuci\'on sobre los estados sucesores es id\'entica a la del caso cuando el estado es $X_t = (4,1)$ y la acci\'on es $A_t = +2$. 
\end{itemize}
\end{itemize}

Finalmente, calculamos las utilidades asociadas con cada para estado-acci\'on. Para esto primero reconocemos que si el sistema termina el d\'ia de trabajo en el estado $X_t = (G,S)$ y se decide no haber re-balanceo, \ie la acci\'on es $A_t = \pm 0$, entonces el n\'umero esperado de carros que ser\'an utilizados por los clientes al d\'ia siguiente es $G \, p + S \, q$. Por lo tanto, en este caso tenemos: 
\[
\Exp[ \, \text{Utilidad}( \, X_t = (G,S) , \, A_t = \pm 0 ) \, ]
\; = \; ( \, G \, p + S \, q \, ) \, u
\]
En cambio, si el sistema termina el d\'ia en el estado $X_t = (G,S)$ y se decide hacer un re-balanceo, \ie la acci\'on es $A_t = k \neq 0$, el n\'umero esperado de carros que ser\'an utilizados por los clientes al d\'ia siguiente es $(G-k) \, p + (S+k) \, q$. As\'i, para este caso tenemos: 
\[
\Exp[ \, \text{Utilidad}( \, X_t = (G,S) , \, A_t = \pm 0 ) \, ]
\; = \; ( \, (G-k) \, p + (S+k) \, q \, ) \, u - 4c - |k|c
\]
Consecuentemente: 
\begin{itemize}
% -------------------------------------------------------------------
\item Para el estado $X_t = (5,0)$ :
\begin{table}[H]
\centering
\begin{tabular}{|c|c|}
\hline
\textbf{Acci\'on} $\boldsymbol{A_t}$
& $\boldsymbol{ \Exp[ \, \textbf{Utilidad}( \, X_t, \, A_t ) \, ]}$ \\ \hline
$\pm 0$ & $5 \, p \, u$ \\ \hline
$+1$ & $( 4p + q ) \, u - 5c$ \\ \hline
$+2$ & $( 3p + 2q ) \, u - 6c$ \\ \hline
\end{tabular}
\end{table}
% -------------------------------------------------------------------
\item Para el estado $X_t = (4,1)$ :
\begin{table}[H]
\centering
\begin{tabular}{|c|c|}
\hline
\textbf{Acci\'on} $\boldsymbol{A_t}$
& $\boldsymbol{ \Exp[ \, \textbf{Utilidad}( \, X_t, \, A_t ) \, ]}$ \\ \hline
$\pm 0$ & $( 4p + q ) \, u$ \\ \hline
$+1$ & $( 3p + 2q ) \, u - 5c$ \\ \hline
$+2$ & $( 2p + 3q ) \, u - 6c$ \\ \hline
$-1$ & $5 \, p \, u - 5c$ \\ \hline
\end{tabular}
\end{table}
% -------------------------------------------------------------------
\item Para el estado $X_t = (3,2)$ :
\begin{table}[H]
\centering
\begin{tabular}{|c|c|}
\hline
\textbf{Acci\'on} $\boldsymbol{A_t}$
& $\boldsymbol{ \Exp[ \, \textbf{Utilidad}( \, X_t, \, A_t ) \, ]}$ \\ \hline
$\pm 0$ & $( 3p + 2q ) \, u$ \\ \hline
$+1$ & $( 2p + 3q ) \, u - 5c$ \\ \hline
$+2$ & $( p + 4q ) \, u - 6c$ \\ \hline
$-1$ & $( 4p + q ) \, u - 5c$ \\ \hline
$-2$ & $5 \, p \, u - 6c$ \\ \hline
\end{tabular}
\end{table}
% -------------------------------------------------------------------
\item Para el estado $X_t = (2,3)$ :
\begin{table}[H]
\centering
\begin{tabular}{|c|c|}
\hline
\textbf{Acci\'on} $\boldsymbol{A_t}$
& $\boldsymbol{ \Exp[ \, \textbf{Utilidad}( \, X_t, \, A_t ) \, ]}$ \\ \hline
$\pm 0$ & $( 2p + 3q ) \, u$ \\ \hline
$+1$ & $( p + 4q ) \, u - 5c$ \\ \hline
$+2$ & $5 \, q \, u - 6c$ \\ \hline
$-1$ & $( 3p + 2q ) \, u - 5c$ \\ \hline
$-2$ & $( 4p + q ) \, u - 6c$ \\ \hline
\end{tabular}
\end{table}
% -------------------------------------------------------------------
\item Para el estado $X_t = (1,4)$ :
\begin{table}[H]
\centering
\begin{tabular}{|c|c|}
\hline
\textbf{Acci\'on} $\boldsymbol{A_t}$
& $\boldsymbol{ \Exp[ \, \textbf{Utilidad}( \, X_t, \, A_t ) \, ]}$ \\ \hline
$\pm 0$ & $( p + 4q ) \, u$ \\ \hline
$+1$ & $5 \, q \, u - 5c$ \\ \hline
$-1$ & $( 2p + 3q ) \, u - 5c$ \\ \hline
$-2$ & $( 3p + 2q ) \, u - 6c$ \\ \hline
\end{tabular}
\end{table}
% -------------------------------------------------------------------
\item Para el estado $X_t = (0,5)$ :
\begin{table}[H]
\centering
\begin{tabular}{|c|c|}
\hline
\textbf{Acci\'on} $\boldsymbol{A_t}$
& $\boldsymbol{ \Exp[ \, \textbf{Utilidad}( \, X_t, \, A_t ) \, ]}$ \\ \hline
$\pm 0$ & $5 \, q \, u$ \\ \hline
$-1$ & $( p + 4q ) \, u - 5c$ \\ \hline
$-2$ & $( 2p + 3q ) \, u - 6c$ \\ \hline
\end{tabular}
\end{table}

\end{itemize}

\end{problem}
\fullskip

\end{document}
