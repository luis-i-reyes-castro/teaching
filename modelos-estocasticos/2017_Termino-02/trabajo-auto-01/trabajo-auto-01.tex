% -----------------------------------------------------------------
% Document class: Article
\documentclass[ a4paper, twoside, 11pt]{article}
\usepackage{../../macros-general}
\usepackage{../../macros-article}
\graphicspath{{./figures/}}
% Uncomment the following line if you need to write Python code
% \usepackage{minted}

% Number of the handout, quiz, exam, etc.
\newcommand{\numero}{01}
\setcounter{numero}{\numero}

% -----------------------------------------------------------------
\begin{document}
\allowdisplaybreaks

% Indices
\newcommand{\iava}{$i$\tsup{ava} }
\newcommand{\iavo}{$i$\tsup{avo} }
\newcommand{\java}{$j$\tsup{ava} }
\newcommand{\javo}{$j$\tsup{avo} }
\newcommand{\kava}{$k$\tsup{ava} }
\newcommand{\kavo}{$k$\tsup{avo} }
\newcommand{\tava}{$t$\tsup{ava} }
\newcommand{\tavo}{$t$\tsup{avo} }
\newcommand{\tmava}{$(t-1)$\tsup{ava} }
\newcommand{\tmavo}{$(t-1)$\tsup{avo} }
\newcommand{\tMava}{$(t+1)$\tsup{ava} }
\newcommand{\tMavo}{$(t+1)$\tsup{avo} }

\begin{center}
\Large Modelos Estoc\'asticos (INDG-1008): Trabajo Aut\'onomo \numero \\[1ex]
\small \textbf{Semestre:} 2017-2018 T\'ermino II \qquad
\textbf{Instructor:} Luis I. Reyes Castro
\end{center}
\fullskip

%\fbox{

\begin{minipage}[b][\height][t]{\textwidth}
\vspace{0.2 cm}

\begin{center}
\textbf{COMPROMISO DE HONOR}
\end{center}
\vspace{0.4 cm}

\scriptsize
{
Yo, \rule{60mm}{.1pt} al firmar este compromiso, reconozco que la presente evaluaci\'on est\'a dise\~nada para ser resuelta de manera individual, que puedo usar un l\'apiz o pluma y una calculadora cient\'ifica, \linebreak que solo puedo comunicarme con la persona responsable de la recepci\'on de la evaluaci\'on, y que cualquier instrumento de comunicaci\'on que hubiere tra\'ido debo apagarlo. Tambi\'en estoy conciente que no debo consultar libros, notas, \linebreak ni materiales did\'acticos adicionales a los que el instructor entregue durante la evaluaci\'on o autorice a utilizar. Finalmente, me comprometo a desarrollar y presentar mis respuestas de manera clara y ordenada. \\

Firmo al pie del presente compromiso como constancia de haberlo le\'ido y aceptado. 
\vspace{0.4 cm}

Firma: \rule{60mm}{.1pt} \qquad N\'umero de matr\'icula: \rule{40mm}{.1pt} \hspace{0.5cm} \\[-0.8ex]
}

\end{minipage}

}

\vspace{\baselineskip}


\halfskip

% -----------------------------------------------------------------
\begin{problem}
\textbf{(2 Puntos)} La biblioteca p\'ublica de Springdale recibe nuevos libros de acuerdo a una distribuci\'on Poisson con media de 25 libros por d\'ia y los exhibe en analaqueles con capacidad para 100 libros. Determine lo siguiente: 
\begin{enumerate}[label=\alph*)]
\item El promedio de anaqueles que se llenar\'an de nuevos libros cada mes (30 d\'ias). 
\item La probabilidad de que se requieran m\'as de 10 libreros cada mes, si un librero se compone de 5 anaqueles.
\end{enumerate}

\end{problem}
\vspace{\baselineskip}

% -----------------------------------------------------------------
\begin{problem}
\textbf{(3 Puntos)} Un coleccionista de arte viaja a subastas de arte una vez al mes en promedio. Cada viaje es seguro que produzca una compra. El tiempo entre viajes est\'a exponencialmente distribuido. Determine lo siguiente: 
\begin{enumerate}[label=\alph*)]
\item La probabilidad de que se realice exactamente una compra en un periodo de 3 meses. 
\item La probabilidad de que se realicen no m\'as de 8 compras por a\~no. 
\item La probabilidad de que el tiempo entre viajes sucesivos exceda de 1 mes. 
\end{enumerate}

\end{problem}
\vspace{\baselineskip}

% -----------------------------------------------------------------
\begin{problem}
\textbf{(2 Puntos)} El tiempo entre llegadas en el restaurante L\&J es exponencial con media de 5 minutos. El restaurante abre a las 11:00 A.M. Determine lo siguiente: 
\begin{enumerate}[label=\alph*)]
\item La probabilidad de que para las 11:12 hallan arribado un total de 10 clientes, dado que para las 11:05 hab\'ian arribado 8 clientes. 
\item Las probabilidad de que al menos un nuevo cliente llegue entre las 11:28 y las 11:33, si el \'ultimo cliente lleg\'o a las 11:25. 
\end{enumerate}

\end{problem}
\vspace{\baselineskip}

% -----------------------------------------------------------------
\begin{problem}
\textbf{(2 Puntos)} La U de A opera dos l\'ineas de autobuses en el campus. \linebreak La l\'inea roja presta servicio al norte del campus, mientras que la l\'inea verde presta servicio \linebreak al sur del campus. Una estaci\'on de transferencia conecta las dos l\'ineas. Los autobuses verdes llegan a la estaci\'on de transferencia de acuerdo a un proceso Poisson con tiempo medio entre arribos de 10 minutos. Los autobuses rojos tienen medio entre arribos de 7 minutos. 
\begin{enumerate}[label=\alph*)]
\item Cu\'al es la probabilidad de que al menos un autobus de cada una de las dos l\'ineas se detengan en la estaci\'on durante un intervalo de 5 minutos?
\item Un estudiante cuyo dormitorio est\'a cerca de la estaci\'on tiene clase en 10 minutos. Cualquiera de los autobuses lo lleva al edificio del sal\'on de clases. El viaje requiere 5 minutos, despu\'es de lo cual el estudiante camina durante aproximadamente 3 minutos para llegar al sal\'on. Cu\'al es la probabilidad de que el estudiante llegue a tiempo a clase?
\end{enumerate}

\end{problem}
\vspace{\baselineskip}

% -----------------------------------------------------------------
\begin{problem}
\textbf{(2 Puntos)} En una tienda de Guayaquil solo se venden dos tipos de cerveza: Pilsener y Club Verde. El n\'umero de botellas de Pilsener que se vende diariamente obedece una distribuci\'on Poisson con par\'ametro $\mu_P$, mientras que las ventas de Club Verde tienen distribuci\'on Poisson con par\'ametro $\mu_{CV}$. Suponiendo que las ventas de estos dos tipos de cervezas son independientes e id\'enticamente distribu\'idas: 
\begin{enumerate}[label=\alph*)]
\item Encuentre la distribuci\'on del n\'umero de cervezas que se venden diariamente como funci\'on de $\mu_{P}$ y $\mu_{CV}$, \ie su soporte y funci\'on de masa probabil\'istica. 
\item En promedio, qu\'e porcentajes del total de cervezas que se venden diariamente corresponden a Pilsener y a Club Verde? 
\end{enumerate}

\end{problem}
\vspace{\baselineskip}

% -----------------------------------------------------------------
\begin{problem}
\textbf{(3 Puntos)} En una oficina p\'ublica solo se ofrece un tr\'amite y solo atiende un empleado. Los ciudadanos arriban de acuerdo a un proceso Poisson con media de tres por hora. \linebreak Los tiempos que le toman al empleado atender a los ciudadanos son independientes y est\'an uniformemente distribu\'idos entre diez y veinte minutos. Si arriba un ciudadano mientras el empleado est\'a ocupado atendiendo a uno anterior, el reci\'en arribado deber\'a sentarse a esperar.  Suponiendo que la oficina abre a las 8:00 A.M., encuentre la distribuci\'on del \'indice del primer ciudadano que tendr\'a que sentarse a esperar, \ie la probabilidad, para cada entero $k \geq 2$, \linebreak de que el \kavo ciudadano sea el primero en tener que sentarse a esperar. 

\end{problem}
\vspace{\baselineskip}

% -----------------------------------------------------------------
\begin{problem}
\textbf{(2 Puntos)} En una transitada esquina de la ciudad un empleado de una agencia publicitaria pasa volantes a los peatones. El jefe del empleado sabe por experiencia que en esa esquina se pueden repartir volantes a una tasa de dos y medio por minuto. Suponiendo que los tiempos entre repartos est\'an exponencialmente distribu\'idos, y que el jefe le entreg\'o al empleado 100 volantes para repartir, conteste las siguientes preguntas: 
\begin{enumerate}[label=\alph*)]
\item Cu\'al es la distribuci\'on del tiempo que le tomar\'a al empleado repartir las volantes? 
\item El empleado sale a repartir las volantes y vuelve 20 minutos despu\'es prometiendo haber hecho su trabajo, pero su jefe no le cree. Asumiendo que el empleado dice la verdad, \linebreak cu\'al es la probabilidad del evento en cuesti\'on? 
\end{enumerate}

\end{problem}
\vspace{\baselineskip}

% -----------------------------------------------------------------
\begin{problem}
En la Rep\'ublica del Banano los servidores p\'ublicos son conocidos por ser bastante corruptos. De hecho, es tan rampante la corrupci\'on en ese pa\'is que la publicaci\'on de esc\'andalos puede ser muy bien modelada como un proceso Poisson con media de seis al a\~no. M\'as a\'un, est\'a tan bien documentada y estudiada la historia de la corrupci\'on en ese pa\'is que se sabe lo siguiente: 
\begin{itemize}
\item Cuatro de cada diez veces un mismo partido alcanza la presidencia y la mayor\'ia legislativa al mismo tiempo. En estos casos, los gobernates siempre eligen crear leyes para amordazar a la ``prensa corrupta que no sirve ni para madurar aguacates''. Como consecuencia de este tipo de leyes los gobernantes logran prevenir la publicaci\'on de siete de cada diez esc\'andalos de corrupci\'on. 
\item La Fiscal\'ia de cada nuevo gobierno no hace el menor esfuerzo por cumplir sus funciones hasta que el tercer o cuarto esc\'andalo de corrupci\'on es expuesto por la prensa. La mitad de las veces la Fiscal\'ia espera hasta el tercer esc\'andalo y la otra hasta el cuarto. 
\item Como regla general, los pobladores del pa\'is destituyen a su gobierno si durante cualquier par de a\~nos consecutivos se publican m\'as de diez esc\'andalos por a\~no.
\end{itemize}

Asumiendo que un nuevo gobierno acaba de ser elegido y de tomar el poder: 
\begin{enumerate}[label=\alph*)]
\item \textbf{(2 Puntos)} Cu\'al es la probabilidad de que para el final del primer a\~no la nueva Fiscal\'ia se haya visto obligada a actuar ante el n\'umero de esc\'andalos publicados?
\item \textbf{(2 Puntos)} Suponiendo que durante los primeros 18 meses no se publicaron esc\'andalos de corrupci\'on, cu\'al es la probabilidad de que el nuevo gobierno haya amordazado a la ``prensa corrupta''?
\item \textbf{(2 Puntos)} Con qu\'e probabilidad completar\'a el gobierno su mandato de 4 a\~nos?
\end{enumerate}

\end{problem}
\vspace{\baselineskip}

\end{document}
