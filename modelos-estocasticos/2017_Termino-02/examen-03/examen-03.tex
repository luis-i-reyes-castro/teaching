% -----------------------------------------------------------------
% Document class: Article
\documentclass[ a4paper, twoside, 11pt]{article}
\usepackage{../../macros-general}
\usepackage{../../macros-article}
% Number of the handout, quiz, exam, etc.
\newcommand{\numero}{03}
\setcounter{numero}{\numero}
\graphicspath{{./figures/}}

% -----------------------------------------------------------------
\begin{document}
\allowdisplaybreaks

% Indices
\newcommand{\iava}{$i$\tsup{ava} }
\newcommand{\iavo}{$i$\tsup{avo} }
\newcommand{\java}{$j$\tsup{ava} }
\newcommand{\javo}{$j$\tsup{avo} }
\newcommand{\kava}{$k$\tsup{ava} }
\newcommand{\kavo}{$k$\tsup{avo} }
\newcommand{\tava}{$t$\tsup{ava} }
\newcommand{\tavo}{$t$\tsup{avo} }
\newcommand{\tmava}{$(t-1)$\tsup{ava} }
\newcommand{\tmavo}{$(t-1)$\tsup{avo} }
\newcommand{\tMava}{$(t+1)$\tsup{ava} }
\newcommand{\tMavo}{$(t+1)$\tsup{avo} }

\begin{center}
\Large Modelos Estoc\'asticos (INDG-1008): Examen \numero \\[2ex]
\small \textbf{Semestre:} 2017-2018 T\'ermino II \qquad
\textbf{Instructor:} Luis I. Reyes Castro
\end{center}
\fullskip

% -----------------------------------------------------------------
\begin{problem}
\label{prob:maquina-caprichosa}
Un fabricante tiene una m\'aquina complicada. Al comienzo de cada d\'ia que la m\'aquina est\'a operativa el riesgo de que la misma se desconponga es $p$. Cuando la m\'aquina se descompone se llama inmediatamente a la agencia de mantenimiento para agendar una visita para el d\'ia siguiente. La visita de la agencia dura $k$ d\'ias y cuesta $\alpha$ d\'olares diarios. Durante la visita los t\'ecnicos instalan un repuesto que tiene una vida \'util m\'inima de $m$ d\'ias. Cada visita sucede de la siguiente manera: 
\begin{itemize}
\item Supongamos que la m\'aquina empieza la ma\~nana del d\'ia $t$ operativa y que durante ese d\'ia la misma se descompone un par de horas antes del final de la jornada. El fabricante entonces llama a la agencia de mantenimiento para agendar una visita. 
\item La ma\~nana del d\'ia $t+1$ llegan los t\'ecnicos de la agencia y empiezan a trabajar en la m\'aquina. Trabajan todo el d\'ia. 
\item Los d\'ias $t+2$, $t+3$, $\dots$, $t+k-1$, los t\'ecnicos de la agencia contin\'uan su trabajo. 
\item La ma\~nana del d\'ia $t+k$ los t\'ecnicos de la agencia contin\'uan su trabajo y le entregan la m\'aquina operativa al fabricante para el final de la jornada. 
\item La ma\~nana del d\'ia $t+k+1$ la m\'aquina est\'a operativa nuevamente, y continua as\'i hasta el d\'ia $t+k+m$ gracias al repuesto especial que instalan los t\'ecnicos. 
\item Al empezar el d\'ia $t+k+m+1$ la m\'aquina puede volverse a descomponer con probabilidad igual a $p$. 
\end{itemize}

Con esto en mente: 
\begin{enumerate}[label=\textbf{\alph*)}]
\item \textbf{2 Puntos:} Modele la situaci\'on descrita como una Cadena de Markov. En particular, presente el grafo de la cadena en funci\'on de $p$ y $k$. 
\item \textbf{3 Puntos:} Calcule: 
\begin{itemize}
\item El porcentaje del tiempo que la m\'aquina opera toda la jornada sin problemas. 
\item El porcentaje del tiempo que la m\'aquina empieza el d\'ia operativa pero se descompone durante alg\'un momento de la jornada. 
\item El porcentaje del tiempo que la m\'aquina recibe mantenimiento. 
\end{itemize}

\end{enumerate}

\end{problem}
\fullskip

% -----------------------------------------------------------------
\begin{problem}
Un inversionista de riesgo se encuentra evaluando varias nuevas empresas, para lo cual las ha clasificado de la siguiente manera: 
\begin{itemize}
\item Las empresas rango $A$ son las mejores. Cada trimestre una empresa rango $A$ logra volverse totalmente rentable con probablidad del 20\%, se mantiene en el mismo rango con probabilidad del 50\% y desciende de rango con probabilidad del 30\%. 
\item Las empresas rango $B$ son las segundas mejores. Cada trimestre una empresa rango $B$ \linebreak asciende a rango $A$ con probabilidad del 25\%, se mantiene en el mismo rango con probabilidad del 55\% y desciende de rango con probabilidad del 20\%. 
\item Las empresas rango $C$ son las problem\'aticas. Cada trimestre una empresa rango $C$ asciende a rango $B$ con probabilidad del 30\%, se mantiene en el mismo rango con probabilidad del 50\% y quiebra con probabilidad del 20\%. 
\end{itemize}

Con esto en mente, complete las siguientes actividades: 
\begin{enumerate}[label=\textbf{\alph*)}]
\item \textbf{1 Punto:} Modele el sistema de clasificaci\'on de empresas del inversionista como una Cadena de Markov con cinco estados. En particular, provea el grafo de la cadena. 
\item \textbf{3 Puntos:} Suponiendo que cada acci\'on de una empresa rango $A$ genera \$500 mensuales, que cada acci\'on de una empresa rango $B$ genera \$350 mensuales, y que cada acci\'on de una empresa rango $C$ genera \$120 mensuales, encuentre el valor acumulado de una acci\'on de una empresa tipo $A$ desde que se compra la acci\'on hasta que la empresa quiebra o se vuelve completamente rentable. 
\end{enumerate}

\end{problem}
\fullskip

% -----------------------------------------------------------------
\begin{problem}
\textbf{[6 Puntos]} Un acaudalado inversionista tiene una empresa que se dedica a identificar \emph{start-ups} prometedoras. De las empresas que \'el usualmente eval\'ua, se sabe que generalmente solo el 40\% tendr\'a \'exito. Actualmente, el inversionista logra identificar correctamente a empresas exitosas con probabilidad del 85\% y a empresas no-exitosas con probabilidad del 75\%. Se supone que \'el invierte si pronostica que la empresa ser\'a exitosa; caso contrario no invierte. Cuando invierte y acierta sus ganancias asciended a los \$180K, mientras que cuando invierte y se equivoca sus p\'erdidas usualmente acumulan unos \$40K. 

Suponga que un consultor dice poder predecir correctamente el futuro de una empresa con probabilidad $p$. Si el consultor ofrece sus servicios por un honorario de \$35K, cu\'al es el m\'inimo valor de $p$ que garantiza que lo contrate el inversionista? 

\end{problem}
\fullskip

\end{document}
