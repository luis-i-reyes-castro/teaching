% -----------------------------------------------------------------
% Document class: Article
\documentclass[ a4paper, twoside, 11pt]{article}
\usepackage{../../macros-general}
\usepackage{../../macros-article}
% Number of the handout, quiz, exam, etc.
\newcommand{\numero}{05}
\setcounter{numero}{\numero}

% -----------------------------------------------------------------
\begin{document}
\allowdisplaybreaks

% Indices
\newcommand{\iava}{$i$\tsup{ava} }
\newcommand{\iavo}{$i$\tsup{avo} }
\newcommand{\java}{$j$\tsup{ava} }
\newcommand{\javo}{$j$\tsup{avo} }
\newcommand{\kava}{$k$\tsup{ava} }
\newcommand{\kavo}{$k$\tsup{avo} }
\newcommand{\tava}{$t$\tsup{ava} }
\newcommand{\tavo}{$t$\tsup{avo} }
\newcommand{\tmava}{$(t-1)$\tsup{ava} }
\newcommand{\tmavo}{$(t-1)$\tsup{avo} }
\newcommand{\tMava}{$(t+1)$\tsup{ava} }
\newcommand{\tMavo}{$(t+1)$\tsup{avo} }

\begin{center}
\Large Modelos Estoc\'asticos (INDG-1008): Lecci\'on \numero \\[1ex]
\small \textbf{Semestre:} 2018-2019 T\'ermino I \qquad
\textbf{Instructor:} Luis I. Reyes Castro
\end{center}
\fullskip

% -----------------------------------------------------------------
\begin{problem}
Un sistema de colas tiene tres servidores y una sala de espera con capacidad para seis clientes. Los clientes arriban de acuerdo a un proceso Poisson con tasa media de 20 por hora, pero la probabilidad de que un cliente decida ingresar al sistema decae 10\% por cada cliente en cola. Los tiempos de servicio tiene distribuci\'on exponencial con tasa media de 6.5 minutos. 

Con todo esto en mente, complete las siguientes actividades: 
\begin{enumerate}[label=\textbf{\alph*)}]
\item Modele este sistema de colas como una Cadena de Markov en Tiempo Continuo. 
\item Encuentre la distribuci\'on estacionaria de la cadena. 
\item Calcule las m\'etricas de desempe\~no. 
\item Calcule la probabilidad de que un nuevo cliente que arriba encuentre al menos dos clientes en cola. 
\end{enumerate}
\QED

\end{problem}
\fullskip


\end{document}
