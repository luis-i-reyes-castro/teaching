% -----------------------------------------------------------------
% Document class: Article
\documentclass[ a4paper, twoside, 11pt]{article}
\usepackage{../../macros-general}
\usepackage{../../macros-article}
% Number of the handout, quiz, exam, etc.
\newcommand{\numero}{01}
\setcounter{numero}{\numero}

% -----------------------------------------------------------------
\begin{document}
\allowdisplaybreaks

% Indices
\newcommand{\iava}{$i$\tsup{ava} }
\newcommand{\iavo}{$i$\tsup{avo} }
\newcommand{\java}{$j$\tsup{ava} }
\newcommand{\javo}{$j$\tsup{avo} }
\newcommand{\kava}{$k$\tsup{ava} }
\newcommand{\kavo}{$k$\tsup{avo} }
\newcommand{\tava}{$t$\tsup{ava} }
\newcommand{\tavo}{$t$\tsup{avo} }
\newcommand{\tmava}{$(t-1)$\tsup{ava} }
\newcommand{\tmavo}{$(t-1)$\tsup{avo} }
\newcommand{\tMava}{$(t+1)$\tsup{ava} }
\newcommand{\tMavo}{$(t+1)$\tsup{avo} }

\begin{center}
\Large Modelos Estoc\'asticos (INDG-1008): Lecci\'on \numero \\[2ex]
\small \textbf{Semestre:} 2018-2019 T\'ermino I \qquad
\textbf{Instructor:} Luis I. Reyes Castro
\end{center}
\fullskip

% -----------------------------------------------------------------
\begin{problem}
GFC (Guayaquil Fried Chicken) est\'a a punto de lanzar su nueva comida r\'apida Wings ‘N Things a nivel nacional. El departamento de investigaci\'on est\'a convencido de que Wings ‘N Things ser\'a un gran \'exito y desea presentarlo de inmediato en todas las tiendas de distribuci\'on de AFC sin publicidad. El departamento de mercadotecnia ve la situaci\'on de forma diferente y desea lanzar una intensa campa\~na publicitaria. La campa\~na publicitaria costar\'a \$100,000, y hay 70\% de probabilidades de que tenga \'exito con ingresos de \$950,000. Si la campaña no tiene éxito, el ingreso estimado bajar\'a a \$200,000. Si no se utiliza publicidad, el ingreso se estima en \$400,000 con una probabilidad del 80\% si los clientes son receptivos al nuevo producto, y de \$200,000 con probabilidad del 20\% si no lo son. 

Con esto en mente, realice las siguientes actividades: 
\begin{enumerate}[label=\alph*)]
\item (\textbf{3 Puntos}) Desarrolle el \'arbol de decisi\'on asociado con este problema. 
\item (\textbf{5 Puntos}) Resuelva el \'arbol anterior y describa la decisi\'on o pol\'itica \'optima, indicando tambi\'en la respectiva utilidad \'optima. 
\end{enumerate}

\end{problem}
\vspace{\baselineskip}

\end{document}
