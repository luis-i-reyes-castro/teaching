% -----------------------------------------------------------------
% Document class: Article
\documentclass[ a4paper, twoside, 11pt]{article}
\usepackage{../../../macros-general}
\usepackage{../../../macros-article}
% Number of the handout, quiz, exam, etc.
\newcommand{\numero}{01}
\setcounter{numero}{\numero}

% -----------------------------------------------------------------
\begin{document}
\allowdisplaybreaks

\begin{center}
\Large Modelos Estoc\'asticos (INDG-1008): Examen \numero \\[1ex]
\small \textbf{Semestre:} 2017-2018 T\'ermino I \qquad
\textbf{Instructor:} Luis I. Reyes Castro
\end{center}
\fullskip

% Indices
\newcommand{\iava}{$i$\tsup{ava} }
\newcommand{\iavo}{$i$\tsup{avo} }
\newcommand{\java}{$j$\tsup{ava} }
\newcommand{\javo}{$j$\tsup{avo} }
\newcommand{\kava}{$k$\tsup{ava} }
\newcommand{\kavo}{$k$\tsup{avo} }
\newcommand{\tava}{$t$\tsup{ava} }
\newcommand{\tavo}{$t$\tsup{avo} }
\newcommand{\tmava}{$(t-1)$\tsup{ava} }
\newcommand{\tmavo}{$(t-1)$\tsup{avo} }
\newcommand{\tMava}{$(t+1)$\tsup{ava} }
\newcommand{\tMavo}{$(t+1)$\tsup{avo} }
\fbox{

\begin{minipage}[b][\height][t]{\textwidth}
\vspace{0.2 cm}

\begin{center}
\textbf{COMPROMISO DE HONOR}
\end{center}
\vspace{0.4 cm}

\scriptsize
{
Yo, \rule{60mm}{.1pt} al firmar este compromiso, reconozco que la presente lecci\'on est\'a dise\~nada para ser resuelta de manera individual, que puedo usar un l\'apiz o pluma y una calculadora cient\'ifica, \linebreak que solo puedo comunicarme con la persona responsable de la recepci\'on de la lecci\'on, y que cualquier instrumento de comunicaci\'on que hubiere tra\'ido debo apagarlo. Tambi\'en estoy conciente que no debo consultar libros, notas, \linebreak ni materiales did\'acticos adicionales a los que el instructor entregue durante la lecci\'on o autorice a utilizar. Finalmente, me comprometo a desarrollar y presentar mis respuestas de manera clara y ordenada. \\

Firmo al pie del presente compromiso como constancia de haberlo le\'ido y aceptado. 
\vspace{0.4 cm}

Firma: \rule{60mm}{.1pt} \qquad N\'umero de matr\'icula: \rule{40mm}{.1pt} \hspace{0.5cm} \\[-0.8ex]
}

\end{minipage}

}

\vspace{\baselineskip}



% -----------------------------------------------------------------
\begin{problem}
\textbf{[4 Puntos]} El call center de una empresa de servicios al consumidor recibe en promedio, $\lambda_1 = 11.9$ llamadas por hora para Servicio al Cliente y $\lambda_2 = 21.4$ llamadas por hora para Servicio T\'ecnico. De los clientes que llaman para Servicio al Cliente el $p_{12} = 5.3\%$ es referido a Servicio T\'ecnico, mientras que de los clientes que llaman a Servicio T\'ecnico el $p_{21} = 17.6\%$ es referido a Servicio al Cliente. Con esto en mente, calcule el n\'umero promedio de clientes por hora que debe atender el departamento de Servicio al Cliente y el deparatamento de Servicio T\'ecnico. 

\emph{Sugerencia:} Piense en t\'erminos de divisi\'on y combinaci\'on de procesos. 

\emph{Soluci\'on:} Pensemos en las llamadas como siendo generadas por dos Procesos Poisson con intensidades $\lambda_1$ y $\lambda_2$. En este contexto, podemos pensar que en cada arribo al primer proceso se lanza una moneda sesgada con probabilidad de cara igual a $p_{12}$, independiente de todos los otros arribos. Si sale cara entonces el arribo es enviado al segundo proceso, caso contrario este se mantiene en el primer proceso. Con estas consideraciones en mente, recordando nuestros resultados para divisi\'on y combinaci\'on de procesos Poisson vemos que el n\'umero promedio de clientes por hora que debe atender el departamento de Servicio al Cliente es: 
\[
\lambda_1 \, ( 1 - p_{12} ) + \lambda_2 \, p_{21} \; = \;
(11.9)(1-0.053) + (21.4)(0.176) \; = \; 15.04
\]
Similarmente, el n\'umero promedio de clientes por hora que debe atender el departamento de Servicio T\'ecnico es: 
\[
\lambda_1 \, p_{12} + \lambda_2 \, ( 1 - p_{21} ) \; = \;
(11.9)(0.053) + (21.4)(1-0.176) \; = \; 18.26
\]

\end{problem}
\vspace{\baselineskip}

% -----------------------------------------------------------------
\begin{problem}
Una m\'aquina caprichosa tiene el siguiente comportamiento: 
\begin{itemize}
\item Si la m\'aquina termina el d\'ia en buen estado, la misma terminar\'a el siguiente d\'ia en buen estado con probabilidad $p$. Caso contrario terminar\'a averiada. 
\item Si la m\'aquina termina el d\'ia averiada entonces los t\'ecnicos se tomar\'an todo el siguiente d\'ia para intentar arreglarla. Con probabilidad $q$ lograr\'an arreglar la m\'aquina; caso contrario, tendr\'an que volverlo a intentar el siguiente d\'ia. 
\end{itemize}

Con todo esto en mente: 
\begin{enumerate}[label=\alph*.]
\item \textbf{[2 Puntos]} Construya un modelo de Cadena de Markov del comporamiento de esta m\'aquina. En particular, bosqueje el grafo y construya la matriz de transici\'on. 
\item \textbf{[4 Puntos]} Encuentre la distribuci\'on en estado estable de su modelo. 
\end{enumerate}

\emph{Soluci\'on al literal (a):} Los estados son: 
\begin{enumerate}
\item La m\'aquina termina el d\'ia en buen estado. 
\item La m\'aquina termina el d\'ia averiada. 
\end{enumerate}
La matriz de transici\'on es: 
\[
\vec{P} \; = \;
\left[
\begin{array}{cc}
p & 1-p \\ q & 1-q
\end{array}
\right]
\]

\emph{Soluci\'on al literal (b):} La ecuaci\'on de estado estable para el primer estado es: 
\[
\pi_1 \; = \; p \, \pi_1 + q \, \pi_2
\]
Consecuentemente: 
\[
\pi_2 \; = \; \left( \frac{1-p}{q} \right) \pi_1
\]
A su vez, dado que $\pi_1 + \pi_2 = 1$, vemos que: 
\[
\left( 1 + \frac{1-p}{q} \right) \pi_1 \; = \; 1
\qquad \Longrightarrow \qquad
\pi_1 \; = \; \frac{q}{1-p+q}
\qquad \Longrightarrow \qquad
\pi_2 \; = \; \frac{1-p}{1-p+q}
\]

\end{problem}
\vspace{\baselineskip}

% -----------------------------------------------------------------
\begin{problem}
\textbf{[7 Puntos]} Un operador de servicios de telefon\'ia celular est\'a en proceso de instalar una antena en un barrio promedio, donde se puede esperar que la antena reciba un paquete de datos para su transmisi\'on durante cada ciclo (\eg durante cada milisegundo) con probabilidad $\lambda \in (0.99,1)$. Para poder satisfacer esta demanda se instal\'o un buffer con capacidad para $M$ paquetes de datos junto con $n$ transmisores que operan en canales independientes pero ruidosos; en particular, cada transmisor que es encargado con el env\'io de un paquete logra transmitirlo exitosamente con probabilidad $\mu \in (0.94,0.98)$. Cuando una transmisi\'on fracasa se mantiene al paquete en el buffer y se reintenta la transmisi\'on en el siguiente per\'iodo. 

Cada ciclo de operaci\'on, digamos el \tavo ciclo, avanza de la siguiente manera: 
\begin{enumerate}
\item Se empieza el ciclo con $X_{t-1}$ paquetes en el b\'uffer. 
\item Si el buffer no est\'a lleno, se puede recibir un nuevo paquete con probabilidad $\lambda$, de tal manera que el nuevo n\'umero de paquetes en el buffer es: 
\[
\min \; \{ \; X_{t-1} + D_t \, , \; M \, \} \, ,
\qquad \text{donde } D_t \sim \Bernoulli(\lambda)
\]
\item Los transmisores intentan enviar cuantos paquetes puedan. Si hay $n$ paquetes o m\'as en el buffer, entonces cada uno de los transmisores es asignado aleatoriamente a un \'unico paquete, y cada transmisor logra enviar su paquete con \'exito con probabilidad $\mu$, independiente de los otros. Si hay menos de $n$ paquetes en el buffer se opera de la misma manera, pero en este caso habr\'a uno o m\'as transmisores a los que no ser\'a necesario asignarles paquetes durante este ciclo. 
\end{enumerate}

Con todo esto en mente, construya un modelo de Cadena de Markov de este proceso para el caso particular cuando $M = 5$ y $n = 3$. En particular, explique cuales son los estados y liste todas las probabilidades de transici\'on positivas. 

\emph{Nota:} En vez de listar las probabilidades usted puede construir la matriz de transici\'on siempre y cuando la desarrolle con suficiente espacio, \eg en una hoja vac\'ia en formato retrato. 

\emph{Sugerencia:} La suma de $k$ variables aleatorias i.i.d. con distribuci\'on $\Bernoulli(\mu)$ es una variable aleatoria con distribuci\'on $\Binomial(k,\mu)$. 

\textbf{Soluci\'on:} Consideramos cada posible estado: 
\begin{itemize}
% ---------------------------------------------------------
\item Si $X_{t-1} = 0$ entonces: 
\begin{itemize}
\item Si $X_t = 0$ es porque hubo un arribo de paquete y una transmisi\'on exitosa o porque no hubo un arribo de paquete, \iec
\[
\vec{P}(0,0) \; = \; \lambda \, \mu + (1-\lambda)
\]
\item Si $X_t = 1$ es porque hubo un arribo de paquete y una transmisi\'on fallida, \iec
\[
\vec{P}(0,1) \; = \; \lambda \, (1-\mu)
\]
\end{itemize}
% ---------------------------------------------------------
\item Si $X_{t-1} = 1$ entonces: 
\begin{itemize}
\item Si $X_t = 0$ es porque hubo un arribo de paquete y dos de las dos transmisiones fueron exitosas o porque no hubo un arribo de paquete y la \'unica transmisi\'on fue exitosa, \iec
\[
\vec{P}(1,0) \; = \; \lambda \, \mu^2 + (1-\lambda) \, \mu
\]
\item Si $X_t = 1$ es porque hubo un arribo de paquete y una de las dos transmisiones fue exitosa o porque no hubo un arribo de paquete y la \'unica transmisi\'on fue fallida, \iec
\[
\vec{P}(1,1) \; = \; \lambda \, { 2 \choose 1 } \, \mu \, (1-\mu) + (1-\lambda) \, (1-\mu)
\]
\item Si $X_t = 2$ es porque hubo un arribo de paquete y dos de las dos transmisiones fueron fallidas, \iec
\[
\vec{P}(1,2) \; = \; \lambda \, (1-\mu)^2
\]
\end{itemize}
% ---------------------------------------------------------
\item Si $X_{t-1} = 2$ entonces: 
\begin{itemize}
\item Si $X_t = 0$ es porque hubo un arribo de paquete y tres de las tres transmisiones fueron exitosas o porque no hubo un arribo de paquete y dos de las dos transmisiones fueron exitosas, \iec
\[
\vec{P}(2,0) \; = \; \lambda \, \mu^3 + (1-\lambda) \, \mu^2
\]
\item Si $X_t = 1$ es porque hubo un arribo de paquete y dos de las tres transmisiones fueron exitosas o porque no hubo un arribo de paquete y una de las dos transmisiones fue exitosa, \iec
\[
\vec{P}(2,1) \; = \; \lambda \, { 3 \choose 2 } \, \mu^2 \, (1-\mu) + (1-\lambda) \, { 2 \choose 1 } \, \mu \, (1-\mu)
\]
\item Si $X_t = 2$ es porque hubo un arribo de paquete y una de las tres transmisiones fue exitosa o porque no hubo un arribo de paquete y dos de las dos transmisiones fueron fallidas, \iec
\[
\vec{P}(2,2) \; = \; \lambda \, { 3 \choose 1 } \, \mu \, (1-\mu)^2 + (1-\lambda) \, (1-\mu)^2
\]
\item Si $X_t = 3$ es porque hubo un arribo de paquete y tres de las tres transmisiones fueron fallidas, \iec
\[
\vec{P}(2,3) \; = \; \lambda \, (1-\mu)^3
\]
\end{itemize}
% ---------------------------------------------------------
\item Si $X_{t-1} = 3$ entonces: 
\begin{itemize}
\item Si $X_t = 0$ es porque no hubo un arribo de paquete y tres de las tres transmisiones fueron exitosas, \iec
\[
\vec{P}(3,0) \; = \; (1-\lambda) \, \mu^3
\]
\item Si $X_t = 1$ es porque hubo un arribo de paquete y tres de las tres transmisiones fueron exitosas o porque no hubo un arribo de paquete y dos de las tres transmisiones fueron exitosas, \iec
\[
\vec{P}(3,1) \; = \; \lambda \, \mu^3 + (1-\lambda) \, { 3 \choose 2 } \, \mu^2 \, (1-\mu)
\]
\item Si $X_t = 2$ es porque hubo un arribo de paquete y dos de las tres transmisiones fueron exitosas o porque no hubo un arribo de paquete y una de las tres transmisiones fue exitosa, \iec
\[
\vec{P}(3,2) \; = \; \lambda \, { 3 \choose 2 } \, \mu^2 \, (1-\mu) + (1-\lambda) \, { 3 \choose 1 } \, \mu \, (1-\mu)^2
\]
\item Si $X_t = 3$ es porque hubo un arribo de paquete y una de las tres transmisiones fue exitosa o porque no hubo un arribo de paquete y tres de las tres transmisiones fueron fallidas, \iec
\[
\vec{P}(3,3) \; = \; \lambda \, { 3 \choose 1 } \, \mu \, (1-\mu)^2 + (1-\lambda) \, (1-\mu)^3
\]
\item Si $X_t = 4$ es porque hubo un arribo de paquete y tres de las tres transmisiones fueron fallidas, \iec
\[
\vec{P}(3,4) \; = \; \lambda \, (1-\mu)^3
\]
\end{itemize}
% ---------------------------------------------------------
\item Si $X_{t-1} = 4$ entonces: 
\begin{itemize}
\item Si $X_t = 1$ es porque no hubo un arribo de paquete y tres de las tres transmisiones fueron exitosas, \iec
\[
\vec{P}(4,1) \; = \; (1-\lambda) \, \mu^3
\]
\item Si $X_t = 2$ es porque hubo un arribo de paquete y tres de las tres transmisiones fueron exitosas o porque no hubo un arribo de paquete y dos de las tres transmisiones fueron exitosas, \iec
\[
\vec{P}(4,2) \; = \; \lambda \, \mu^3 + (1-\lambda) \, { 3 \choose 2 } \, \mu^2 \, (1-\mu)
\]
\item Si $X_t = 3$ es porque hubo un arribo de paquete y dos de las tres transmisiones fueron exitosas o porque no hubo un arribo de paquete y una de las tres transmisiones fue exitosa, \iec
\[
\vec{P}(4,3) \; = \; \lambda \, { 3 \choose 2 } \, \mu^2 \, (1-\mu) + (1-\lambda) \, { 3 \choose 1 } \, \mu \, (1-\mu)^2
\]
\item Si $X_t = 4$ es porque hubo un arribo de paquete y una de las tres transmisiones fue exitosa o porque no hubo un arribo de paquete y tres de las tres transmisiones fueron fallidas, \iec
\[
\vec{P}(4,4) \; = \; \lambda \, { 3 \choose 1 } \, \mu \, (1-\mu)^2 + (1-\lambda) \, (1-\mu)^3
\]
\item Si $X_t = 5$ es porque hubo un arribo de paquete y tres de las tres transmisiones fueron fallidas, \iec
\[
\vec{P}(4,5) \; = \; \lambda \, (1-\mu)^3
\]
\end{itemize}
% ---------------------------------------------------------
\item Si $X_{t-1} = 5$ entonces: 
\begin{itemize}
\item Si $X_t = 2$ es porque tres de las tres transmisiones fueron exitosas, \iec
\[
\vec{P}(5,2) \; = \; \mu^3
\]
\item Si $X_t = 3$ es porque dos de las tres transmisiones fueron exitosas, \iec
\[
\vec{P}(5,3) \; = \; { 3 \choose 2 } \, \mu^2 \, (1-\mu)
\]
\item Si $X_t = 4$ es porque una de las tres transmisiones fue exitosa, \iec
\[
\vec{P}(5,4) \; = \; { 3 \choose 1 } \, \mu \, (1-\mu)^2
\]
\item Si $X_t = 5$ es porque tres de las tres transmisiones fueron fallidas, \iec
\[
\vec{P}(5,5) \; = \; (1-\mu)^3
\]
\end{itemize}
\end{itemize}


\end{problem}
\vspace{\baselineskip}

%% -----------------------------------------------------------------
%\begin{problem}
%Considere un proceso estoc\'astico Markoviano $X_0, \, X_1, \, X_2,\, \dots$ cuyos estados son los enteros no-negativos. El proceso evoluciona de la siguiente manera: 
%\begin{itemize}
%\item Si $X_t = 0$ entonces: 
%\[
%X_{t+1} \; = \; 
%\begin{cases}
%1, & \text{con probabilidad } p \\
%0, & \text{con probabilidad } 1-p 
%\end{cases}
%\]
%\item Si $X_t > 0$ entonces: 
%\[
%X_{t+1} \; = \; 
%\begin{cases}
%X_t + 1, & \text{con probabilidad } p \\
%X_t - 1, & \text{con probabilidad } q \\
%X_t, & \text{con probabilidad } 1-p-q
%\end{cases}
%\]
%\end{itemize}
%
%Con todo esto en mente: 
%\begin{enumerate}[label=\alph*.]
%\item \textbf{[2 Puntos]} Construya un modelo de Cadena de Markov de nacimiento-muerte para este proceso. En particular, bosqueje el grafo para los primeros cuatro estados. 
%\item \textbf{[3 Puntos]} Escriba las ecuaciones de estado estable para los primeros tres estados. 
%\item \textbf{[2 Puntos]} Escriba las probabilidades en estado estable de los n\'umeros uno y dos en funci\'on de la probabilidad en estado estable del n\'umero cero. 
%\item \textbf{[3 Puntos]} Escriba una expresi\'on para la probabilidad en estado estable de cualquier n\'umero $i \geq 1$ como funci\'on de la probabilidad en estado estable del n\'umero cero. 
%\end{enumerate}
%
%\end{problem}
%\vspace{\baselineskip}

\end{document}
