% -----------------------------------------------------------------
% Document class: Article
\documentclass[ a4paper, twoside, 11pt]{article}
\usepackage{../../macros-general}
\usepackage{../../macros-article}
%\graphicspath{{./figures/}}
% Number of the handout, quiz, exam, etc.
\newcommand{\numero}{04}
\setcounter{numero}{\numero}

% -----------------------------------------------------------------
\begin{document}
\allowdisplaybreaks

% Indices
\newcommand{\iava}{$i$\tsup{ava} }
\newcommand{\iavo}{$i$\tsup{avo} }
\newcommand{\java}{$j$\tsup{ava} }
\newcommand{\javo}{$j$\tsup{avo} }
\newcommand{\kava}{$k$\tsup{ava} }
\newcommand{\kavo}{$k$\tsup{avo} }
\newcommand{\tava}{$t$\tsup{ava} }
\newcommand{\tavo}{$t$\tsup{avo} }
\newcommand{\tmava}{$(t-1)$\tsup{ava} }
\newcommand{\tmavo}{$(t-1)$\tsup{avo} }
\newcommand{\tMava}{$(t+1)$\tsup{ava} }
\newcommand{\tMavo}{$(t+1)$\tsup{avo} }

\begin{center}
\Large Modelos Estoc\'asticos (INDG-1008): Lecci\'on \numero \\[2ex]
\small \textbf{Semestre:} 2017-2018 T\'ermino II \qquad
\textbf{Instructor:} Luis I. Reyes Castro
\end{center}
\fullskip

% -----------------------------------------------------------------
\begin{problem}
\textbf{[12 Puntos]} Suponga que en un futuro cercano una empresa se dedica a transportar pasageros entre sus dos estaciones en Guayaquil y Salinas utilizando carros aut\'onomos. \linebreak La empresa tiene una flota de $N = 5$ carros que oscila entre sus dos estaciones, y cada carro puede ser utilizado no m\'as de una vez al d\'ia. Todas las ma\~nanas, cada cada carro que amanece en la estaci\'on de Guayaquil es utilizado por alg\'un cliente para ir a Salinas con probabilidad igual a $p$. Similarmente, cada cada carro que amanece en la estaci\'on de Salinas es utilizado por alg\'un cliente para ir a Guayaquil con probabilidad $q$. La empresa percibe ingresos de $u$ d\'olares por viaje, sin importar su direcci\'on. 

Puesto que usualmente $p > q$, \ie en promedio m\'as clientes quieren viajar de Guayaquil a Salinas que de Salindas a Guayaquil, la empresa usualmente debe re-balancear su flota, lo cual siempre hace de noche despu\'es de cerrar sus operaciones por el d\'ia. Para esto se ordena a los carros manejarse vac\'ios de una estaci\'on a otra. El costo de cada viaje vac\'io de re-balanceo es de $c$ d\'olares. Por seguridad, un ser humano debe monitorear el viaje de los veh\'iculos en una computadora, por lo que el n\'umero de carros que se re-balancea por noche no puede exceder de $M = 2$. Por esta misma raz\'on, se incurre un costo fijo de $4c$ d\'olares por re-balanceo, sin importar el n\'umero de carros movidos. 

Con todo esto en mente, modele el problema de encontrar una pol\'itica \'optima de re-balanceo como un Proceso de Decisi\'on Markoviano (PDM). Al construir su modelo, por favor defina sus estados de tal manera que en cada per\'iodo primero se toma la decisi\'on de re-balanceo, luego se ordena a los carros viajar vac\'ios, de ser necesario, y finalmente los clientes utilizan los carros para movilizarse entre las estaciones. 

\emph{Nota:} Suponga que todos los carros aut\'onomos son id\'enticos. De esta manera, usted puede darse cuenta f\'acilmente que los viajes de re-balanceo siempre ser\'an en una sola direcci\'on. 

\emph{Sugerencia:} Para ahorrarse tiempo al describir las probabilidades de transici\'on para cada par estado-acci\'on, por favor utilice la siguiente notaci\'on: 
\begin{align*}
\alpha(k,m) \; & \define \; \Pr( \, \Binomial(p,m) = k \, ) \\
\beta(\ell,n) \; & \define \; \Pr( \, \Binomial(q,n) = \ell \, )
\end{align*}

\end{problem}
\fullskip

\end{document}
