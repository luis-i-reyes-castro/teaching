% -----------------------------------------------------------------
% Document class: Article
\documentclass[ a4paper, twoside, 11pt]{article}
\usepackage{../../macros-general}
\usepackage{../../macros-article}
\graphicspath{{./figures/}}
% Uncomment the following line if you need to write Python code
% \usepackage{minted}

% Number of the handout, quiz, exam, etc.
\newcommand{\numero}{01}
\setcounter{numero}{\numero}

% -----------------------------------------------------------------
\begin{document}
\allowdisplaybreaks

% Indices
\newcommand{\iava}{$i$\tsup{ava} }
\newcommand{\iavo}{$i$\tsup{avo} }
\newcommand{\java}{$j$\tsup{ava} }
\newcommand{\javo}{$j$\tsup{avo} }
\newcommand{\kava}{$k$\tsup{ava} }
\newcommand{\kavo}{$k$\tsup{avo} }
\newcommand{\tava}{$t$\tsup{ava} }
\newcommand{\tavo}{$t$\tsup{avo} }
\newcommand{\tmava}{$(t-1)$\tsup{ava} }
\newcommand{\tmavo}{$(t-1)$\tsup{avo} }
\newcommand{\tMava}{$(t+1)$\tsup{ava} }
\newcommand{\tMavo}{$(t+1)$\tsup{avo} }

\begin{center}
\Large Modelos Estoc\'asticos (INDG-1008): Trabajo Aut\'onomo \numero \\[1ex]
\small \textbf{Semestre:} 2017-2018 T\'ermino II \qquad
\textbf{Instructor:} Luis I. Reyes Castro
\end{center}
\fullskip

%\fbox{

\begin{minipage}[b][\height][t]{\textwidth}
\vspace{0.2 cm}

\begin{center}
\textbf{COMPROMISO DE HONOR}
\end{center}
\vspace{0.4 cm}

\scriptsize
{
Yo, \rule{60mm}{.1pt} al firmar este compromiso, reconozco que la presente evaluaci\'on est\'a dise\~nada para ser resuelta de manera individual, que puedo usar un l\'apiz o pluma y una calculadora cient\'ifica, \linebreak que solo puedo comunicarme con la persona responsable de la recepci\'on de la evaluaci\'on, y que cualquier instrumento de comunicaci\'on que hubiere tra\'ido debo apagarlo. Tambi\'en estoy conciente que no debo consultar libros, notas, \linebreak ni materiales did\'acticos adicionales a los que el instructor entregue durante la evaluaci\'on o autorice a utilizar. Finalmente, me comprometo a desarrollar y presentar mis respuestas de manera clara y ordenada. \\

Firmo al pie del presente compromiso como constancia de haberlo le\'ido y aceptado. 
\vspace{0.4 cm}

Firma: \rule{60mm}{.1pt} \qquad N\'umero de matr\'icula: \rule{40mm}{.1pt} \hspace{0.5cm} \\[-0.8ex]
}

\end{minipage}

}

\vspace{\baselineskip}



\halfskip

% -----------------------------------------------------------------
\begin{problem}
\textbf{(2 Puntos)} La biblioteca p\'ublica de Springdale recibe nuevos libros de acuerdo a una distribuci\'on Poisson con media de 25 libros por d\'ia y los exhibe en analaqueles con capacidad para 100 libros. Determine lo siguiente: 
\begin{enumerate}[label=\alph*)]
\item El promedio de anaqueles que se llenar\'an de nuevos libros cada mes (30 d\'ias). 
\item La probabilidad de que se requieran m\'as de 10 libreros cada mes, si un librero se compone de 5 anaqueles.
\end{enumerate}

\end{problem}
\vspace{\baselineskip}

% -----------------------------------------------------------------
\begin{problem}
\textbf{(3 Puntos)} Un coleccionista de arte viaja a subastas de arte una vez al mes en promedio. Cada viaje es seguro que produzca una compra. El tiempo entre viajes est\'a exponencialmente distribuido. Determine lo siguiente: 
\begin{enumerate}[label=\alph*)]
\item La probabilidad de que se realice exactamente una compra en un periodo de 3 meses. 
\item La probabilidad de que se realicen no m\'as de 8 compras por año. 
\item La probabilidad de que el tiempo entre viajes sucesivos exceda de 1 mes. 
\end{enumerate}

\end{problem}
\vspace{\baselineskip}

% -----------------------------------------------------------------
\begin{problem}
\textbf{(2 Puntos)} El tiempo entre llegadas en el restaurante L\&J es exponencial con media de 5 minutos. El restaurante abre a las 11:00 A.M. Determine lo siguiente: 
\begin{enumerate}[label=\alph*)]
\item La probabilidad de que para las 11:12 hallan arribado 10 clientes, dado que para las 11:05 hab\'ian arribado 8 clientes. 
\item Las probabilidad de que un nuevo cliente llegue entre las 11:28 y las 11:33, si el \'ultimo cliente lleg\'o a las 11:25. 
\end{enumerate}

\end{problem}
\vspace{\baselineskip}

% -----------------------------------------------------------------
\begin{problem}
\textbf{(4 Puntos)} La U de A opera dos l\'ineas de autobuses en el campus. \linebreak La l\'inea roja presta servicio al norte del campus, mientras que la l\'inea verde presta servicio \linebreak al sur del campus. Una estaci\'on de transferencia conecta las dos l\'ineas. Los autobuses verdes llegan a la estaci\'on de transferencia de acuerdo a un proceso Poisson con tiempo medio entre arribos de 10 minutos. Los autobuses rojos tienen medio entre arribos de 7 minutos. 
\begin{enumerate}[label=\alph*)]
\item Cu\'al es la probabilidad de que al menos un autobus de cada una de las dos l\'ineas se detengan en la estaci\'on durante un intervalo de 5 minutos?
\item Un estudiante cuyo dormitorio est\'a cerca de la estaci\'on tiene clase en 10 minutos. Cualquiera de los autobuses lo lleva al edificio del sal\'on de clases. El viaje requiere 5 minutos, despu\'es de lo cual el estudiante camina durante aproximadamente 3 minutos para llegar al sal\'on. Cu\'al es la probabilidad de que el estudiante llegue a tiempo a clase?
\end{enumerate}

\end{problem}
\vspace{\baselineskip}

% -----------------------------------------------------------------
\begin{problem}


\end{problem}
\vspace{\baselineskip}

% -----------------------------------------------------------------
\begin{problem}

\end{problem}
\vspace{\baselineskip}

\end{document}
