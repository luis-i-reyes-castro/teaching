% -----------------------------------------------------------------
% Document class: Article
\documentclass[ a4paper, twoside, 11pt]{article}
\usepackage{../../macros-general}
\usepackage{../../macros-article}
% Number of the handout, quiz, exam, etc.
\newcommand{\numero}{02}
\setcounter{numero}{\numero}

% -----------------------------------------------------------------
\begin{document}
\allowdisplaybreaks

% Indices
\newcommand{\iava}{$i$\tsup{ava} }
\newcommand{\iavo}{$i$\tsup{avo} }
\newcommand{\java}{$j$\tsup{ava} }
\newcommand{\javo}{$j$\tsup{avo} }
\newcommand{\kava}{$k$\tsup{ava} }
\newcommand{\kavo}{$k$\tsup{avo} }
\newcommand{\tava}{$t$\tsup{ava} }
\newcommand{\tavo}{$t$\tsup{avo} }
\newcommand{\tmava}{$(t-1)$\tsup{ava} }
\newcommand{\tmavo}{$(t-1)$\tsup{avo} }
\newcommand{\tMava}{$(t+1)$\tsup{ava} }
\newcommand{\tMavo}{$(t+1)$\tsup{avo} }

\begin{center}
\Large Modelos Estoc\'asticos (INDG-1008): Lecci\'on \numero \\[2ex]
\small \textbf{Semestre:} 2017-2018 T\'ermino II \qquad
\textbf{Instructor:} Luis I. Reyes Castro
\end{center}
\fullskip

% -----------------------------------------------------------------
\begin{problem}
\textbf{(5 Puntos)}
Suponga que usted ha sido encargado con del manejo del inventario de algunos productos no-perecederos en un supermercado. Para mantener una consistencia en el manejo de inventario a trav\'es de los varios productos que se venden, el gerente ha dispuesto que todos los inventarios se controlen mediante pol\'iticas de punto de reposici\'on. 

M\'as precisamente, al final de cada d\'ia se cuenta el inventario del producto y si este es menor o igual a $I_{rep}$ unidades se hace un pedido de reposici\'on por $Q_{rep}$ unidades al proveedor, el cual es entregado por el mismo al comienzo del siguiente d\'ia antes de la hora de apertura de la tienda; caso contrario, no se hace un pedido de reposici\'on. F\'ijese que bajo estas suposiciones el m\'aximo inventario posible es:
\[
I_{max} \, = \, I_{rep} + Q_{rep}
\]
Adicionalmente,como es de costumbre en este campo de estudio, usted hace la suposici\'on simplificatoria que las demandas del producto $D_1, \, D_2, \, \dots$ constituyen una secuencia de variables aleatorias i.i.d. con distribuci\'on Poisson con par\'ametro $\lambda$. M\'as a\'un, si para cada d\'ia $t$ denotamos a $I_t$ como el inventario a la hora de apertura de la tienda del \tavo d\'ia, a $D_t$ como la demanda del d\'ia, \ie la cantidad que se vender\'ia ese d\'ia si el inventario fuere inagotable, a $R_t$ como la cantidad repuesta entre la noche del \tavo d\'ia y la hora de apertura del \tMavo d\'ia, tenemos que:
\[
I_{t+1} \, = \, \max \{ \, 0, \, I_t - D_t \, \} + R_t \, ; \qquad
R_t \, = \, 
\begin{cases}
Q_{rep}, & \text{si } 0 \leq I_t \leq I_{rep} \, ; \\
0, & \text{caso contrario;}
\end{cases}
\]

Con todo esto en mente, para cada problema de manejo de inventario con demanda $\lambda$ y pol\'itica de punto de reposici\'on $(I_{rep},Q_{rep})$ podemos construir una Cadena de Markov con \linebreak $n = I_{max} + 1$ estados (cada uno asociado a un nivel de inventario). 

Construya la matriz de transici\'on para el problema para cada uno de los siguientes casos: 
\begin{table}[htb]
\centering
\begin{tabular}{|c|c|c|c|}
\hline
\textbf{Caso}           & $\bm{\lambda}$ & $\bm{I_{rep}}$ & $\bm{Q_{rep}}$ \\ \hline
Producto 1 & 0.8 & 1  & 2                  \\ \hline
Producto 2 & 1.5 & 2  & 2                  \\ \hline
\end{tabular}
\end{table}

Por favor presente sus matrices como tablas con los decimales redondeados a tres d\'igitos. 

\end{problem}
\vspace{\baselineskip}

% -----------------------------------------------------------------
\begin{problem}
Un inversionista de riesgo se encuentra evaluando varias nuevas empresas, para lo cual las ha clasificado de la siguiente manera: 
\begin{itemize}
\item Las empresas rango $A$ son las mejores. Cada trimestre una empresa rango $A$ logra volverse totalmente rentable con probablidad del 20\%, se mantiene en el mismo rango con probabilidad del 50\% y desciende de rango con probabilidad del 30\%. 
\item Las empresas rango $B$ son las segundas mejores. Cada trimestre una empresa rango $B$ \linebreak asciende a rango $A$ con probabilidad del 25\%, se mantiene en el mismo rango con probabilidad del 55\% y desciende de rango con probabilidad del 20\%. 
\item Las empresas rango $C$ son las problem\'aticas. Cada trimestre una empresa rango $C$ asciende a rango $B$ con probabilidad del 30\%, se mantiene en el mismo rango con probabilidad del 50\% y quiebra con probabilidad del 20\%. 
\end{itemize}

Con esto en mente, complete las siguientes actividades: 
\begin{enumerate}[label=\textbf{\alph*)}]
\item \textbf{1 Punto:} Modele el sistema de clasificaci\'on de empresas del inversionista como una Cadena de Markov con cinco estados. En particular, provea el grafo de la cadena. 
\item \textbf{3 Puntos:} Para cada rango de empresa, calcule la probabilidad de que una empresa de ese rango eventualmente \textit{(i)} logre volverse totalmente rentable o \textit{(ii)} quiebre. 
\item \textbf{3 Puntos:} Para cada rango de empresa, calcule el tiempo esperado que transcurre desde que una empresa entra en ese rango hasta que la empresa logra volverse totalmente rentable o quiebra. 
\end{enumerate}

\end{problem}
\vspace{\baselineskip}

\end{document}
