% -----------------------------------------------------------------
% Document class: Article
\documentclass[ a4paper, twoside, 11pt]{article}
\usepackage{../../macros-general}
\usepackage{../../macros-article}
\graphicspath{{./figures/}}
% Number of the handout, quiz, exam, etc.
\newcommand{\numero}{03}
\setcounter{numero}{\numero}

% -----------------------------------------------------------------
\begin{document}
\allowdisplaybreaks

% Indices
\newcommand{\iava}{$i$\tsup{ava} }
\newcommand{\iavo}{$i$\tsup{avo} }
\newcommand{\java}{$j$\tsup{ava} }
\newcommand{\javo}{$j$\tsup{avo} }
\newcommand{\kava}{$k$\tsup{ava} }
\newcommand{\kavo}{$k$\tsup{avo} }
\newcommand{\tava}{$t$\tsup{ava} }
\newcommand{\tavo}{$t$\tsup{avo} }
\newcommand{\tmava}{$(t-1)$\tsup{ava} }
\newcommand{\tmavo}{$(t-1)$\tsup{avo} }
\newcommand{\tMava}{$(t+1)$\tsup{ava} }
\newcommand{\tMavo}{$(t+1)$\tsup{avo} }

\begin{center}
\Large Modelos Estoc\'asticos (INDG-1008): Lecci\'on \numero \\[2ex]
\small \textbf{Semestre:} 2017-2018 T\'ermino II \qquad
\textbf{Instructor:} Luis I. Reyes Castro
\end{center}
\fullskip

% -----------------------------------------------------------------
\begin{problem}
\label{prob:arribo-geometrico-prima}
Considere la siguiente variante del modelo M/M/1, donde $q \in (0,1)$ y la tasa de arribo decae geom\'etricamente con el n\'umero de clientes en cola. 

\begin{figure}[htb]
\centering
\def\svgwidth{0.9\columnwidth}
\input{figures/modelo-MM1-geometrico-prima.eps_tex}
\end{figure}
\halfskip

Complete las siguientes actividades: 
\begin{enumerate}[label=\textbf{\alph*)}]
\item \textbf{3 Puntos:} Calcule la distribuci\'on estacionaria del sistema, en los casos cuando existe, como funci\'on de $\lambda$, $\mu$ y $k$. 
\item \textbf{2 Puntos:} Calcule las m\'etricas de desempe\~no $L$ y $L_q$. 
\item \textbf{2 Puntos:} Calcule la tasa de arribo promedio $\bar{\lambda}$ junto con las m\'etricas $W$ y $W_q$. 
\end{enumerate}

\end{problem}
\fullskip

% -----------------------------------------------------------------
\begin{problem}
\label{prob:m_m_infinito}
En un restaurante \emph{all-you-can-eat} (\ie ``todo lo que puedas comer'') los clientes arriban de acuerdo a un proceso Poisson con tasa $\lambda$ por hora. El restaurante ofrece un variado buffet, y los clientes se sirven a si mismos; en particular, la distribuci\'on del tiempo de auto-servicio es exponencial con par\'amatro $\mu$ por hora. El restaurante tiene capacidad para hasta $C$ clientes. 

Modelando este sistema como una cola M/M/$\infty$/$C$, \ie como una cola M/M/$K$/$C$ con infinitos servidores, complete las siguientes actividades: 
\begin{enumerate}[label=\textbf{\alph*)}]
\item \textbf{2 Puntos:} Calcule la distribuci\'on estacionaria del sistema, en los casos cuando existe, como funci\'on de $\lambda$, $\mu$ y $k$. 
\item \textbf{3 Puntos:} Calcule las m\'etricas de desempe\~no $L$ y $L_q$, $W$ y $W_q$. 
\end{enumerate}

\end{problem}
\fullskip

\end{document}
