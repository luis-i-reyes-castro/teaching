% -----------------------------------------------------------------
% Document class: Article
\documentclass[ a4paper, twoside, 11pt]{article}
\usepackage{../../macros-general}
\usepackage{../../macros-article}
\graphicspath{{./figures/}}
% Uncomment the following line if you need to write Python code
% \usepackage{minted}

% Number of the handout, quiz, exam, etc.
\newcommand{\numero}{02}
\setcounter{numero}{\numero}

% -----------------------------------------------------------------
\begin{document}
\allowdisplaybreaks

% Indices
\newcommand{\iava}{$i$\tsup{ava} }
\newcommand{\iavo}{$i$\tsup{avo} }
\newcommand{\java}{$j$\tsup{ava} }
\newcommand{\javo}{$j$\tsup{avo} }
\newcommand{\kava}{$k$\tsup{ava} }
\newcommand{\kavo}{$k$\tsup{avo} }
\newcommand{\tava}{$t$\tsup{ava} }
\newcommand{\tavo}{$t$\tsup{avo} }
\newcommand{\tmava}{$(t-1)$\tsup{ava} }
\newcommand{\tmavo}{$(t-1)$\tsup{avo} }
\newcommand{\tMava}{$(t+1)$\tsup{ava} }
\newcommand{\tMavo}{$(t+1)$\tsup{avo} }

\begin{center}
\Large Modelos Estoc\'asticos (INDG-1008): Trabajo Aut\'onomo \numero \\[1ex]
\small \textbf{Semestre:} 2017-2018 T\'ermino II \qquad
\textbf{Instructor:} Luis I. Reyes Castro
\end{center}
\fullskip

%\fbox{

\begin{minipage}[b][\height][t]{\textwidth}
\vspace{0.2 cm}

\begin{center}
\textbf{COMPROMISO DE HONOR}
\end{center}
\vspace{0.4 cm}

\scriptsize
{
Yo, \rule{60mm}{.1pt} al firmar este compromiso, reconozco que la presente lecci\'on est\'a dise\~nada para ser resuelta de manera individual, que puedo usar un l\'apiz o pluma y una calculadora cient\'ifica, \linebreak que solo puedo comunicarme con la persona responsable de la recepci\'on de la lecci\'on, y que cualquier instrumento de comunicaci\'on que hubiere tra\'ido debo apagarlo. Tambi\'en estoy conciente que no debo consultar libros, notas, \linebreak ni materiales did\'acticos adicionales a los que el instructor entregue durante la lecci\'on o autorice a utilizar. Finalmente, me comprometo a desarrollar y presentar mis respuestas de manera clara y ordenada. \\

Firmo al pie del presente compromiso como constancia de haberlo le\'ido y aceptado. 
\vspace{0.4 cm}

Firma: \rule{60mm}{.1pt} \qquad N\'umero de matr\'icula: \rule{40mm}{.1pt} \hspace{0.5cm} \\[-0.8ex]
}

\end{minipage}

}

\vspace{\baselineskip}


\halfskip

% -----------------------------------------------------------------
\begin{problem}
Un fabricante tiene una m\'aquina complicada. Al comienzo de cada d\'ia si la m\'aquina est\'a operativa entonces con probabilidad $p$ la misma se descompondr\'a durante el d\'ia y los t\'ecinos ser\'an llamados para arreglarla al d\'ia siguiente. Suponiendo que los t\'ecnicos siempre se toman dos d\'ias para arreglar la m\'aquina, modele esta situaci\'on

En el problema anterior, suponga que el fabricante tiene una m\'aquina de repuesto que utiliza solamente cuando la m\'aquina principal est\'a descompuesta. Similar al caso de la m\'aquina principal, si al comienzo de cualquier d\'ia la m\'aquina de repuesto est\'a operativa entonces con probabilidad $q$ se descompondr\'a durante el d\'ia y los t\'ecnicos ser\'an llamados a repararla al d\'ia siguiente. Nuevamente, los t\'ecnicos se tomar\'an dos d\'ias en arreglar la m\'aquina. Modele esta nueva situaci\'on como una cadena de Markov y construya la matriz de transici\'on en funci\'on de $p$ y $q$. No necesita bosquejar el grafo de la cadena ni calcular probabilidades en estado estable. 

\end{problem}
\vspace{\baselineskip}

% -----------------------------------------------------------------
\begin{problem}

\end{problem}
\vspace{\baselineskip}

% -----------------------------------------------------------------
\begin{problem}

\end{problem}
\vspace{\baselineskip}

% -----------------------------------------------------------------
\begin{problem}

\end{problem}
\vspace{\baselineskip}

% -----------------------------------------------------------------
\begin{problem}

\end{problem}
\vspace{\baselineskip}

% -----------------------------------------------------------------
\begin{problem}

\end{problem}
\vspace{\baselineskip}

% -----------------------------------------------------------------
\begin{problem}

\end{problem}
\vspace{\baselineskip}

% -----------------------------------------------------------------
\begin{problem}

\end{problem}
\vspace{\baselineskip}

\end{document}
