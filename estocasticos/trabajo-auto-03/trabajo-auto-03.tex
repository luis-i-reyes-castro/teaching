% -----------------------------------------------------------------
% Document class: Article
\documentclass[ a4paper, twoside, 11pt]{article}
\usepackage{../../macros-general}
\usepackage{../../macros-article}
\graphicspath{{./figures/}}
% Uncomment the following line if you need to write Python code
% \usepackage{minted}

% Number of the handout, quiz, exam, etc.
\newcommand{\numero}{03}
\setcounter{numero}{\numero}

% -----------------------------------------------------------------
\begin{document}
\allowdisplaybreaks

% Indices
\newcommand{\iava}{$i$\tsup{ava} }
\newcommand{\iavo}{$i$\tsup{avo} }
\newcommand{\java}{$j$\tsup{ava} }
\newcommand{\javo}{$j$\tsup{avo} }
\newcommand{\kava}{$k$\tsup{ava} }
\newcommand{\kavo}{$k$\tsup{avo} }
\newcommand{\tava}{$t$\tsup{ava} }
\newcommand{\tavo}{$t$\tsup{avo} }
\newcommand{\tmava}{$(t-1)$\tsup{ava} }
\newcommand{\tmavo}{$(t-1)$\tsup{avo} }
\newcommand{\tMava}{$(t+1)$\tsup{ava} }
\newcommand{\tMavo}{$(t+1)$\tsup{avo} }

\begin{center}
\Large Modelos Estoc\'asticos (INDG-1008): Trabajo Aut\'onomo \numero \\[1ex]
\small \textbf{Semestre:} 2017-2018 T\'ermino II \qquad
\textbf{Instructor:} Luis I. Reyes Castro
\end{center}
\fullskip

%\fbox{

\begin{minipage}[b][\height][t]{\textwidth}
\vspace{0.2 cm}

\begin{center}
\textbf{COMPROMISO DE HONOR}
\end{center}
\vspace{0.4 cm}

\scriptsize
{
Yo, \rule{60mm}{.1pt} al firmar este compromiso, reconozco que la presente lecci\'on est\'a dise\~nada para ser resuelta de manera individual, que puedo usar un l\'apiz o pluma y una calculadora cient\'ifica, \linebreak que solo puedo comunicarme con la persona responsable de la recepci\'on de la lecci\'on, y que cualquier instrumento de comunicaci\'on que hubiere tra\'ido debo apagarlo. Tambi\'en estoy conciente que no debo consultar libros, notas, \linebreak ni materiales did\'acticos adicionales a los que el instructor entregue durante la lecci\'on o autorice a utilizar. Finalmente, me comprometo a desarrollar y presentar mis respuestas de manera clara y ordenada. \\

Firmo al pie del presente compromiso como constancia de haberlo le\'ido y aceptado. 
\vspace{0.4 cm}

Firma: \rule{60mm}{.1pt} \qquad N\'umero de matr\'icula: \rule{40mm}{.1pt} \hspace{0.5cm} \\[-0.8ex]
}

\end{minipage}

}

\vspace{\baselineskip}


\halfskip

% -----------------------------------------------------------------
\begin{problem}
Considere la variante del modelo M/M/1 mostrada en la siguiente figura, donde la tasa de arribo decae harm\'onicamente con el n\'umero de clientes en el sistema. 

\begin{figure}[htb]
\centering
\def\svgwidth{0.9\columnwidth}
\input{figures/modelo-MM1-harmonico.eps_tex}
\end{figure}
\halfskip

Tomando $\rho = \lambda \, / \, \mu$ como es usual, complete las siguientes actividades: 
\begin{enumerate}[label=\textbf{\alph*)}]
\item \textbf{3 Puntos:} Encuentre la probabilidad estacionaria de los estados uno, dos y tres en funci\'on de $\rho$ y de la probabilidad estacionaria del estado cero, \ie exprese $\pi_1$, $\pi_2$ y $\pi_3$ en funci\'on de $\rho$ y $\pi_0$. 
\item \textbf{2 Puntos:} Demuestre que si existe alg\'un $n \geq 1$ tal que
\[
\forall \, k \in \{ \, 1, \, \dots, \, n \, \} \; \colon \;
\pi_k \; = \; \frac{ \rho^k \, \pi_0 }{k!}
\]
entonces es el caso que: 
\[
\pi_{n+1} \; = \; \frac{ \rho^{n+1} \, \pi_0 }{(n+1)!}
\]
\item \textbf{1 Punto:} Existe alguna condici\'on sobre $\rho$ (aparte de ser positivo) que se debe cumplir para que el sistema tenga una distribuci\'on estacionaria? 
\item \textbf{1 Punto:} Calcule la distribuci\'on estacionaria del sistema, en los casos cuando existe, como funci\'on de $\rho$ y $k$. 
\item \textbf{2 Puntos:} Calcule las m\'etricas de desempe\~no $L$ y $L_q$. 
\item \textbf{2 Puntos:} Calcule la tasa de arribo promedio
\[
\bar{\lambda} \; = \; \sum_{k=0}^{\infty} \lambda_{k} \, \pi_k
\]
junto con las m\'etricas de desempe\~no $W$ y $W_q$. 
\end{enumerate}

\end{problem}
\vspace{\baselineskip}

% -----------------------------------------------------------------
\begin{problem}
\label{prob:MM1-harmonico-prima}
Considere la variante del modelo M/M/1 mostrada en la siguiente figura, donde la tasa de arribo decae harm\'onicamente con el n\'umero de clientes en cola. Tomando $\rho = \lambda \, / \, \mu$ como es usual, complete las siguientes actividades: 

\begin{figure}[htb]
\centering
\def\svgwidth{0.9\columnwidth}
\input{figures/modelo-MM1-harmonico-prima.eps_tex}
\end{figure}
\halfskip

\begin{enumerate}[label=\textbf{\alph*)}]
\item \textbf{3 Puntos:} Calcule la distribuci\'on estacionaria del sistema, en los casos cuando existe, como funci\'on de $\rho$ y $k$. 
\item \textbf{2 Puntos:} Calcule las m\'etricas de desempe\~no $L$ y $L_q$. 
\item \textbf{2 Puntos:} Calcule la tasa de arribo promedio
\[
\bar{\lambda} \; = \; \sum_{k=0}^{\infty} \lambda_{k} \, \pi_k
\]
junto con las m\'etricas de desempe\~no $W$ y $W_q$. 
\end{enumerate}

\end{problem}
\vspace{\baselineskip}

% -----------------------------------------------------------------
\begin{problem}
\textbf{[7 Puntos]} Considere la siguiente variante del modelo M/M/1, donde $p \in (0,1)$ y la tasa de arribo decae geom\'etricamente con el n\'umero de clientes en el sistema. Tomando $\rho = \lambda \, / \, \mu$, complete las mismas tres actividades del Problema \ref{prob:MM1-harmonico-prima}. 

\begin{figure}[htb]
\centering
\def\svgwidth{0.9\columnwidth}
\input{figures/modelo-MM1-geometrico.eps_tex}
\end{figure}
\halfskip

\end{problem}
\vspace{\baselineskip}

% -----------------------------------------------------------------
\begin{problem}
\textbf{[7 Puntos]} Considere la siguiente variante del modelo M/M/1, donde $p \in (0,1)$ y la tasa de arribo decae geom\'etricamente con el n\'umero de clientes en cola. Tomando $\rho = \lambda \, / \, \mu$, complete las mismas tres actividades del Problema \ref{prob:MM1-harmonico-prima}. 

\begin{figure}[htb]
\centering
\def\svgwidth{0.9\columnwidth}
\input{figures/modelo-MM1-geometrico-prima.eps_tex}
\end{figure}
\halfskip

\end{problem}
\vspace{\baselineskip}

% -----------------------------------------------------------------
\begin{problem}
Considere la siguiente variante del modelo M/M/1 para una aplicaci\'on en telecomunicaciones digitales. En este modelo, cada vez que el sistema se vac\'ia, \ie que se queda sin paquetes que transmitir, se activa un reloj exponencial con par\'ametro $\alpha$ por milisegundo. Si no llega un nuevo paquete antes del primer tick del reloj entonces el sistema entra en modo de enfriamiento, el cual tiene una duraci\'on exponencial con media de $\beta$ milisegundos. Cuando el sistema est\'a en modo de enfriamiento todos los paquetes que arriban son rechazados. 

\begin{figure}[htb]
\centering
\def\svgwidth{0.9\columnwidth}
\input{figures/modelo-MM1-enfriamiento.eps_tex}
\end{figure}
\halfskip

Tomando $\rho = \lambda \, / \, \mu$ como es usual, complete las siguientes actividades: 
\begin{enumerate}[label=\textbf{\alph*)}]
\item \textbf{3 Puntos:} Calcule la distribuci\'on estacionaria del sistema, en los casos cuando existe, como funci\'on de $\alpha$, $\beta$, $\rho$, y $k$. 
\item \textbf{2 Puntos:} Calcule las m\'etricas de desempe\~no $L$ y $L_q$, $W$ y $W_q$. 
\item \textbf{1 Punto:} Calcule el n\'umero esperado de paquetes rechazados por milisegundo. 
\end{enumerate}

\end{problem}
\vspace{\baselineskip}

% -----------------------------------------------------------------
\begin{problem}
Considere la siguiente variante del modelo M/M/2. En este modelo un \'unico servidor atiene a los clientes en pares a una tasa de $\mu$ por per\'iodo, pero no puede atender clientes de manera individual, \eg si el sistema est\'a vac\'io y llega un cliente, se debe esperar hasta la llegada del siguiente cliente para empezar el servicio. 

\begin{figure}[htb]
\centering
\def\svgwidth{0.9\columnwidth}
\input{figures/modelo-MM1-tandem.eps_tex}
\end{figure}
\halfskip

Tomando $\rho = \lambda \, / \, \mu$ como es usual, complete las siguientes actividades: 
\begin{enumerate}[label=\textbf{\alph*)}]
\item \textbf{3 Puntos:} Calcule la distribuci\'on estacionaria del sistema, en los casos cuando existe, como funci\'on de $\rho$, y $k$. 
\item \textbf{2 Puntos:} Calcule las m\'etricas de desempe\~no $L$ y $L_q$, $W$ y $W_q$. 
\end{enumerate}

\end{problem}
\vspace{\baselineskip}

\end{document}

% -----------------------------------------------------------------
\begin{problem}

\end{problem}
\vspace{\baselineskip}
