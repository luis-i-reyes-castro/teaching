% -----------------------------------------------------------------
% Document class: Article
\documentclass[ a4paper, twoside, 11pt]{article}
\usepackage{../../macros-general}
\usepackage{../../macros-article}
% Number of the handout, quiz, exam, etc.
\newcommand{\numero}{01}
\setcounter{numero}{\numero}
\graphicspath{{./figures/}}

% -----------------------------------------------------------------
\begin{document}
\allowdisplaybreaks

% Indices
\newcommand{\iava}{$i$\tsup{ava} }
\newcommand{\iavo}{$i$\tsup{avo} }
\newcommand{\java}{$j$\tsup{ava} }
\newcommand{\javo}{$j$\tsup{avo} }
\newcommand{\kava}{$k$\tsup{ava} }
\newcommand{\kavo}{$k$\tsup{avo} }
\newcommand{\tava}{$t$\tsup{ava} }
\newcommand{\tavo}{$t$\tsup{avo} }
\newcommand{\tmava}{$(t-1)$\tsup{ava} }
\newcommand{\tmavo}{$(t-1)$\tsup{avo} }
\newcommand{\tMava}{$(t+1)$\tsup{ava} }
\newcommand{\tMavo}{$(t+1)$\tsup{avo} }

\begin{center}
\Large Modelos Estoc\'asticos (INDG-1008): Tarea \numero \\[1ex]
\small \textbf{Semestre:} 2017-2018 T\'ermino I \qquad
\textbf{Instructor:} Luis I. Reyes Castro
\end{center}
\fullskip

%\fbox{

\begin{minipage}[b][\height][t]{\textwidth}
\vspace{0.2 cm}

\begin{center}
\textbf{COMPROMISO DE HONOR}
\end{center}
\vspace{0.4 cm}

\scriptsize
{
Yo, \rule{60mm}{.1pt} al firmar este compromiso, reconozco que la presente evaluaci\'on est\'a dise\~nada para ser resuelta de manera individual, que puedo usar un l\'apiz o pluma y una calculadora cient\'ifica, \linebreak que solo puedo comunicarme con la persona responsable de la recepci\'on de la evaluaci\'on, y que cualquier instrumento de comunicaci\'on que hubiere tra\'ido debo apagarlo. Tambi\'en estoy conciente que no debo consultar libros, notas, \linebreak ni materiales did\'acticos adicionales a los que el instructor entregue durante la evaluaci\'on o autorice a utilizar. Finalmente, me comprometo a desarrollar y presentar mis respuestas de manera clara y ordenada. \\

Firmo al pie del presente compromiso como constancia de haberlo le\'ido y aceptado. 
\vspace{0.4 cm}

Firma: \rule{60mm}{.1pt} \qquad N\'umero de matr\'icula: \rule{40mm}{.1pt} \hspace{0.5cm} \\[-0.8ex]
}

\end{minipage}

}

\vspace{\baselineskip}



\textbf{Instrucciones:} Por favor entregue todas sus soluciones menos el c\'odigo fuente de manera escrita en un solo archivo en formato PDF de acuerdo al siguiente formato: 
\halfcut
\begin{center}
{\tt apellido1\_apellido2\_tarea\_01.pdf }
\end{center}
Adem\'as, por favor entregue todas las funciones en Python que se le soliciten en un solo archivo de acuerdo al siguiente formato: 
\halfcut
\begin{center}
{\tt apellido1\_apellido2\_tarea\_01.py }
\end{center}
\fullskip
\halfskip

% -----------------------------------------------------------------
\begin{problem}
\textbf{(2 Puntos)} Un fabricante tiene una m\'aquina complicada. Al comienzo de cada d\'ia si la m\'aquina est\'a operativa entonces con probabilidad $p$ la misma se descompondr\'a durante el d\'ia y los t\'ecinos ser\'an llamados para arreglarla al d\'ia siguiente. Suponiendo que los t\'ecnicos siempre se toman dos d\'ias para arreglar la m\'aquina, modele esta situaci\'on como una cadena de Markov con tres estados y construya la matriz de transici\'on de la cadena en funci\'on del par\'ametro $p$ y encuentre el porcentaje de los d\'ias que la m\'aquina empieza el d\'ia estando operativa. 

\end{problem}
\vspace{\baselineskip}

% -----------------------------------------------------------------
\begin{problem}
\textbf{(6 Puntos)} En el problema anterior, suponga que el fabricante tiene una m\'aquina de repuesto que utiliza solamente cuando la m\'aquina principal est\'a descompuesta. Similar al caso de la m\'aquina principal, si al comienzo de cualquier d\'ia la m\'aquina de repuesto est\'a operativa entonces con probabilidad $q$ se descompondr\'a durante el d\'ia y los t\'ecnicos ser\'an llamados a repararla al d\'ia siguiente. Nuevamente, los t\'ecnicos se tomar\'an dos d\'ias en arreglar la m\'aquina. Modele esta nueva situaci\'on como una cadena de Markov y construya la matriz de transici\'on en funci\'on de $p$ y $q$. No necesita bosquejar el grafo de la cadena ni calcular probabilidades en estado estable. 

\end{problem}
\vspace{\baselineskip}

% -----------------------------------------------------------------
\begin{problem}
\textbf{(3 Puntos)} Suponga que usted trabaja en una concesionaria de maquinaria de construcci\'on. Usted ha sido encargado con calcular el costo de la garant\'ia de tres a\~nos para excavadoras que ofrece la empresa. De los archivos del departamento de ventas, usted ha podido observar que: 
\begin{itemize}
\item Solo el 2\% de los clientes piden la garant\'ia en el primer a\~no. 
\item El 5\% de los clientes que no tuvieron problemas con su m\'aquina en el primer a\~no piden la garant\'ia en el segundo a\~no. 
\item El 9\% de los clientes que no tuvieron problemas con su m\'aquina en el primer y segundo a\~no piden la garant\'ia en el tercer a\~no. 
\end{itemize}
Suponiendo que cada excavadora se vende en 200 mil d\'olares, calcule el costo promedio de una garant\'ia de tres a\~nos. 

\end{problem}
\vspace{\baselineskip}

% -----------------------------------------------------------------
\begin{problem}
Suponga que usted ha sido encargado con del manejo del inventario de algunos productos no-perecederos en un supermercado que abre todos los d\'ias a sus clientes por la misma cantidad de tiempo. Para mantener una consistencia en el manejo de inventario a trav\'es de los varios productos que se venden en el supermercado, el gerente ha dispuesto que todos los inventarios se controlen mediante pol\'iticas de punto de reposici\'on. 

M\'as precisamente, al final de cada d\'ia se cuenta el inventario del producto y si este es menor o igual a $I_{rep}$ unidades se hace un pedido de reposici\'on por $Q_{rep}$ unidades al proveedor, el cual es entregado por el mismo al comienzo del siguiente d\'ia antes de la hora de apertura de la tienda; caso contrario, no se hace un pedido de reposici\'on. F\'ijese que bajo estas suposiciones el m\'aximo inventario posible es:
\[
I_{max} \, = \, I_{rep} + Q_{rep}
\]
Adicionalmente,como es de costumbre en este campo de estudio, usted hace la suposici\'on simplificatoria que las demandas del producto $D_1, \, D_2, \, \dots$ constituyen una secuencia de variables aleatorias i.i.d. con distribuci\'on Poisson con par\'ametro $\lambda$. M\'as a\'un, si para cada d\'ia $t$ denotamos a $I_t$ como el inventario a la hora de apertura de la tienda del \tavo d\'ia, a $D_t$ como la demanda del d\'ia, \ie la cantidad que se vender\'ia ese d\'ia si el inventario fuere inagotable, a $R_t$ como la cantidad repuesta entre la noche del \tavo d\'ia y la hora de apertura del \tMavo d\'ia, tenemos que:
\[
I_{t+1} \, = \, \max \{ \, 0, \, I_t - D_t \, \} + R_t \, ; \qquad
R_t \, = \, 
\begin{cases}
Q_{rep}, & \text{si } 0 \leq I_t \leq I_{rep} \, ; \\
0, & \text{caso contrario;}
\end{cases}
\]

Con todo esto en mente, para cada problema de manejo de inventario con demanda $\lambda$ y pol\'itica de punto de reposici\'on $(I_{rep},Q_{rep})$ podemos construir una Cadena de Markov en $n = I_{max} + 1$ estados. M\'as precisamente, los estados ser\'ian: Cero unidades en inventario, una unidad en inventario, dos unidades en inventario, y asi sucesivamente, hasta $I_{max}$ unidades en inventario. 

\textbf{Primera Actividad: (6 Puntos)}

Escriba una funci\'on en Python que tome como entrada (1) el par\'ametro $\lambda$ de la distribuci\'on Poisson de la demanda, (2) el inventario de reposici\'on $I_{rep}$ y (3) la cantidad por reposici\'on $Q_{rep}$, \linebreak y que produzca como \'unica salida la matriz de transici\'on de la cadena de Markov correspondiente como un arreglo {\tt numpy} de taman\~no $(n,n)$. Por favor defina el nombre de su funci\'on como {\tt construye\_mat\_inventario}. Por ejemplo, yo usar\'ia mi funci\'on asi:
\halfskip

\begin{center}
\begin{minipage}{0.9\textwidth}
\begin{minted}[mathescape,
               linenos,
               numbersep=5pt,
               gobble=0,
               frame=lines,
               framesep=2mm]{Python}
import numpy as np
from reyes_castro_tarea_01 import contruye_mat_inventario
# Entradas:
lambda_dem = 6.9 # Parametro de la demanda
I_rep = 5 # Inventario de reposicion
Q_rep = 5 # Cantidad por reposicion
# Salida: Matriz de probabilidad de la Cadena de Markov
P = contruye_mat_inventario( lambda_dem, I_rep, Q_rep)
\end{minted}
\end{minipage}
\end{center}
\fullskip

\emph{Sugerencia:} Para hacerse la vida m\'as f\'acil a la hora de calcular probabilidades le recomiendo que utilice la funcion {\tt poisson.pmf()} de la librer\'ia {\tt scipy.stats}. \href{https://docs.scipy.org/doc/scipy/reference/generated/scipy.stats.poisson.html}{Click aqu\'i} para m\'as informaci\'on. Por ejemplo: 
\halfskip

\begin{center}
\begin{minipage}{0.9\textwidth}
\begin{minted}[mathescape,
               linenos,
               numbersep=5pt,
               gobble=0,
               frame=lines,
               framesep=2mm]{Python}
from scipy.stats import poisson
# Le asignamos a `prob' la probabilidad de que una 
# variable aleatoria Poisson con parametro mu = 4.8 
# tome el valor k = 3
prob = poisson.pmf( mu = 4.8, k = 3)
\end{minted}
\end{minipage}
\end{center}
\fullskip

\textbf{Segunda Actividad: (2 Puntos)}

Construya la matriz de transici\'on para el problema para cada uno de los siguientes casos: 
\begin{table}[htb]
\centering
\begin{tabular}{|c|c|c|c|}
\hline
\textbf{Caso}           & $\bm{\lambda}$ & $\bm{I_{rep}}$ & $\bm{Q_{rep}}$ \\ \hline
Producto A - Pol\'itica 1 & 1.6                & 3                  & 2                  \\ \hline
Producto A - Pol\'itica 2 & 1.6                & 1                  & 4                  \\ \hline
Producto B - Pol\'itica 1 & 4.1                & 2                  & 5                  \\ \hline
Producto B - Pol\'itica 2 & 4.1                & 4                  & 3                  \\ \hline
\end{tabular}
\end{table}

Por favor presente sus matrices como tablas con los decimales redondeados a tres d\'igitos.

\emph{Nota:} Usted puede llevar a cabo esta actividad a\'un sin haber realizado la primera. 

\textbf{Tercera Actividad: (4 Puntos)}

Calcule la distribuci\'on en estado estable para cada uno de los cuatro casos de la actividad anterior. Por favor presente sus resultados en dos tablas separados, una para cada producto, de tal manera que se pueda apreciar la influencia de la pol\'itica de manejo de inventario sobre la distribuci\'on estacionaria de la cadena. 

\textbf{Cuarta Actividad: (4 Puntos)}

Para comparar las pol\'iticas necesitamos una funci\'on de utilidad. En nuestro caso, la funci\'on de utilidad ser\'a la diferencia entre la utilidad neta por ventas y el costo de retenci\'on de la mercader\'ia. Si para cualquier d\'ia definimos a la variable aleatoria $D$ como la demanda de ese d\'ia, \linebreak a la variable aleatoria $V$ como el n\'umero de unidades que se vender\'an ese d\'ia, a $u$ como la utilidad neta por unidad del producto vendido, y a $c$ como el costo de retenci\'on por unidad del producto por d\'ia, entonces la utilidad diaria de un nivel de inventario $I \in \{ \, 0, \, 1, \, \dots, \, I_{max} \, \}$ se calcula de la siguiente manera: 
\[
\text{Utilidad}(I) \, = \, 
u \, \Exp[ V ] - c \, I \, ; \qquad
V \, = \, \min \, \{ \, D, \, I \, \}
\]
Con esto en mente, es evidente que podemos evaluar el desempe\~no de una pol\'itica ponderando las utilidades de los niveles de inventario por la distribuci\'on en estado estable de la cadena de Markov correspondiente:
\[
\text{Utilidad} \, (\text{Pol\'itica}) \; = \; 
\sum_{ I = 0 }^{I_{max}} \pi^*(I) \, \text{Utilidad}(I)
\]

Finalmente, suponiendo que el Producto A tiene una utilidad neta por unidad de $u =$ \$1.25 y un costo de retenci\'on por unidad por d\'ia de $c =$ \$0.10, calcule la utilidad de cada una de las dos pol\'iticas propuestas e indique cu\'al de las dos es mejor. Luego repita el ejercicio para el Producto B suponiendo que $u =$ \$0.60 y que $c =$ \$0.08. 

\end{problem}
\vspace{\baselineskip}

% -----------------------------------------------------------------
\begin{problem}
Considere un conjunto de instalaciones industriales vigiladas por robots como se muestra en la figura de abajo. En cada instalaci\'on, el robot empieza en un \'area aleatoria y hace transiciones a otras \'areas adjacentes (o a la misma \'area) con probabilidad uniforme. 

\begin{figure}[htb]
\centering
\def\svgwidth{0.9\columnwidth}
\input{figures/prob_robot-vigilante.eps_tex}
\end{figure}

\textbf{Primera Actividad: (4 Puntos)}

Escriba una funci\'on en Python que tome como entrada (1) el n\'umero $k$ de filas del arreglo que representa a la instalaci\'on, (2) el n\'umero $\ell$ de columnas del mismo arreglo y (3) una lista de 2-tuplas que especifica (en cualquier orden) los \'indices de los pares de \'areas que definen las paredes internas de la instalaci\'on, y que produzca como \'unica salida la matriz de transici\'on de la cadena de Markov correspondiente como un arreglo {\tt numpy} de taman\~no $(n,n)$, donde $n = k \, \ell$. Por favor defina el nombre de su funci\'on como {\tt construye\_mat\_robot}. Por ejemplo, para la geometr\'ia de la Instalaci\'on $A$, yo usar\'ia mi funci\'on asi:
\halfskip

\begin{center}
\begin{minipage}{0.9\textwidth}
\begin{minted}[mathescape,
               linenos,
               numbersep=5pt,
               gobble=0,
               frame=lines,
               framesep=2mm]{Python}
import numpy as np
from reyes_castro_tarea_01 import contruye_mat_robots
# Entradas:
k = 4 # Numero de filas
l = 5 # Numero de columnas
paredes = [ (14,24), (15,25), (21,31), (22,32), (22,23), \
(31,32), (33,34), (32,42), (34,44) ] # Lista de paredes
# Salida: Matriz de probabilidad de la Cadena de Markov
P = contruye_mat_robots( k, l, paredes)
\end{minted}
\end{minipage}
\end{center}
\fullskip

\textbf{Segunda Actividad: (2 Puntos)}

Para las dos instalaciones mostradas en la figura, compute la frecuencia de visitas a las \'areas de la instalaci\'on y pres\'entelas como un mapa de calor. Adicionalmente, indique cu\'ales son las \'areas m\'as seguras y las \'areas m\'as vulnerables. 

\textbf{Tercera Actividad: (2 Puntos)}

Empezando con la pol\'itica aleatoria uniforme, modifique las probabilidades de transici\'on del robot del tal manera que su distribuci\'on en estado estable sea `tan cerca' de la distribuci\'on uniforme como le sea posible. Justifique su razonamiento explicando sus desiciones de dise\~no, \eg indicando cu\'ales probabilidades de transici\'on fueron incrementadas o disminuidas. 

\emph{Nota:} Este problema no tiene una respuesta `correcta'. Lo que busca es que ustedes practiquen con un problema de dise\~no minimamente realista. 

\end{problem}
\vspace{\baselineskip}

\end{document}
