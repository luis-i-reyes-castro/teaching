% -----------------------------------------------------------------
% Document class: Article
\documentclass[ a4paper, twoside, 11pt]{article}
\usepackage{../../macros-general}
\usepackage{../../macros-article}
% Number of the handout, quiz, exam, etc.
\newcommand{\numero}{01}
\setcounter{numero}{\numero}

% -----------------------------------------------------------------
\begin{document}
\allowdisplaybreaks

\begin{center}
\Large Modelos Estoc\'asticos (INDG-1008): Lecci\'on \numero \\[1ex]
\small \textbf{Semestre:} 2017-2018 T\'ermino I \qquad
\textbf{Instructor:} Luis I. Reyes Castro
\end{center}
\halfskip

\fbox{

\begin{minipage}[b][\height][t]{\textwidth}
\vspace{0.2 cm}

\begin{center}
\textbf{COMPROMISO DE HONOR}
\end{center}
\vspace{0.4 cm}

\scriptsize
{
Yo, \rule{60mm}{.1pt} al firmar este compromiso, reconozco que la presente evaluaci\'on est\'a dise\~nada para ser resuelta de manera individual, que puedo usar un l\'apiz o pluma y una calculadora cient\'ifica, \linebreak que solo puedo comunicarme con la persona responsable de la recepci\'on de la evaluaci\'on, y que cualquier instrumento de comunicaci\'on que hubiere tra\'ido debo apagarlo. Tambi\'en estoy conciente que no debo consultar libros, notas, \linebreak ni materiales did\'acticos adicionales a los que el instructor entregue durante la evaluaci\'on o autorice a utilizar. Finalmente, me comprometo a desarrollar y presentar mis respuestas de manera clara y ordenada. \\

Firmo al pie del presente compromiso como constancia de haberlo le\'ido y aceptado. 
\vspace{0.4 cm}

Firma: \rule{60mm}{.1pt} \qquad N\'umero de matr\'icula: \rule{40mm}{.1pt} \hspace{0.5cm} \\[-0.8ex]
}

\end{minipage}

}

\vspace{\baselineskip}



% -----------------------------------------------------------------
\begin{problem}
\textbf{[5 Puntos]} Los sistemas de telecomunicaciones digitales siempre involucran aleatoriedad, puesto que las transmisiones inal\'ambricas son altamente ruidosas, especialmente sobre largas distancias. Considere un transmisor sobre un canal ruidoso donde, en cada ciclo, \linebreak si hay al menos un paquete de datos en cola entonces el paquete es transmitido exitosamente con probabilidad $p$; caso contrario, el transmisor intentar\'a reenviar el paquete en el siguiente ciclo. Suponga adem\'as que la cola del transmisor tiene tama\~no $B$, lo que significa que si durante un ciclo hay $B$ paquetes en cola entonces cualquier otro paquete que llegue a la cola durante ese ciclo ser\'a rechazado, y por ende perdido para siempre. Finalmente, suponga que los n\'umeros de paquetes que arriban a la cola del transmisor en cada ciclo son variables aleatoria i.i.d. con distribuci\'on Poisson con par\'ametro $\lambda$. 

Con esto en mente, modele la situaci\'on antes mencionada como una Cadena de Markov para el caso cuando $B = 4$. En particular, provea el grafo de la cadena y la matriz de transici\'on. 

\emph{Soluci\'on:} Consideramos cada posible estado: 
\begin{itemize}
% ---------------------------------------------------------
\item Si $X_t = 0$ entonces: 
\begin{itemize}
\item Si $X_{t+1} = k$ para cualquier $k \in \{ 0, 1, 2, 3 \}$ es porque hubieron $k$ arribos, \iec
\[
\forall k \in \{ 0, 1, 2, 3 \} \; \colon \;
\vec{P}_{0k} \; = \;
\Pr \, ( \, \Poisson(\lambda) = k \, ) \; = \; 
\frac{ \lambda^k \, e^{-\lambda} }{k!}
\]
\item Si $X_{t+1} = 4$ es porque hubieron cuatro o m\'as arribos, \iec
\[
\vec{P}_{04} \; = \;
\Pr \, ( \, \Poisson(\lambda) \geq 4 \, ) \; = \; 
\sum_{k=4}^{\infty} \frac{ \lambda^k \, e^{-\lambda} }{k!}
\]
\end{itemize}
% ---------------------------------------------------------
\item Si $X_t = 1$ entonces: 
\begin{itemize}
\item Si $X_{t+1} = 0$ es porque hubo un env\'io exitoso y ningun arribo, \iec 
\[
\vec{P}_{10} \; = \;
p \cdot \Pr \, ( \, \Poisson(\lambda) = 0 \, ) \; = \;
p \, e^{-\lambda}
\]
\item Si $X_{t+1} = 1$ es porque hubo un env\'io exitoso y un arribo, o porque el env\'io fracas\'o y no hubo un arribo, \iec 
\begin{align*}
\vec{P}_{11} \; 
& = \; p \cdot \Pr \, ( \, \Poisson(\lambda) = 1 \, ) +
(1-p) \cdot \Pr \, ( \, \Poisson(\lambda) = 0 \, ) \\
& = \; p \, \lambda \, e^{-\lambda} + (1-p) \, e^{-\lambda}
\end{align*}
\item Si $X_{t+1} = 2$ es porque hubo un env\'io exitoso y dos arribos, o porque el env\'io fracas\'o y hubo un arribo, \iec 
\begin{align*}
\vec{P}_{12} \; 
& = \; p \cdot \Pr \, ( \, \Poisson(\lambda) = 2 \, ) +
(1-p) \cdot \Pr \, ( \, \Poisson(\lambda) = 1 \, ) \\
& = \; p \, \frac{\lambda^2 \, e^{-\lambda}}{2!} 
+ (1-p) \, \lambda \, e^{-\lambda}
\end{align*}
\item Si $X_{t+1} = 3$ es porque hubo un env\'io exitoso y tres arribos, o porque el env\'io fracas\'o y hubo dos arribos, \iec 
\begin{align*}
\vec{P}_{13} \; 
& = \; p \cdot \Pr \, ( \, \Poisson(\lambda) = 3 \, ) +
(1-p) \cdot \Pr \, ( \, \Poisson(\lambda) = 2 \, ) \\
& = \; p \, \frac{\lambda^3 \, e^{-\lambda}}{3!} 
+ (1-p) \, \frac{\lambda^2 \, e^{-\lambda}}{2!} 
\end{align*}
\item Si $X_{t+1} = 4$ es porque hubo un env\'io exitoso y cuatro o m\'as arribos, o porque el env\'io fracas\'o y hubo tres o m\'as arribos, \iec 
\begin{align*}
\vec{P}_{14} \; 
& = \; p \cdot \Pr \, ( \, \Poisson(\lambda) \geq 4 \, ) +
(1-p) \cdot \Pr \, ( \, \Poisson(\lambda) \geq 3 \, ) \\
& = \; 
p \, \sum_{k=4}^{\infty} \frac{\lambda^k \, e^{-\lambda}}{k!} 
+ (1-p) \, \sum_{k=3}^{\infty} \frac{\lambda^k \, e^{-\lambda}}{k!} 
\end{align*}
\end{itemize}
% ---------------------------------------------------------
\item Si $X_t = 2$ entonces: 
\begin{itemize}
\item Si $X_{t+1} = 1$ es porque hubo un env\'io exitoso y ningun arribo, \iec 
\[
\vec{P}_{21} \; = \;
p \cdot \Pr \, ( \, \Poisson(\lambda) = 0 \, ) \; = \;
p \, e^{-\lambda}
\]
\item Si $X_{t+1} = 2$ es porque hubo un env\'io exitoso y un arribo, o porque el env\'io fracas\'o y no hubo un arribo, \iec 
\begin{align*}
\vec{P}_{22} \; 
& = \; p \cdot \Pr \, ( \, \Poisson(\lambda) = 1 \, ) +
(1-p) \cdot \Pr \, ( \, \Poisson(\lambda) = 0 \, ) \\
& = \; p \, \lambda \, e^{-\lambda} + (1-p) \, e^{-\lambda}
\end{align*}
\item Si $X_{t+1} = 3$ es porque hubo un env\'io exitoso y dos arribos, o porque el env\'io fracas\'o y hubo un arribo, \iec 
\begin{align*}
\vec{P}_{23} \; 
& = \; p \cdot \Pr \, ( \, \Poisson(\lambda) = 2 \, ) +
(1-p) \cdot \Pr \, ( \, \Poisson(\lambda) = 1 \, ) \\
& = \; p \, \frac{\lambda^2 \, e^{-\lambda}}{2!} 
+ (1-p) \, \lambda \, e^{-\lambda}
\end{align*}
\item Si $X_{t+1} = 4$ es porque hubo un env\'io exitoso y tres o m\'as arribos, o porque el env\'io fracas\'o y hubo dos o m\'as arribos, \iec 
\begin{align*}
\vec{P}_{24} \; 
& = \; p \cdot \Pr \, ( \, \Poisson(\lambda) \geq 3 \, ) +
(1-p) \cdot \Pr \, ( \, \Poisson(\lambda) \geq 2 \, ) \\
& = \;
p \, \sum_{k=3}^{\infty} \frac{\lambda^k \, e^{-\lambda}}{k!} 
+ (1-p) \, \sum_{k=2}^{\infty} \frac{\lambda^k \, e^{-\lambda}}{k!} 
\end{align*}
\end{itemize}
% ---------------------------------------------------------
\item Si $X_t = 3$ entonces: 
\begin{itemize}
\item Si $X_{t+1} = 2$ es porque hubo un env\'io exitoso y ningun arribo, \iec 
\[
\vec{P}_{32} \; = \;
p \cdot \Pr \, ( \, \Poisson(\lambda) = 0 \, ) \; = \;
p \, e^{-\lambda}
\]
\item Si $X_{t+1} = 3$ es porque hubo un env\'io exitoso y un arribo, o porque el env\'io fracas\'o y no hubo un arribo, \iec 
\begin{align*}
\vec{P}_{33} \; 
& = \; p \cdot \Pr \, ( \, \Poisson(\lambda) = 1 \, ) +
(1-p) \cdot \Pr \, ( \, \Poisson(\lambda) = 0 \, ) \\
& = \; p \, \lambda \, e^{-\lambda} + (1-p) \, e^{-\lambda}
\end{align*}
\item Si $X_{t+1} = 4$ es porque hubo un env\'io exitoso y dos o m\'as arribos, o porque el env\'io fracas\'o y hubo uno o m\'as arribos, \iec 
\begin{align*}
\vec{P}_{34} \; 
& = \; p \cdot \Pr \, ( \, \Poisson(\lambda) \geq 2 \, ) +
(1-p) \cdot \Pr \, ( \, \Poisson(\lambda) \geq 1 \, ) \\
& = \; 
p \, \sum_{k=2}^{\infty} \frac{\lambda^k \, e^{-\lambda}}{k!} 
+ (1-p) \, ( \, 1 - e^{-\lambda} \, )
\end{align*}
\end{itemize}
% ---------------------------------------------------------
\item Si $X_t = 4$ entonces: 
\begin{itemize}
\item Si $X_{t+1} = 3$ es porque hubo un env\'io exitoso y ningun arribo, \iec 
\[
\vec{P}_{43} \; = \;
p \cdot \Pr \, ( \, \Poisson(\lambda) = 0 \, ) \; = \;
p \, e^{-\lambda}
\]
\item Caso contrario $X_{t+1} = 4$, \iec 
\[
\vec{P}_{44} \; = \; 1 - p \, e^{-\lambda}
\]
\end{itemize}

\end{itemize}

\end{problem}
\vspace{\baselineskip}

% -----------------------------------------------------------------
\begin{problem}
Usualmente las Cadenas de Markov no son modelos suficientemente complejos como para representar el comportamiento humano, puesto que nuestros cerebros no tienen la Propiedad Markoviana, \ie tenemos memoria. A\'un asi, podemos usar cadenas de Markov para modelar a personas err\'aticas, impredecibles y posiblemente irracionales. Por ejemplo, considere el siguiente modelo de la presidencia de Donald Trump, donde los estados son: 
\begin{enumerate}
\item Hablar de como gan\'o las elecciones obteniendo la mayor\'ia de los colegios electorales, \linebreak a pesar de haber perdido el voto popular por m\'as de dos millones de votos. 
\item Defenderse del esc\'andalo de Russia. 
\item Insistir en construir una pared en la frontera sur con M\'exico. 
\item Hablar de los acuerdos de comerciales que dice que renegociar\'a (\eg NAFTA) y que dice que ser\'an ventajosos para su pa\'is. 
\item Pelearse con celebridades en Twitter. 
\end{enumerate}
La matriz de transici\'on para este modelo es: 
\[
\vec{P} \; = \; 
\left[
\begin{array}{ccccc}
0.5 & 0 & 0.25 & 0.25 & 0 \\
0 & 0.6 & 0 & 0 & 0.4 \\
0 & 1 & 0 & 0 & 0 \\
0.9 & 0.1 & 0 & 0 & 0 \\
0 & 0.3 & 0.3 & 0 & 0.4
\end{array}
\right]
\]
Con esto en mente: 
\begin{itemize}
\item \textbf{[1 Punto]} Provea un an\'alisis de estados. En particular, indique: 
\begin{itemize}
\item Los estados transitorios
\item Los estados recurrentes
\item El numero de clases recurrentes
\item La periodicidad de la cadena
\item Si es o no es una cadena irreducible (\ie una uni-cadena)
\end{itemize}
\item \textbf{[2 Puntos]} Escriba un conjunto de ecuaciones lineales linealmente independientes de cuya soluci\'on se pueda obtener la distribuci\'on en estado estable de la cadena. 
\item \textbf{[2 Puntos]} Calcule el porcentaje del tiempo que Donald Trump insistir\'a en construir la pared con M\'exico. 
\end{itemize}

\emph{Soluci\'on:}
\begin{itemize}
\item El an\'alisis de estados es como sigue: 
\begin{itemize}
\item Los estados transitorios son el 1 (Elecciones) y el 4 (NAFTA). 
\item Los estados recurrentes son el 2 (Rusia), el 3 (Pared) y el 5 (Twitter). 
\item Existe una \'unica clase recurrente compuesta por tres estados recurrentes. 
\item La cadena es aperi\'odica. 
\item La cadena no es irreducible puesto que tiene estados transitorios. 
\end{itemize}
\item Las ecuaciones de estado estacionario son: 
\begin{align*}
\pi_1 \; & \; = \; 
0.5 \, \pi_1 + 0.9 \, \pi_4 \\
\pi_2 \; & \; = \; 
0.6 \, \pi_2 + \pi_3 + 0.1 \, \pi_4 + 0.3 \, \pi_5 \\
\pi_3 \; & \; = \; 0.25 \, \pi_1 + 0.3 \, \pi_5 \\
\pi_4 \; & \; = \; 0.25 \, \pi_1 \\
\pi_5 \; & \; = \; 0.4 \, \pi_2 + 0.4 \, \pi_5
\end{align*}
Recordando que los estados 1 y 4 son transitorios, tenemos $\pi_1 = \pi_4 = 0$. A su vez esto implica que las ecuaciones asociadas con esos estados son triviales, por lo que nuestro sistema de ecuaciones se reduce a: 
\begin{align*}
\pi_2 \; & \; = \; 
0.6 \, \pi_2 + \pi_3 + 0.3 \, \pi_5 \\
\pi_3 \; & \; = \; 0.3 \, \pi_5 \\
\pi_5 \; & \; = \; 0.4 \, \pi_2 + 0.4 \, \pi_5
\end{align*}
Finalmente, reemplazando cualquiera de las tres ecuaciones anteriores, \eg digamos que la \'ultima, por la condici\'on de que la suma de las probabilidades debe ser uno, \linebreak llegamos al siguiente sistema: 
\begin{align*}
\pi_2 \; & \; = \; 
0.6 \, \pi_2 + \pi_3 + 0.3 \, \pi_5 \\
\pi_3 \; & \; = \; 0.3 \, \pi_5 \\
\pi_2 + \pi_3 + \pi_5 \; & \; = \; 1
\end{align*}
\item Las probabilidades en estado estable son: 
\[
\pi_2 \; \simeq \; 0.536 \qquad \qquad
\pi_3 \; \simeq \; 0.107 \qquad \qquad
\pi_5 \; \simeq \; 0.537
\]
Consecuentemente, de acuerdo a este modelo el porcentaje del tiempo que Donald Trump insistir\'a en construir la pared es del 10.7\%. 


\end{itemize}

\end{problem}
\vspace{\baselineskip}

\end{document}
