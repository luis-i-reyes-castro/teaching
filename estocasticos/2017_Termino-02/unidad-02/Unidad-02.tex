% =================================================================
\documentclass[ 10pt, xcolor = dvipsnames]{beamer}
\usepackage{ beamerthemesplit, lmodern}
\usetheme{Madrid}
\usecolortheme[named=Brown]{structure}
\useinnertheme{rectangles}
\setbeamertemplate{frametitle continuation}{}
\beamertemplatenavigationsymbolsempty
\usepackage{../../macros-general}
\usepackage{../../macros-beamer}
\graphicspath{{./figures/}}

% =================================================================
\newcommand{\shorttitle}{Modelos Estoc\'asticos - Unidad 02}
\title[\shorttitle]{Modelos Estoc\'asticos para Manufactura y Servicios (INDG-1008): \textbf{Unidad 02} }
\author[L. I. Reyes Castro]{Luis I. Reyes Castro}
\institute[ESPOL]{\normalsize Escuela Superior Polit\'ecnica del Litoral (ESPOL) \\ Guayaquil - Ecuador}
\date[2017-T1]{2017 - Primer T\'ermino}

% -----------------------------------------------------------------
\begin{document}
\begin{frame}[noframenumbering]
\titlepage
\end{frame}
\begin{frame}[noframenumbering]
\frametitle{\shorttitle}
\tableofcontents[ subsectionstyle = hide]
\end{frame}

\AtBeginSection[]
{
\begin{frame}
\frametitle{Contenido del Tema}
\tableofcontents[ currentsection, sectionstyle = show/shaded, subsectionstyle = show/show/hide]
\end{frame}
}
\AtBeginSubsection[]
{
\begin{frame}
\frametitle{Contenido del Tema}
\tableofcontents[ currentsection, currentsubsection, sectionstyle = show/shaded, subsectionstyle = show/shaded/hide]
\end{frame}
}

% Indices
\newcommand{\iava}{$i$\tsup{ava} }
\newcommand{\iavo}{$i$\tsup{avo} }
\newcommand{\java}{$j$\tsup{ava} }
\newcommand{\javo}{$j$\tsup{avo} }
\newcommand{\kava}{$k$\tsup{ava} }
\newcommand{\kavo}{$k$\tsup{avo} }
\newcommand{\tava}{$t$\tsup{ava} }
\newcommand{\tavo}{$t$\tsup{avo} }
\newcommand{\tmava}{$(t-1)$\tsup{ava} }
\newcommand{\tmavo}{$(t-1)$\tsup{avo} }
\newcommand{\tMava}{$(t+1)$\tsup{ava} }
\newcommand{\tMavo}{$(t+1)$\tsup{avo} }

% =================================================================
\section{Nociones B\'asicas de Sistemas de Colas}

% -----------------------------------------------------------------
\begin{frame}[allowframebreaks]
\frametitle{\insertsection}

\textbf{Actores Principales:}
\begin{itemize}
\item \underline{Clientes:} Aquellos que desean el servicio que ofrece el sistema. En nuestros modelos sus actividades son: 
\begin{itemize}
\item Arribar al sistema, y esperar en cola de ser necesario. 
\item Ser servido, y abandonar el sistema una vez servido. 
\end{itemize}
\item \underline{Servidores:} Aquellos que proveen a los clientes el servicio que ofrece el sistema. En nuestros modelos sus actividades son: 
\begin{itemize}
\item Esperar el arribo de clientes para servirlos, \ie no hacer nada. 
\item Servir clientes. 
\end{itemize}
\end{itemize}
\framebreak

\textbf{Disciplina de Servicio:}
\begin{itemize}
\item Solamente consideraremos sistemas de colas FIFO \emph{(First-In, First-Out)}, \ie sistemas donde todos los clientes son considerados id\'enticos y se los atiende en order de llegada. 
\item Si un cliente arriba al sistema y hay al menos un servidor esperando clientes, entonces el nuevo cliente no necesitar\'a esperar en cola y empezar\'a a ser servido inmediatamente. 
\item Si un cliente arriba al sistema y todos los servidores est\'an ocupados con otros clientes que arribaron anteriormente, entonces el nuevo cliente necesitar\'a tomar el \'ultimo puesto en la cola y esperar para poder ser servido. 
\end{itemize}
\framebreak

\textbf{M\'etricas de Desempe\~no:}
\begin{itemize}
\item $\boldsymbol{L \, \colon}$ N\'umero esperado de clientes en el sistema. 
\begin{itemize}
\item Incluye a los clientes en servicio y en cola. 
\end{itemize}
\item $\boldsymbol{L_q \, \colon}$ N\'umero esperado de clientes en cola. 
\begin{itemize}
\item No incluye a los clientes en servicio. 
\end{itemize}
\item $\boldsymbol{W \, \colon}$ Tiempo esperado (promedio) de espera en el sistema por cliente. 
\begin{itemize}
\item Incluye el tiempo de espera en cola y el tiempo de servicio. 
\end{itemize}
\item $\boldsymbol{W_q \, \colon}$ Tiempo esperado (promedio) de espera en el cola por cliente. 
\begin{itemize}
\item No incluye el tiempo de servicio. 
\end{itemize}
\end{itemize}

\end{frame}

% -----------------------------------------------------------------
\begin{frame}[allowframebreaks]
\frametitle{\insertsection}

\textbf{Teorema - Ley de Little :}

Para todo sistema de colas que recibe clientes a una tasa de $\lambda$ por per\'iodo, \linebreak es el caso que: 
\[
L \; = \; \lambda \, W \qquad \qquad \qquad
L_q \; = \; \lambda \, W_q
\]

N\'otese que este teorema no requiere que el proceso de arribo sea Poisson, \ie \linebreak no requiere que los tiempos entre arribos est\'en exponencialmente distribu\'idos. \linebreak De hecho, si suponemos que la variable aleatoria $X$ representa la distribuci\'on del tiempo entre arribos al sistema de colas, tenemos que: 
\[
\lambda \; = \; \frac{1}{\Exp[X]}
\]

%Finalmente, si la tasa de arribo depende del estado del sistema entonces podemos aplicar el teorema promediando las tasas de arribo por la distribuci\'on estacionaria del sistema. 

\end{frame}

% =================================================================
\section{Cadenas de Markov en Tiempo Continuo}

% =================================================================
\subsection{Modelamiento}

% -----------------------------------------------------------------
\begin{frame}[allowframebreaks]
\frametitle{\insertsubsection}

Una \textbf{Cadena de Markov de Tiempo Continuo} es un modelo matem\'atico de un proceso estoc\'astico en tiempo continuo constitu\'ido por: 
\begin{itemize}
\item Conjunto finito de $n$ estados, donde cada estado es una representaci\'on de una posible situaci\'on de inter\'es. 
\item Matriz de tasas de transici\'on $\vec{Q} \in \Re^{n \times n}$, donde: 
\begin{itemize}
\item Para todo estado $i$ suponemos que $\vec{Q}(i,i) = 0$
\item Para todo estado $i$ definimos a la tasa del salida de ese estado como: 
\[
\vec{Q}(i) \; \define \; \sum_{ 1 \leq j \leq n } \vec{Q}(i,j)
\]
\fullcut
\item Para cada par de estados $i,j$: 
\[
\vec{P}(i,j) \; = \; 
\Pr \, ( \, \text{siguiente estado sea } j \mid \text{estado actual es } i \, ) \; = \; 
\frac{ \vec{Q}(i,j) }{ \; \vec{Q}(i) \; }
\]
\framebreak
\item Cada vez que la cadena ingresa a un estado $i$ la duraci\'on del intervalo \linebreak durante el cual la cadena permanece en ese estado, denotado $T_i$, es una variable aleatoria con distribuci\'on exponencial. M\'as precisamente: 
\[
T_i \, \sim \, \Exponential \, ( \, \vec{Q}(i) \, )
\]

\item Observe adem\'as que para cualquier estado $i$ la sumatoria de las tasas de transici\'on $\vec{Q}(i,j)$ es positiva pero generalmente diferente de uno. 
\end{itemize}
\end{itemize}

\end{frame}

% -----------------------------------------------------------------
\begin{frame}[allowframebreaks]
\frametitle{\insertsubsection}

\underline{Ejemplo:}

Un taller tiene dos m\'aquinas id\'enticas en operaci\'on continua excepto cuando se descomponen. Como lo hacen con bastante frecuencia, el taller dispone de un t\'ecnico contratado a tiempo completo para repararlas cuando es necesario. \linebreak El tiempo que se requiere para reparar una m\'aquina tiene distribuci\'on exponencial con media de 0.5 d\'ias. Una vez que se termina la reparaci\'on, el tiempo que transcurre hasta la siguiente descompostura tiene distribuci\'on exponencial con media de un 1.0 d\'ias. Estas distribuciones son independientes. 

Modele la situaci\'on descrita como una Cadena de Markov en Tiempo Continuo. 

\end{frame}

% -----------------------------------------------------------------
\begin{frame}[allowframebreaks]
\frametitle{\insertsubsection}

\underline{Ejemplo: Modelo M/M/1}

Un sistema de colas cuenta con un \'unico servidor cuyo tiempo de servicio est\'a exponencialmente distribu\'ido con una tasa de $\mu$ por hora. Asumiendo que los clientes arriban de acuerdo a un proceso Poisson con tasa de $\lambda$ por hora, modele este sistema de colas como una Cadena de Markov en Tiempo Continuo. 

\fullskip

\underline{Ejemplo: Modelo M/M/2}

Un sistema de colas cuenta con dos servidores id\'enticos cuyos tiempos de servicio est\'an exponencialmente distribu\'idos con una tasa de $\mu$ por hora. Asumiendo nuevamente que los clientes arriban de acuerdo a un proceso Poisson con tasa \linebreak de $\lambda$ por hora, modele este sistema de colas como una Cadena de Markov en \linebreak Tiempo Continuo. 

\framebreak

Cadena de Markov para el Modelo M/M/1: 
\begin{figure}[htb]
\centering
\def\svgwidth{0.75\columnwidth}
\input{figures/modelo-MM1.eps_tex}
\end{figure}

Cadena de Markov para el Modelo M/M/2: 
\begin{figure}[htb]
\centering
\def\svgwidth{0.75\columnwidth}
\input{figures/modelo-MM2.eps_tex}
\end{figure}

\framebreak

\underline{Ejemplo: Modelo M/M/3}

Un sistema de colas cuenta con tres servidores id\'enticos cuyos tiempos de servicio est\'an exponencialmente distribu\'idos con una tasa de $\mu$ por hora. Asumiendo que los clientes arriban de acuerdo a un proceso Poisson con tasa de $\lambda$ por hora, modele este sistema de colas como una Cadena de Markov en \linebreak Tiempo Continuo. 

\fullskip

\underline{Ejemplo: Modelo M/M/4}

Un sistema de colas cuenta con cuatro servidores id\'enticos cuyos tiempos de servicio est\'an exponencialmente distribu\'idos con una tasa de $\mu$ por hora. Asumiendo que los clientes arriban de acuerdo a un proceso Poisson con tasa \linebreak de $\lambda$ por hora, modele este sistema de colas como una Cadena de Markov en Tiempo Continuo. 

\framebreak

Cadena de Markov para el Modelo M/M/3: 
\begin{figure}[htb]
\centering
\def\svgwidth{0.75\columnwidth}
\input{figures/modelo-MM3.eps_tex}
\end{figure}

Cadena de Markov para el Modelo M/M/4: 
\begin{figure}[htb]
\centering
\def\svgwidth{0.75\columnwidth}
\input{figures/modelo-MM4.eps_tex}
\end{figure}

\end{frame}

% =================================================================
\subsection{Distribuci\'on Estacionaria}

% -----------------------------------------------------------------
\begin{frame}[allowframebreaks]
\frametitle{\insertsubsection}

C\'alculo de la Distribuci\'on en Estado Estable: 
\begin{enumerate}
\item Descartamos todos los estados transitorios. 
\item Escribimos una ecuaci\'on de balance para cada uno de los $n$ estados: 
\[
\forall \, \text{estado } x \colon \, \vec{\pi}(x) \, \vec{Q}(x) \; = \; 
\sum_{\text{estados }i} \vec{\pi}(i) \, \vec{Q}(i,x) 
\]
\item Desechamos arbitrariamente una de las $n$ ecuaciones anteriores y la reemplazamos por: 
\[
\sum_{\text{estados }x} \vec{\pi}(x) \; = \; 1
\]
\item Resolvemos el sistema de ecuaciones lineales resultante, el cual tiene \linebreak $n$ inc\'ognitas y $n$ ecuaciones linealmente independientes. 

\end{enumerate}

\end{frame}

% -----------------------------------------------------------------
\begin{frame}[allowframebreaks]
\frametitle{\insertsubsection}

\underline{Ejemplo:}

Un taller tiene dos m\'aquinas id\'enticas en operaci\'on continua excepto cuando se descomponen. Como lo hacen con bastante frecuencia, el taller dispone de un t\'ecnico contratado a tiempo completo para repararlas cuando es necesario. \linebreak El tiempo que se requiere para reparar una m\'aquina tiene distribuci\'on exponencial con media de 0.5 d\'ias. Una vez que se termina la reparaci\'on, el tiempo que transcurre hasta la siguiente descompostura tiene distribuci\'on exponencial con media de un 1.0 d\'ias. Estas distribuciones son independientes. 

Calcule el porcentaje del tiempo que el taller tiene:
\begin{itemize}
\item Las dos m\'aquinas operativas. 
\item Una m\'aquina operativa y una descompuesta. 
\item Las dos m\'aquinas descompuestas. 
\end{itemize}

\end{frame}

% -----------------------------------------------------------------
\begin{frame}[allowframebreaks]
\frametitle{\insertsubsection}

\underline{Ejemplo: Modelo M/M/1}

Un sistema de colas cuenta con un \'unico servidor cuyo tiempo de servicio est\'a exponencialmente distribu\'ido con una tasa de $\mu$ por hora. Los clientes arriban de acuerdo a un proceso Poisson con tasa de $\lambda$ por hora. 

Definiendo a la tasa de utilizaci\'on 
\[
\rho \; \define \; \frac{\lambda}{\mu}
\]
encuentre: 

\begin{itemize}
\item La condici\'on que deben cumplir $\lambda$ y $\mu$, o equivalentemente $\rho$, para que \linebreak exista una distribuci\'on estacionaria del sistema. 
\item La distribuci\'on estacionaria en funci\'on de $\rho$. 
\item El n\'umero esperado de clientes en el sistema y en cola en funci\'on de $\rho$. 
\end{itemize}

\end{frame}

% -----------------------------------------------------------------
\begin{frame}[allowframebreaks]
\frametitle{\insertsubsection}

\underline{Ejemplo: Modelo M/M/2}

Un sistema de colas cuenta con dos servidores id\'enticos cuyos tiempos de servicio est\'an exponencialmente distribu\'idos con una tasa de $\mu$ por hora. Los clientes arriban de acuerdo a un proceso Poisson con tasa de $\lambda$ por hora. 

Definiendo a la tasa de utilizaci\'on 
\[
\rho \; \define \; \frac{\lambda}{2\mu}
\]
encuentre: 

\begin{itemize}
\item La condici\'on que deben cumplir $\lambda$ y $\mu$, o equivalentemente $\rho$, para que \linebreak exista una distribuci\'on estacionaria del sistema. 
\item La distribuci\'on estacionaria en funci\'on de $\rho$. 
\item El n\'umero esperado de clientes en el sistema y en cola en funci\'on de $\rho$. 
\end{itemize}

\end{frame}

\end{document}
