% =================================================================
\documentclass[ 10pt, xcolor = dvipsnames]{beamer}
\usepackage{ beamerthemesplit, lmodern}
\usetheme{Madrid}
\usecolortheme[named=Brown]{structure}
\useinnertheme{rectangles}
\setbeamertemplate{frametitle continuation}{}
\beamertemplatenavigationsymbolsempty
\usepackage{../../macros-general}
\usepackage{../../macros-beamer}
%\graphicspath{{./figures/}}

% =================================================================
\newcommand{\shorttitle}{Modelos Estoc\'asticos - Unidad 03}
\title[\shorttitle]{Modelos Estoc\'asticos para Manufactura y Servicios (INDG-1008): \textbf{Unidad 02} }
\author[L. I. Reyes Castro]{Luis I. Reyes Castro}
\institute[ESPOL]{\normalsize Escuela Superior Polit\'ecnica del Litoral (ESPOL) \\ Guayaquil - Ecuador}
\date[2017-T1]{2017 - Primer T\'ermino}

% -----------------------------------------------------------------
\begin{document}
\begin{frame}[noframenumbering]
\titlepage
\end{frame}
\begin{frame}[noframenumbering]
\frametitle{\shorttitle}
\tableofcontents[ subsectionstyle = hide]
\end{frame}

\AtBeginSection[]
{
\begin{frame}
\frametitle{Contenido del Tema}
\tableofcontents[ currentsection, sectionstyle = show/shaded, subsectionstyle = show/show/hide]
\end{frame}
}
\AtBeginSubsection[]
{
\begin{frame}
\frametitle{Contenido del Tema}
\tableofcontents[ currentsection, currentsubsection, sectionstyle = show/shaded, subsectionstyle = show/shaded/hide]
\end{frame}
}

% Indices
\newcommand{\iava}{$i$\tsup{ava} }
\newcommand{\iavo}{$i$\tsup{avo} }
\newcommand{\java}{$j$\tsup{ava} }
\newcommand{\javo}{$j$\tsup{avo} }
\newcommand{\kava}{$k$\tsup{ava} }
\newcommand{\kavo}{$k$\tsup{avo} }
\newcommand{\tava}{$t$\tsup{ava} }
\newcommand{\tavo}{$t$\tsup{avo} }
\newcommand{\tmava}{$(t-1)$\tsup{ava} }
\newcommand{\tmavo}{$(t-1)$\tsup{avo} }
\newcommand{\tMava}{$(t+1)$\tsup{ava} }
\newcommand{\tMavo}{$(t+1)$\tsup{avo} }

% =================================================================
\section{\'Arboles de Decisi\'on}

% -----------------------------------------------------------------
\begin{frame}[allowframebreaks]
\frametitle{\insertsection}

\end{frame}

% =================================================================
\section{Inferencia Bayesiana (Experimentaci\'on)}

% -----------------------------------------------------------------
\begin{frame}[allowframebreaks]
\frametitle{\insertsection}

Como motivaci\'on, empecemos considerando la \textbf{Paradoja del Test}. 
\begin{itemize}
\item Un raro tipo de c\'ancer ha sido descubierto hace poco. Se estima que el 0.001\% de la poblaci\'on es gen\'eticamente vulnerable a este c\'ancer. 
\item Un test para comprobar la vulnerabilidad gen\'etica a este nuevo tipo de c\'ancer acaba de ser desarrollado. El desempe\~no del test es como sigue: 
\begin{itemize}
\item Si el paciente es gen\'eticamente vulnerable, el test arroja resultado positivo \linebreak el 98\% de las veces. 
\item Si el paciente no es vulnerable, el test arroja resultado negativo el 94\% \linebreak de las veces. 
\end{itemize}
\item Usted se acaba de hacer el test, y el resultado fue positivo. Cu\'al es la probabilidad de que usted realmente sea gen\'eticamente vulnerable? 
\end{itemize}

\end{frame}

% -----------------------------------------------------------------
\begin{frame}[allowframebreaks]
\frametitle{\insertsection}

\textbf{Problema Prototipo:}
\begin{itemize}
\item Dos variables aleatorias: 
\begin{itemize}
\item Una variable aleatoria $X$ que representa la categor\'ia real pero desconocida. Usualmente decimos que $X$ es la variable no observada. 
\item Una variable aleatoria $Y$ que representa el resultado de un test o clasificador que depende probabilisticamente en $X$. Usualmente decimos que $Y$ es la variable observada. 
\end{itemize}
\item Se nos provee la distribuci\'on de la variable no observada $X$. \Iec
\[
\forall \, x \in \support(X) \; \colon \; \Pr( \, X = x \, )
\]
\fullcut
\halfcut
\begin{itemize}
\item Usualmente se la denomina \emph{distribuci\'on a priori}. 
\end{itemize}
\framebreak

\item Tambi\'en se nos provee la distribuci\'on de la variable observada $Y$ condicional en la variable no observada $X$. \Iec
\[
\forall \, x \in \support(X), \;
\forall \, y \in \support(Y) \; \colon \; \Pr( \, Y = y \mid X = x \, )
\]
\fullcut
\halfcut
\begin{itemize}
\item En el contexto de clasificadores, usualmente conocemos a esta distribuci\'on como la \emph{matriz de confusi\'on}. 
\end{itemize}
\halfskip

\item Nuestro objetivo es calcular la distribuci\'on de la variable no observada $X$ condicional en la variable observada $Y$. \Ie 
\[
\forall \, x \in \support(X), \;
\forall \, y \in \support(Y) \; \colon \; \Pr( \, X = x \mid Y = y \, )
\]
\fullcut
\halfcut
\begin{itemize}
\item Usualmente se la denomina \emph{distribuci\'on posterior predictiva}. 
\end{itemize}
\framebreak

\item Para lograr nuestro objetivo generalmente necesitamos calcular con anterioridad la distribuci\'on marginal de la variable observada $Y$. \Ie 
\[
\forall \, y \in \support(Y) \; \colon \; \Pr( Y = y \, )
\]
\fullcut
\halfcut
\begin{itemize}
\item Usualmente se la denomina \emph{distribuci\'on posterior}. 
\end{itemize}
\item Una vez calculada...

\end{itemize}

\end{frame}

% =================================================================
\section{\'Arboles de Decisi\'on con Experimentaci\'on}

% -----------------------------------------------------------------
\begin{frame}[allowframebreaks]
\frametitle{\insertsection}

\end{frame}

\end{document}
