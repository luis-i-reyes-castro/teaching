% -----------------------------------------------------------------
% Document class: Article
\documentclass[ a4paper, twoside, 11pt]{article}
\usepackage{../../macros-general}
\usepackage{../../macros-article}
% Number of the handout, quiz, exam, etc.
\newcommand{\numero}{02}
\setcounter{numero}{\numero}

% -----------------------------------------------------------------
\begin{document}
\allowdisplaybreaks

\begin{center}
\Large Sistemas de Control (EYAG-1005): Taller \numero \\[1ex]
\small \textbf{Semestre:} 2017-2018 T\'ermino I \qquad
\textbf{Instructor:} Luis I. Reyes Castro
\end{center}
\halfskip

Integrantes del Grupo:
\fullskip
\fullskip
\fullskip

\textbf{Nota:} Para los siguientes problemas asuma un lazo de control en cascada con planta $G_P(s)$, compensador $G_C(s)$ y sensor perfecto, tal como se muestra en la figura de abajo. 

\begin{figure}[htb]
\centering
\def\svgwidth{0.8\columnwidth}
\input{fig_Cascade-compensation.eps_tex}
\end{figure}
\fullskip

% -----------------------------------------------------------------
\begin{problem}
\textbf{[3 Puntos]} Para la planta 
\[
G_P(s) \; = \; 
\frac{s+2}{s(s+4)(s+6)(s+10)}
\]
dise\~ne un compensador proporcional, \ie $G_C(s) = K$, tal que en circuito abierto el sistema tenga 10 decibeles de margen de ganancia. 

\end{problem}
\fullskip

% -----------------------------------------------------------------
\begin{problem}
\textbf{[3 Puntos]} Para la planta 
\[
G_P(s) \; = \; 
\frac{1}{s(s+8)(s+15)}
\]
dise\~ne un compensador proporcional, \ie $G_C(s) = K$, tal que en circuito cerrado el sistema tenga 20\% de sobrepaso. 

\end{problem}
\fullskip

% -----------------------------------------------------------------
\begin{problem}
\textbf{[4 Puntos]} Para la planta 
\[
G_P(s) \; = \; 
\frac{s+4}{(s+2)(s+6)(s+8)}
\]
dise\~ne un compensador de atraso de fase, \ie
\[
G_C(s) \; = \; K \, \frac{s+z}{s+p}, 
\qquad \text{donde } 0 < p < z,
\]
tal que \emph{(i)} en circuito abierto el sistema tenga 45\deg de margen de fase y \emph{(ii)} en circuito cerrado el sistema tenga 1\% de error en estado estable ante una entrada escal\'on. 

\end{problem}
\fullskip

\end{document}
