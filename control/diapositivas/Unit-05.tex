% =================================================================
\documentclass[ 10pt, xcolor = dvipsnames]{beamer}
\usepackage{ beamerthemesplit, lmodern}
\usetheme{Madrid}
\usecolortheme[named=Brown]{structure}
\useinnertheme{rectangles}
\setbeamertemplate{frametitle continuation}{}
\beamertemplatenavigationsymbolsempty
\usepackage{../../macros-general}
\usepackage{../../macros-beamer}
\graphicspath{{./figures/}}

\newcommand{\theoremblock}[2]{
\begin{center}
\begin{minipage}{0.9\columnwidth}
\begin{block}{#1}
#2
\end{block}
\end{minipage}
\end{center}
}
\newcommand{\Laplace}{\mathcal{L}}
\newcommand{\LaplaceInv}{\Laplace^{-1}}

% =================================================================
\newcommand{\shorttitle}{Control Systems Engineering - Unit 03}
\title[\shorttitle]{Control Systems Engineering (EYAG-1005): \\ \textbf{Unit 03} }
\author[L. I. Reyes-Castro]{Luis I. Reyes-Castro}
\institute[ESPOL]{\normalsize Escuela Superior Polit\'ecnica del Litoral (ESPOL) \\ Guayaquil - Ecuador}
\date[2017-T1]{Semester: 2017-T1}

% -----------------------------------------------------------------
\begin{document}
\begin{frame}[noframenumbering]
\titlepage
\end{frame}
\begin{frame}[noframenumbering]
\frametitle{\shorttitle}
\tableofcontents[ subsectionstyle = hide]
\end{frame}

\AtBeginSection[]
{
\begin{frame}
\frametitle{Contenido del Tema}
\tableofcontents[ currentsection, sectionstyle = show/shaded, subsectionstyle = show/show/hide]
\end{frame}
}
\AtBeginSubsection[]
{
\begin{frame}
\frametitle{Contenido del Tema}
\tableofcontents[ currentsection, currentsubsection, sectionstyle = show/shaded, subsectionstyle = show/shaded/hide]
\end{frame}
}


% =================================================================
\section{Bode Plots}

% =================================================================
\subsection{Introduction}

% -----------------------------------------------------------------
\begin{frame}[allowframebreaks]
\frametitle{\insertsection}

Consider a stable system with transfer function $G(s)$. 
\begin{itemize}
\item Suppose the input $r(t)$ is a sinusoid with frequency $\omega$ and unit amplitude. 
\item Then the steady-state output $c_{ss}(t)$ must be a a sinusoid with the same frequency $\omega$ but with a particular amplitude $M$ and phase $\phi$ which depend \linebreak on the transfer function $G(s)$ and on the input frequency $\omega$. 
\end{itemize}
\halfskip

\begin{figure}[htb]
\centering
\def\svgwidth{0.8\columnwidth}
\input{figures/Bode-plot_definition.eps_tex}
\end{figure}
\framebreak

\begin{itemize}
\item Given $G(s)$ and a frequency $\omega$ we can evaluate the amplitude and phase of the steady-state output by computing the phasor $G(j\omega)$. In particular: 
\begin{itemize}
\item The phasor's magnitude yields the amplitude $M(\omega)$. 
\item The phasor's angle with $+\Re$ yields the phase $\phi(\omega)$. 
\end{itemize}
\halfskip
\begin{figure}[htb]
\centering
\def\svgwidth{0.72\columnwidth}
\input{figures/Bode-plot_phasor.eps_tex}
\end{figure}

\end{itemize}
\framebreak

\begin{itemize}
\item We can also estimate $M(\omega)$ and $\phi(\omega)$ experimentally: 
\halfskip
\begin{figure}[htb]
\centering
\def\svgwidth{0.8\columnwidth}
\input{figures/Bode-plot_sinewaves.eps_tex}
\end{figure}
\end{itemize}
\framebreak

\begin{itemize}
\item Furthermore, notice that: 
\begin{itemize}
\item Amplitude $M$ is always positive. 
\begin{itemize}
\item If $M \in (0,1)$ we get attenuation. 
\item If $M = 1$ we get amplitude matching. 
\item If $M > 1$ we get amplification. 
\end{itemize}
\item Phase may be negative, zero or postive. 
\begin{itemize}
\item If $\phi < 0$ then the output lags the input. 
\item If $\phi = 0$ then the output matches the input. 
\item If $\phi > 0$ then the output leads the input. 
\end{itemize}
\end{itemize}
\end{itemize}

\end{frame}

% -----------------------------------------------------------------
\begin{frame}[allowframebreaks]
\frametitle{\insertsection}

\textbf{Bode Plots} are diagrams of $M(\omega)$ and $\phi(\omega)$. More precisely, they consist of \linebreak the following two plots: 
\begin{itemize}
\item \textbf{Magnitude Plot}: Amplitude $M(\omega)$ versus frequency $\omega$. 
\begin{itemize}
\item The $x$-axis is frequency $\omega$ in decades, \ie $x = \log_{10}(\omega)$. 
\item The $y$-axis is amplitude $M(\omega)$ in decibels, \ie $y = 20 \cdot \log(M(\omega))$. 
\end{itemize}
\item \textbf{Phase Plot}: Phase $\phi(\omega)$ versus frequency $\omega$. 
\begin{itemize}
\item The $x$-axis is frequency $\omega$ in decades, \ie $x = \log_{10}(\omega)$. 
\item The $y$-axis is phase in degrees, \ie $y = \phi(\omega)$. 
\end{itemize}
\end{itemize}
\halfskip

Notice that when sketching Bode Plots by hand, we usually don't draw exactly \linebreak the functions $M(\omega)$ and $\phi(\omega)$ but instead sketch \emph{asymptotic approximations}. 

\end{frame}

% =================================================================
\subsection{Elementary Systems}

% -----------------------------------------------------------------
\begin{frame}[allowframebreaks]
\frametitle{\insertsection}

Bode plot for a simple amplifier: $G(s) = K$
\begin{itemize}
\item Magnitude plot is constant at $y = 20 \cdot \log_{10}(K)$ decibels. 
\item Phase plot is constant at $y = 0$ degree. 
\end{itemize}
\framebreak

Bode plot for an integrator: $\displaystyle G(s) = \frac{1}{s}$
\begin{itemize}
\item Phasor as a function of $\omega$ : 
\[
G(j\omega) \; = \; \frac{1}{j\omega} \; = \; -\frac{j}{\omega} 
\qquad \Longrightarrow \qquad
M(\omega) \; = \; \frac{1}{\omega} \quad \& \quad
\phi(\omega) \; = \; -90^{\circ}
\]
\item Magnitude plot is $y = -20 \cdot \log_{10}(\omega)$ decibels, \ie it is a line with slope of $-20$ decibels per decade which hits zero decibels at $\omega = 1$ rad/s. 
\item Phase plot is constant at $y = -90$ degree. 
\end{itemize}
\framebreak

Bode plot for a differentiator: $\displaystyle G(s) = s$
\begin{itemize}
\item Phasor as a function of $\omega$ : 
\[
G(j\omega) \; = \; j\omega \qquad \Longrightarrow \qquad
M(\omega) \; = \; \omega \quad \& \quad
\phi(\omega) \; = \; +90^{\circ}
\]
\item Magnitude plot is $y = +20 \cdot \log_{10}(\omega)$ decibels, \ie it is a line with slope of $+20$ decibels per decade which hits zero decibels at $\omega = 1$ rad/s. 
\item Phase plot is constant at $y = +90$ degree. 
\end{itemize}

\end{frame}

\end{document}
