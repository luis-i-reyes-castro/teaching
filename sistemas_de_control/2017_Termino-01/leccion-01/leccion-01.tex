% -----------------------------------------------------------------
% Document class: Article
\documentclass[ a4paper, twoside, 11pt]{article}
\usepackage{../../../macros-general}
\usepackage{../../../macros-article}
% Number of the handout, quiz, exam, etc.
\newcommand{\numero}{01}
\setcounter{numero}{\numero}

% -----------------------------------------------------------------
\begin{document}
\allowdisplaybreaks

\begin{center}
\Large Sistemas de Control (EYAG-1005): Lecci\'on \numero \\[1ex]
\small \textbf{Semestre:} 2017-2018 T\'ermino I \qquad
\textbf{Instructor:} Luis I. Reyes Castro
\end{center}
\halfskip

\fbox{

\begin{minipage}[b][\height][t]{\textwidth}
\vspace{0.2 cm}

\begin{center}
\textbf{COMPROMISO DE HONOR}
\end{center}
\vspace{0.4 cm}

\scriptsize
{
Yo, \rule{60mm}{.1pt} al firmar este compromiso, reconozco que la presente evaluaci\'on est\'a dise\~nada para ser resuelta de manera individual, que puedo usar un l\'apiz o pluma y una calculadora cient\'ifica, \linebreak que solo puedo comunicarme con la persona responsable de la recepci\'on de la evaluaci\'on, y que cualquier instrumento de comunicaci\'on que hubiere tra\'ido debo apagarlo. Tambi\'en estoy conciente que no debo consultar libros, notas, \linebreak ni materiales did\'acticos adicionales a los que el instructor entregue durante la evaluaci\'on o autorice a utilizar. Finalmente, me comprometo a desarrollar y presentar mis respuestas de manera clara y ordenada. \\

Firmo al pie del presente compromiso como constancia de haberlo le\'ido y aceptado. 
\vspace{0.4 cm}

Firma: \rule{60mm}{.1pt} \qquad N\'umero de matr\'icula: \rule{40mm}{.1pt} \hspace{0.5cm} \\[-0.8ex]
}

\end{minipage}

}

\vspace{\baselineskip}



% =============================================
\begin{problem}
\textbf{[10 Puntos]} Considere el siguiente modelo de un tubo de pitot electr\'onico montado en la punta de la nariz de un aeronave de alas fijas. El aire entra a una presi\'on $p(t)$, el cual desplaza un tope de masa $m$ y \'area frontal $A$ que esta conectado a la estructura por un resorte con constante $k_r$. Adem\'as el tope experimenta una fuerza de fricci\'on proporcional a su velocidad por una constante $k_f$. El tope hace girar un engranaje de radio $r_e$ que est\'a conectado a un micro-generador que produce un voltage proporcional a la velocidad angular del engranaje. El engranage es muy ligero, asi que podemos suponer que su momento de inercia es despreciable. Finalmente, el micro-generador est\'a conectado a un circuito RC, como se muestra abajo. 

\begin{figure}[htb]
\centering
\def\svgwidth{0.8\columnwidth}
\input{leccion-01_prob-01.eps_tex}
\end{figure}

Encuentre la funci\'on de transferencia cuya entrada es la presi\'on y cuya salida sea el voltage sobre el capacitor, \ie si denotamos como $v_C(t)$ al voltage del capacitor entonces: 
\[
G(s) \; = \; \frac{V_C(s)}{P(s)}
\]

\emph{Soluci\'on:} Definiendo la direcci\'on hacia la derecha como positiva, vemos que la sumatoria de fuerzas en el tope es 
\begin{equation}
m \, \ddot{x}(t) \, = \, 
A \, p(t) - k_r \, x(t) - k_f \, \dot{x}(t) \, ,
\label{eq:mecanico}
\end{equation}
donde el primer t\'ermino del lado derecho es la fuerza debida a la presi\'on, el segundo es la fuerza debida al resorte y el tercero es la fuerza debida a la fricci\'on. Adem\'as, la relaci\'on entre la velocidad del tope y la velocidad angular del engranaje es: 
\[
\dot{x}(t) \, = \, r_e \, \omega(t) \quad \Longleftrightarrow \quad
\omega(t) \, = \, (1/r_e) \, \dot{x}(t)
\]
A su vez, dado que el micro-generador produce un voltage $v(t) = k_\omega \, \omega(t)$, vemos que: 
\begin{equation}
v(t) \, = \, (k_\omega/r_e) \, \dot{x}(t)
\label{eq:engranaje}
\end{equation}
Luego analizamos el circuito. Como el \'unico elemento almacenador de energ\'ia es el capacitor, la \'unica ecuaci\'on de estado es $\dot{v}_C(t) \, = \, (1/C) \, i(t)$. Adem\'as, usando la Ley de Voltages de Kirchhoff vemos que: 
\[
v(t) \, = \, R \, i(t) + v_c(t) \quad \Longrightarrow \quad
i(t) \, = \, (1/R) \, ( \, v(t) - v_C(t) \, )
\]
Consecuentemente la ecuaci\'on de estado es: 
\begin{equation}
\dot{v}_C(t) \, = \, (1/RC) \, v(t) - (1/RC) \, v_C(t)
\label{eq:electrico}
\end{equation}

Despu\'es, tomamos la Transformaci\'on de Laplace de cada una de las tres ecuaciones diferenciales obtenidas anteriormente: 
\begin{align}
( \, m \, s^2 + k_f \, s + k_r \, ) \, X(s) \, 
& = \, A \, P(s) \\
V(s) \, & = \, (k_\omega/r_e) \, s \, X(s) \\
s \, V_C(s) \, & = \, (1/RC) \, ( \, V(s) - V_C(s) \, )
\end{align}
Ahora despejamos para la funci\'on de transferencia. Si empezamos con la \'ultima ecuaci\'on y reemplazamos $V(s)$ usando la segunda ecuaci\'on, obtenemos: 
\begin{align*}
& ( \, s + 1/RC \, ) \, V_C(s) \, = \, (1/RC) \, V(s)  \\
& \Longrightarrow \;
( \, s + 1/RC \, ) \, V_C(s) \, = \, (1/RC) \, (k_\omega/r_e) \, X(s)
\end{align*}
Finalmente, reemplazando para $X(s)$ usando la primera ecuaci\'on, tenemos: 
\[
( \, s + 1/RC \, ) \, V_C(s) \, = \,
\left( \frac{k_\omega}{r_e \, R \, C} \right)
\left( \frac{A}{m \, s^2 + k_f \, s + k_r} \right) P(s)
\]
En conclusi\'on: 
\[
G(s) \, = \, 
\frac{A \, k_\omega}{ ( \, r_e \, R \, C \, ) \, ( \, s + 1/RC \, ) \, ( \, m \, s^2 + k_f \, s + k_r \, )}
\]

\end{problem}
\vspace{\baselineskip}

\end{document}
